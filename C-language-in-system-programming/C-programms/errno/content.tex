Теперь поговорим об ошибках, возникающих при работе с системными вызовами, и о том, как можно узнать, почему тот или иной вызов не завершился успешно.

Системный вызов является либо вызовом инструкции int, либо вызовом инструкции syscall. При выполнение этой инструкции, система возвращает статус системного вызова через регистр АХ вызова (то есть код возврата). Помним, что неотрицательный код возврата, это успешное выполнение, отрицательный (чаще всего -1) --- ошибка. 

Однако, нам часто хочется узнать более точную причину произошедшей ошибки, а не просто факт ее возникновения. Для осуществления этой возможности и была придумана специальная \textbf{переменная errno}. Эта переменнтая содержится в заголовочном файле errno.h и имеет тип external int.

Помимо errno в <errno.h> хранятся также и директивы с определением ошибок и их кодов.


\begin{CCode}{Пример содержимого errno.h}
	...
	#define ENODEV      19  /* No such device */
	#define ENOTDIR     20  /* Not a directory */
	#define EISDIR      21  /* Is a directory */ 
	... \end{CCode}

\textbf{Как это работает?}

Дело в том, что в случае возникновения ошибки, системный вызов не только возвращает отрицательное значение. Он также устанавливает значение errno в зависимости от того, по какой причине эта самая ошибка случилась (если ошибки не произошло, значение errno не изменяется). 

\textbf{В чем еще преимущество errno?}

В разных ОС коды возврата системных вызовов они чуть-чуть отличаются друг от друга, поэтому если наша программа анализировала содержимое AX регистра, то мы бы устали для каждой ОС делать исключения. Libc предлагает нам унифицировать не только ввод-вывод, но и взаимодействие системных вызовов. 

Как это работает? Чтобы программистам на языке Си было проще, код возврата функции -- обертки системного вызова приводится к форме отрицательного числа уже средствами самой libC. 
