% made by: KorG
% vim: ft=tex et :
\documentclass[oneside]{book}

\usepackage[T1,T2A]{fontenc}       % fonts
\usepackage[utf8]{inputenc}     % UTF-8
\usepackage[english,russian]{babel}     % russian
\usepackage{cmap}               % russian search in pdf
\usepackage{pscyr}		        % PSCyr
\usepackage[14pt]{extsizes}     % for compatibility
\usepackage{float}		        % essential for [H]

\usepackage{indentfirst}        % first string indention

\usepackage{graphicx}           % graphics
\usepackage{ltxtable}           % tables

\usepackage{amsmath}            % math
\usepackage{nccmath}            % math medium sizes
\usepackage{amsfonts}           % math fonts
\usepackage{amssymb}            % math symb
\usepackage[a4paper, left=2cm, right=2cm, top=2cm,
        bottom=3cm]{geometry}   % pagesize

\usepackage{color}
\usepackage{xcolor}
\definecolor{light-gray}{gray}{0.9}

\usepackage{titlesec}
\usepackage{float}
\usepackage{multirow}
\usepackage{tabularx}
\usepackage{placeins}
\usepackage{totcount}
\usepackage[hidelinks,pdftex]{hyperref}
\usepackage[numbers,sort&compress]{natbib}
\usepackage{caption}
\usepackage{subcaption}
\frenchspacing
\setcounter{page}{1}
\bibliographystyle{utf8gost705u}
\makeatletter % [1] -> 1.
\renewcommand\@biblabel[1]{#1.}
\makeatother
\addto\captionsrussian{\def\refname{Библиографический список}}
\regtotcounter{page}
\regtotcounter{figure}
\regtotcounter{table}
\captionsetup{justification=centering,labelsep=period}
\captionsetup[table]{aboveskip=0pt,belowskip=0pt}

\usepackage{soul}               % \so{} & \ul{} - source and underline
\usepackage{soulutf8}           % UTF-8 for soul
\usepackage{verbatim}           % \verb{} and verbatim environment
\usepackage{listings}           % source code `bl
\lstset{
   escapeinside={\#@}{@},
        extendedchars=\true,
        numbers=left,
        inputencoding=utf8,
        keepspaces=true,
        basicstyle=\footnotesize\ttfamily,
        backgroundcolor=\color{light-gray},
        tabsize=3,
        breaklines=true,
        postbreak=\raisebox{0ex}[0ex][0ex]{\ensuremath{\color{red}\hookrightarrow\space}}
     }

\usepackage{fancyhdr}
\pagestyle{fancy}
\fancyhf{}                      % clear header and footer
\fancyfoot[C]{\thepage}         % page number
\renewcommand{\headrulewidth}{0pt}      % remove that line

\linespread{1.4}                % line interval ~1.5
\renewcommand{\rmdefault}{ftm}

\newcommand\header[3]{          % `H {sub}{*}{TITLE}
   \def \a {#1}
   \def \b {sub}
   \ifx\a\b
   \subsection#2{#3}
   \else
   \section#2{#3}
   \fi
   \ifx x#2x
   \else
   \addcontentsline{toc}{#1section}{#3}
   \fi
}

% ------------------------------
% Table of contents

\setcounter{secnumdepth}{4}
\usepackage{mathtext}
\renewcommand*{\familydefault}{\sfdefault} % clear font defaults
\usepackage{microtype}
\sloppy

% ------------------------------
% Page geometry & text spacing

\usepackage{parskip}
\setlength{\parindent}{0em}
\setlength{\parskip}{1em}

%% \usepackage{atbegshi} % http://ctan.org/pkg/atbegshi
%% \AtBeginDocument{\AtBeginShipoutNext{\AtBeginShipoutDiscard}}

\let\OLDitemize\itemize % itemize spacing
\renewcommand\itemize{\OLDitemize\addtolength{\itemsep}{-5pt}}

   \renewcommand{\arraystretch}{1.5}

   % ------------------------------
   % Code listings

\definecolor{mGreen}{rgb}{0,0.6,0}
\definecolor{mGray}{rgb}{0.5,0.5,0.5}
\definecolor{mPurple}{rgb}{0.58,0,0.82}
\definecolor{backgroundColour}{rgb}{0.95,0.96,0.97}


% C language codestyle
\lstdefinestyle{CStyle}{
   backgroundcolor=\color{backgroundColour},   
    commentstyle=\color{mGreen},
    keywordstyle=\color{blue},
    numberstyle=\tiny\color{mGray},
    stringstyle=\color{mPurple},
    basicstyle=\footnotesize,
    breakatwhitespace=false,         
    breaklines=true,                 
    captionpos=b,                    
    keepspaces=true,                 
    numbers=left,                    
    numbersep=5pt,                  
    showspaces=false,                
    showstringspaces=false,
    showtabs=false,                  
    tabsize=2,
    postbreak=\%\space,
    %belowskip=-1,
    language=C
 }

 % sh language codestyle
\lstdefinestyle{shStyle} {
   backgroundcolor=\color{backgroundColour},   
    commentstyle=\color{mGreen},
    keywordstyle=\color{magenta},
    numberstyle=\tiny\color{mGray},
    stringstyle=\color{mPurple},
    basicstyle=\footnotesize,
    breakatwhitespace=false,         
    breaklines=true,                 
    captionpos=b,                    
    keepspaces=true,                 
    numbers=left,                    
    numbersep=5pt,                  
    showspaces=false,                
    showstringspaces=false,
    showtabs=false,                  
    tabsize=2,
    postbreak=\%\space,
    morecomment=[s][\color{red}]{\ -}{\ },
    otherkeywords={$,@,>,<,\#,!,ELIF,IF,THEN,ELSE,FI,ESAC,CASE,WHILE,DO,UNTIL,FOR,echo,man,chmod},
    language=sh
 }
 %$$

 % listings titles style
\newenvironment{funci}[1]{{{\textbf{#1}}}}{}

% for C language code
\lstnewenvironment{CCode}[1]{
   \lstset{style=cStyle}
   \funci{#1}~}{
}

% for sh code
\lstnewenvironment{shCode}[1]{
   \lstset{style=shStyle}
   \funci{#1}~}{
}

% ------------------------------
% Special commands an environments


% for spaced environvent (example^ prompt string section)
\newenvironment{myenv}[2]{
   \renewcommand{\tabcolsep}{0cm}
   \begin{tabular}{p{0.20\linewidth}p{0.80\linewidth}}
      \textbf{#1} & #2
   \end{tabular}\\[0.5em]
   }{
      }

\usepackage{xpatch}
\makeatletter
\xpatchcmd{\@thm}{\thm@headpunct{.}}{\thm@headpunct{ --- \\}}{}{}
\makeatother

% for definitions
\newenvironment{defi}[1]{
   \textbf{#1} ---
      \\[0.3em]
      \begin{tabular}{p{0.1\linewidth}p{0.85\linewidth}}
         & 
         }
         {
      \end{tabular}
      }

      % for important addings	
\newenvironment{important}{\textbf{Важно: }{}}

\newenvironment{boxy}{
   \renewcommand{\arraystretch}{1.8}
   \begin{tabular}{|p{1\linewidth}|}
      \hline
      }
      {
         \\ \hline
   \end{tabular}
   }

\begin{document}

\cleardoublepage\clearpage


\input{titlepage.tex}

\clearpage

\tableofcontents

\clearpage 
\addcontentsline{toc}{chapter}{Благодарности}
\chapter*{Благодарности}
\input{acknowledgements.tex}

% ------------------------------
% Часть: Основы системного программирования

\clearpage \part{Системное программное обеспечение}
%Жизненный цикл процесса начинается с того, что его кто-то порождает. Единственный процесс, который никто не порождает, это процесс init. Порождение процесса осуществляется с помощью fork (2) или vfork (2). 

Первая стадия в жизни процесса --- \textbf{“инициализация“}. В это время ядро производит подготовительные работы к дальнейшей работе процесса. На самом деле, это не совсем состояние, но логически для пользователя оно существует.

\begin{figure}[htbp]
  \centering
  \includegraphics[width=0.7\textwidth]{./processes-and-threads/processes/lifecycle/proc-lifecycle.png}
\end{figure}

Когда инициализация процесса завершается --- он оказывается в стадии \textbf{“готов“}. С этого момента процесс находится в ожидании момента, когда его выберет планировщик процессов и даст ему какой-то квант процессорного времени.

При получении процессорного времени процесс переходит в состояние \textbf{“выполняется“}. Выполняться процесс может в режиме ядра --- при осуществлении системных вызовов, прерываний и в режиме задачи --- выполнять инструкции процессора. По окончании квоты времени, процесс может снова вернуться в состояние \textbf{“готов“}.

Также из состояния “выполняется“ процесс может перейти еще в два состояния --- \textbf{“остановлен“} и \textbf{“ожидает“}. Если он остановлен, то он остановлен пользователем (например, получен сигнал SIGSTOP). Ожидание наступает тогда, когда процессу нужны какие-то ресурсы. Как только условие, которого ждет процессор, выполняется, он переходит в состояние \textbf{“готов“}, то есть ожидает своей очереди на выполнение. 

Из каждого состояния процесс может перейти в состояние \textbf{“зомби“}. Этот процесс нужен для того, чтобы родительский процесс мог получить код возврата этого процесса. Это промежуточное состояние --- процесса уже нет, но он технически еще есть. Тем не менее, это самое что ни на есть валидное состояние. 

\textbf{В какой момент процесс окончательно исчезает из таблицы процессов?}

После того как его родительский процесс вызовет wait (2), waitpid (2), waitid(2) - позволяет завершить процесс, если он таки стал зомби. 

Традиционная проблема --- порождая процесс, необходимо в какой-то момент сказать ему wait (2), чтобы не плодить зомби. В противном случае у вас может закончится лимит на создание новых процессов. 

\textbf{Что происходит с зомби, если его родитель не вызвал wait (2), а погиб?}

Его родителем становится init. Для всех процессов, которые становятся его потомками он делает wait (2). По факту, если вы написали какой-то код, который выполняет некоторое количество процессов после чего завершился, то на самом деле зомби в системе не останутся, т.к. после того как ваш основной процесс завершился, то потомков этого процесса подхватит init и он сам вызовет для них wait (2).


   \chapter{Операционная система}

      \section{Роль операционной системы}
      Жизненный цикл процесса начинается с того, что его кто-то порождает. Единственный процесс, который никто не порождает, это процесс init. Порождение процесса осуществляется с помощью fork (2) или vfork (2). 

Первая стадия в жизни процесса --- \textbf{“инициализация“}. В это время ядро производит подготовительные работы к дальнейшей работе процесса. На самом деле, это не совсем состояние, но логически для пользователя оно существует.

\begin{figure}[htbp]
  \centering
  \includegraphics[width=0.7\textwidth]{./processes-and-threads/processes/lifecycle/proc-lifecycle.png}
\end{figure}

Когда инициализация процесса завершается --- он оказывается в стадии \textbf{“готов“}. С этого момента процесс находится в ожидании момента, когда его выберет планировщик процессов и даст ему какой-то квант процессорного времени.

При получении процессорного времени процесс переходит в состояние \textbf{“выполняется“}. Выполняться процесс может в режиме ядра --- при осуществлении системных вызовов, прерываний и в режиме задачи --- выполнять инструкции процессора. По окончании квоты времени, процесс может снова вернуться в состояние \textbf{“готов“}.

Также из состояния “выполняется“ процесс может перейти еще в два состояния --- \textbf{“остановлен“} и \textbf{“ожидает“}. Если он остановлен, то он остановлен пользователем (например, получен сигнал SIGSTOP). Ожидание наступает тогда, когда процессу нужны какие-то ресурсы. Как только условие, которого ждет процессор, выполняется, он переходит в состояние \textbf{“готов“}, то есть ожидает своей очереди на выполнение. 

Из каждого состояния процесс может перейти в состояние \textbf{“зомби“}. Этот процесс нужен для того, чтобы родительский процесс мог получить код возврата этого процесса. Это промежуточное состояние --- процесса уже нет, но он технически еще есть. Тем не менее, это самое что ни на есть валидное состояние. 

\textbf{В какой момент процесс окончательно исчезает из таблицы процессов?}

После того как его родительский процесс вызовет wait (2), waitpid (2), waitid(2) - позволяет завершить процесс, если он таки стал зомби. 

Традиционная проблема --- порождая процесс, необходимо в какой-то момент сказать ему wait (2), чтобы не плодить зомби. В противном случае у вас может закончится лимит на создание новых процессов. 

\textbf{Что происходит с зомби, если его родитель не вызвал wait (2), а погиб?}

Его родителем становится init. Для всех процессов, которые становятся его потомками он делает wait (2). По факту, если вы написали какой-то код, который выполняет некоторое количество процессов после чего завершился, то на самом деле зомби в системе не останутся, т.к. после того как ваш основной процесс завершился, то потомков этого процесса подхватит init и он сам вызовет для них wait (2).


      \section{Типовая ОС UNIX}
      Жизненный цикл процесса начинается с того, что его кто-то порождает. Единственный процесс, который никто не порождает, это процесс init. Порождение процесса осуществляется с помощью fork (2) или vfork (2). 

Первая стадия в жизни процесса --- \textbf{“инициализация“}. В это время ядро производит подготовительные работы к дальнейшей работе процесса. На самом деле, это не совсем состояние, но логически для пользователя оно существует.

\begin{figure}[htbp]
  \centering
  \includegraphics[width=0.7\textwidth]{./processes-and-threads/processes/lifecycle/proc-lifecycle.png}
\end{figure}

Когда инициализация процесса завершается --- он оказывается в стадии \textbf{“готов“}. С этого момента процесс находится в ожидании момента, когда его выберет планировщик процессов и даст ему какой-то квант процессорного времени.

При получении процессорного времени процесс переходит в состояние \textbf{“выполняется“}. Выполняться процесс может в режиме ядра --- при осуществлении системных вызовов, прерываний и в режиме задачи --- выполнять инструкции процессора. По окончании квоты времени, процесс может снова вернуться в состояние \textbf{“готов“}.

Также из состояния “выполняется“ процесс может перейти еще в два состояния --- \textbf{“остановлен“} и \textbf{“ожидает“}. Если он остановлен, то он остановлен пользователем (например, получен сигнал SIGSTOP). Ожидание наступает тогда, когда процессу нужны какие-то ресурсы. Как только условие, которого ждет процессор, выполняется, он переходит в состояние \textbf{“готов“}, то есть ожидает своей очереди на выполнение. 

Из каждого состояния процесс может перейти в состояние \textbf{“зомби“}. Этот процесс нужен для того, чтобы родительский процесс мог получить код возврата этого процесса. Это промежуточное состояние --- процесса уже нет, но он технически еще есть. Тем не менее, это самое что ни на есть валидное состояние. 

\textbf{В какой момент процесс окончательно исчезает из таблицы процессов?}

После того как его родительский процесс вызовет wait (2), waitpid (2), waitid(2) - позволяет завершить процесс, если он таки стал зомби. 

Традиционная проблема --- порождая процесс, необходимо в какой-то момент сказать ему wait (2), чтобы не плодить зомби. В противном случае у вас может закончится лимит на создание новых процессов. 

\textbf{Что происходит с зомби, если его родитель не вызвал wait (2), а погиб?}

Его родителем становится init. Для всех процессов, которые становятся его потомками он делает wait (2). По факту, если вы написали какой-то код, который выполняет некоторое количество процессов после чего завершился, то на самом деле зомби в системе не останутся, т.к. после того как ваш основной процесс завершился, то потомков этого процесса подхватит init и он сам вызовет для них wait (2).


      \section{Ядро типовой ОС UNIX}
      Жизненный цикл процесса начинается с того, что его кто-то порождает. Единственный процесс, который никто не порождает, это процесс init. Порождение процесса осуществляется с помощью fork (2) или vfork (2). 

Первая стадия в жизни процесса --- \textbf{“инициализация“}. В это время ядро производит подготовительные работы к дальнейшей работе процесса. На самом деле, это не совсем состояние, но логически для пользователя оно существует.

\begin{figure}[htbp]
  \centering
  \includegraphics[width=0.7\textwidth]{./processes-and-threads/processes/lifecycle/proc-lifecycle.png}
\end{figure}

Когда инициализация процесса завершается --- он оказывается в стадии \textbf{“готов“}. С этого момента процесс находится в ожидании момента, когда его выберет планировщик процессов и даст ему какой-то квант процессорного времени.

При получении процессорного времени процесс переходит в состояние \textbf{“выполняется“}. Выполняться процесс может в режиме ядра --- при осуществлении системных вызовов, прерываний и в режиме задачи --- выполнять инструкции процессора. По окончании квоты времени, процесс может снова вернуться в состояние \textbf{“готов“}.

Также из состояния “выполняется“ процесс может перейти еще в два состояния --- \textbf{“остановлен“} и \textbf{“ожидает“}. Если он остановлен, то он остановлен пользователем (например, получен сигнал SIGSTOP). Ожидание наступает тогда, когда процессу нужны какие-то ресурсы. Как только условие, которого ждет процессор, выполняется, он переходит в состояние \textbf{“готов“}, то есть ожидает своей очереди на выполнение. 

Из каждого состояния процесс может перейти в состояние \textbf{“зомби“}. Этот процесс нужен для того, чтобы родительский процесс мог получить код возврата этого процесса. Это промежуточное состояние --- процесса уже нет, но он технически еще есть. Тем не менее, это самое что ни на есть валидное состояние. 

\textbf{В какой момент процесс окончательно исчезает из таблицы процессов?}

После того как его родительский процесс вызовет wait (2), waitpid (2), waitid(2) - позволяет завершить процесс, если он таки стал зомби. 

Традиционная проблема --- порождая процесс, необходимо в какой-то момент сказать ему wait (2), чтобы не плодить зомби. В противном случае у вас может закончится лимит на создание новых процессов. 

\textbf{Что происходит с зомби, если его родитель не вызвал wait (2), а погиб?}

Его родителем становится init. Для всех процессов, которые становятся его потомками он делает wait (2). По факту, если вы написали какой-то код, который выполняет некоторое количество процессов после чего завершился, то на самом деле зомби в системе не останутся, т.к. после того как ваш основной процесс завершился, то потомков этого процесса подхватит init и он сам вызовет для них wait (2).


         \subsection{Подсистемы ядра}
         Ядро операционной системы UNIX состоит из следующих подсистем:

\begin{itemize}
	\item \textbf{Файловая подсистема}

		Эта подсистема определяет унифицированный интерфейс доступа к файлам. Отвечает за такие процессы, как создание, удаление, чтение и запись файлов. Также файловая подсистема обеспечивает контроль прав доступа к файлам.

	\item \textbf{Подсистема управления процессами и памятью}

		Подсистема управления процессами занимается планированием процессорного времени, инициализацией и завершением процессов. Отвечает за синхронизацию , а также за распредление системных ресурсов между процессами.

	\newpage
	\item \textbf{Подсистема ввода-вывода}

		Данная подсистема берет на себя взаимодействие с драйверами устройств и обеспечивает буферизацию данных.
\end{itemize}


         \subsection{Исключения, прерывания, системные вызовы}

            \subsubsection{Исключения}
            Жизненный цикл процесса начинается с того, что его кто-то порождает. Единственный процесс, который никто не порождает, это процесс init. Порождение процесса осуществляется с помощью fork (2) или vfork (2). 

Первая стадия в жизни процесса --- \textbf{“инициализация“}. В это время ядро производит подготовительные работы к дальнейшей работе процесса. На самом деле, это не совсем состояние, но логически для пользователя оно существует.

\begin{figure}[htbp]
  \centering
  \includegraphics[width=0.7\textwidth]{./processes-and-threads/processes/lifecycle/proc-lifecycle.png}
\end{figure}

Когда инициализация процесса завершается --- он оказывается в стадии \textbf{“готов“}. С этого момента процесс находится в ожидании момента, когда его выберет планировщик процессов и даст ему какой-то квант процессорного времени.

При получении процессорного времени процесс переходит в состояние \textbf{“выполняется“}. Выполняться процесс может в режиме ядра --- при осуществлении системных вызовов, прерываний и в режиме задачи --- выполнять инструкции процессора. По окончании квоты времени, процесс может снова вернуться в состояние \textbf{“готов“}.

Также из состояния “выполняется“ процесс может перейти еще в два состояния --- \textbf{“остановлен“} и \textbf{“ожидает“}. Если он остановлен, то он остановлен пользователем (например, получен сигнал SIGSTOP). Ожидание наступает тогда, когда процессу нужны какие-то ресурсы. Как только условие, которого ждет процессор, выполняется, он переходит в состояние \textbf{“готов“}, то есть ожидает своей очереди на выполнение. 

Из каждого состояния процесс может перейти в состояние \textbf{“зомби“}. Этот процесс нужен для того, чтобы родительский процесс мог получить код возврата этого процесса. Это промежуточное состояние --- процесса уже нет, но он технически еще есть. Тем не менее, это самое что ни на есть валидное состояние. 

\textbf{В какой момент процесс окончательно исчезает из таблицы процессов?}

После того как его родительский процесс вызовет wait (2), waitpid (2), waitid(2) - позволяет завершить процесс, если он таки стал зомби. 

Традиционная проблема --- порождая процесс, необходимо в какой-то момент сказать ему wait (2), чтобы не плодить зомби. В противном случае у вас может закончится лимит на создание новых процессов. 

\textbf{Что происходит с зомби, если его родитель не вызвал wait (2), а погиб?}

Его родителем становится init. Для всех процессов, которые становятся его потомками он делает wait (2). По факту, если вы написали какой-то код, который выполняет некоторое количество процессов после чего завершился, то на самом деле зомби в системе не останутся, т.к. после того как ваш основной процесс завершился, то потомков этого процесса подхватит init и он сам вызовет для них wait (2).


            \subsubsection{Прерывания}
            Жизненный цикл процесса начинается с того, что его кто-то порождает. Единственный процесс, который никто не порождает, это процесс init. Порождение процесса осуществляется с помощью fork (2) или vfork (2). 

Первая стадия в жизни процесса --- \textbf{“инициализация“}. В это время ядро производит подготовительные работы к дальнейшей работе процесса. На самом деле, это не совсем состояние, но логически для пользователя оно существует.

\begin{figure}[htbp]
  \centering
  \includegraphics[width=0.7\textwidth]{./processes-and-threads/processes/lifecycle/proc-lifecycle.png}
\end{figure}

Когда инициализация процесса завершается --- он оказывается в стадии \textbf{“готов“}. С этого момента процесс находится в ожидании момента, когда его выберет планировщик процессов и даст ему какой-то квант процессорного времени.

При получении процессорного времени процесс переходит в состояние \textbf{“выполняется“}. Выполняться процесс может в режиме ядра --- при осуществлении системных вызовов, прерываний и в режиме задачи --- выполнять инструкции процессора. По окончании квоты времени, процесс может снова вернуться в состояние \textbf{“готов“}.

Также из состояния “выполняется“ процесс может перейти еще в два состояния --- \textbf{“остановлен“} и \textbf{“ожидает“}. Если он остановлен, то он остановлен пользователем (например, получен сигнал SIGSTOP). Ожидание наступает тогда, когда процессу нужны какие-то ресурсы. Как только условие, которого ждет процессор, выполняется, он переходит в состояние \textbf{“готов“}, то есть ожидает своей очереди на выполнение. 

Из каждого состояния процесс может перейти в состояние \textbf{“зомби“}. Этот процесс нужен для того, чтобы родительский процесс мог получить код возврата этого процесса. Это промежуточное состояние --- процесса уже нет, но он технически еще есть. Тем не менее, это самое что ни на есть валидное состояние. 

\textbf{В какой момент процесс окончательно исчезает из таблицы процессов?}

После того как его родительский процесс вызовет wait (2), waitpid (2), waitid(2) - позволяет завершить процесс, если он таки стал зомби. 

Традиционная проблема --- порождая процесс, необходимо в какой-то момент сказать ему wait (2), чтобы не плодить зомби. В противном случае у вас может закончится лимит на создание новых процессов. 

\textbf{Что происходит с зомби, если его родитель не вызвал wait (2), а погиб?}

Его родителем становится init. Для всех процессов, которые становятся его потомками он делает wait (2). По факту, если вы написали какой-то код, который выполняет некоторое количество процессов после чего завершился, то на самом деле зомби в системе не останутся, т.к. после того как ваш основной процесс завершился, то потомков этого процесса подхватит init и он сам вызовет для них wait (2).


            \subsubsection{Системные вызовы.}
            Жизненный цикл процесса начинается с того, что его кто-то порождает. Единственный процесс, который никто не порождает, это процесс init. Порождение процесса осуществляется с помощью fork (2) или vfork (2). 

Первая стадия в жизни процесса --- \textbf{“инициализация“}. В это время ядро производит подготовительные работы к дальнейшей работе процесса. На самом деле, это не совсем состояние, но логически для пользователя оно существует.

\begin{figure}[htbp]
  \centering
  \includegraphics[width=0.7\textwidth]{./processes-and-threads/processes/lifecycle/proc-lifecycle.png}
\end{figure}

Когда инициализация процесса завершается --- он оказывается в стадии \textbf{“готов“}. С этого момента процесс находится в ожидании момента, когда его выберет планировщик процессов и даст ему какой-то квант процессорного времени.

При получении процессорного времени процесс переходит в состояние \textbf{“выполняется“}. Выполняться процесс может в режиме ядра --- при осуществлении системных вызовов, прерываний и в режиме задачи --- выполнять инструкции процессора. По окончании квоты времени, процесс может снова вернуться в состояние \textbf{“готов“}.

Также из состояния “выполняется“ процесс может перейти еще в два состояния --- \textbf{“остановлен“} и \textbf{“ожидает“}. Если он остановлен, то он остановлен пользователем (например, получен сигнал SIGSTOP). Ожидание наступает тогда, когда процессу нужны какие-то ресурсы. Как только условие, которого ждет процессор, выполняется, он переходит в состояние \textbf{“готов“}, то есть ожидает своей очереди на выполнение. 

Из каждого состояния процесс может перейти в состояние \textbf{“зомби“}. Этот процесс нужен для того, чтобы родительский процесс мог получить код возврата этого процесса. Это промежуточное состояние --- процесса уже нет, но он технически еще есть. Тем не менее, это самое что ни на есть валидное состояние. 

\textbf{В какой момент процесс окончательно исчезает из таблицы процессов?}

После того как его родительский процесс вызовет wait (2), waitpid (2), waitid(2) - позволяет завершить процесс, если он таки стал зомби. 

Традиционная проблема --- порождая процесс, необходимо в какой-то момент сказать ему wait (2), чтобы не плодить зомби. В противном случае у вас может закончится лимит на создание новых процессов. 

\textbf{Что происходит с зомби, если его родитель не вызвал wait (2), а погиб?}

Его родителем становится init. Для всех процессов, которые становятся его потомками он делает wait (2). По факту, если вы написали какой-то код, который выполняет некоторое количество процессов после чего завершился, то на самом деле зомби в системе не останутся, т.к. после того как ваш основной процесс завершился, то потомков этого процесса подхватит init и он сам вызовет для них wait (2).



   \chapter{Концепция системного программирования}
   Жизненный цикл процесса начинается с того, что его кто-то порождает. Единственный процесс, который никто не порождает, это процесс init. Порождение процесса осуществляется с помощью fork (2) или vfork (2). 

Первая стадия в жизни процесса --- \textbf{“инициализация“}. В это время ядро производит подготовительные работы к дальнейшей работе процесса. На самом деле, это не совсем состояние, но логически для пользователя оно существует.

\begin{figure}[htbp]
  \centering
  \includegraphics[width=0.7\textwidth]{./processes-and-threads/processes/lifecycle/proc-lifecycle.png}
\end{figure}

Когда инициализация процесса завершается --- он оказывается в стадии \textbf{“готов“}. С этого момента процесс находится в ожидании момента, когда его выберет планировщик процессов и даст ему какой-то квант процессорного времени.

При получении процессорного времени процесс переходит в состояние \textbf{“выполняется“}. Выполняться процесс может в режиме ядра --- при осуществлении системных вызовов, прерываний и в режиме задачи --- выполнять инструкции процессора. По окончании квоты времени, процесс может снова вернуться в состояние \textbf{“готов“}.

Также из состояния “выполняется“ процесс может перейти еще в два состояния --- \textbf{“остановлен“} и \textbf{“ожидает“}. Если он остановлен, то он остановлен пользователем (например, получен сигнал SIGSTOP). Ожидание наступает тогда, когда процессу нужны какие-то ресурсы. Как только условие, которого ждет процессор, выполняется, он переходит в состояние \textbf{“готов“}, то есть ожидает своей очереди на выполнение. 

Из каждого состояния процесс может перейти в состояние \textbf{“зомби“}. Этот процесс нужен для того, чтобы родительский процесс мог получить код возврата этого процесса. Это промежуточное состояние --- процесса уже нет, но он технически еще есть. Тем не менее, это самое что ни на есть валидное состояние. 

\textbf{В какой момент процесс окончательно исчезает из таблицы процессов?}

После того как его родительский процесс вызовет wait (2), waitpid (2), waitid(2) - позволяет завершить процесс, если он таки стал зомби. 

Традиционная проблема --- порождая процесс, необходимо в какой-то момент сказать ему wait (2), чтобы не плодить зомби. В противном случае у вас может закончится лимит на создание новых процессов. 

\textbf{Что происходит с зомби, если его родитель не вызвал wait (2), а погиб?}

Его родителем становится init. Для всех процессов, которые становятся его потомками он делает wait (2). По факту, если вы написали какой-то код, который выполняет некоторое количество процессов после чего завершился, то на самом деле зомби в системе не останутся, т.к. после того как ваш основной процесс завершился, то потомков этого процесса подхватит init и он сам вызовет для них wait (2).


      \section{Языки системного программирования}
      Жизненный цикл процесса начинается с того, что его кто-то порождает. Единственный процесс, который никто не порождает, это процесс init. Порождение процесса осуществляется с помощью fork (2) или vfork (2). 

Первая стадия в жизни процесса --- \textbf{“инициализация“}. В это время ядро производит подготовительные работы к дальнейшей работе процесса. На самом деле, это не совсем состояние, но логически для пользователя оно существует.

\begin{figure}[htbp]
  \centering
  \includegraphics[width=0.7\textwidth]{./processes-and-threads/processes/lifecycle/proc-lifecycle.png}
\end{figure}

Когда инициализация процесса завершается --- он оказывается в стадии \textbf{“готов“}. С этого момента процесс находится в ожидании момента, когда его выберет планировщик процессов и даст ему какой-то квант процессорного времени.

При получении процессорного времени процесс переходит в состояние \textbf{“выполняется“}. Выполняться процесс может в режиме ядра --- при осуществлении системных вызовов, прерываний и в режиме задачи --- выполнять инструкции процессора. По окончании квоты времени, процесс может снова вернуться в состояние \textbf{“готов“}.

Также из состояния “выполняется“ процесс может перейти еще в два состояния --- \textbf{“остановлен“} и \textbf{“ожидает“}. Если он остановлен, то он остановлен пользователем (например, получен сигнал SIGSTOP). Ожидание наступает тогда, когда процессу нужны какие-то ресурсы. Как только условие, которого ждет процессор, выполняется, он переходит в состояние \textbf{“готов“}, то есть ожидает своей очереди на выполнение. 

Из каждого состояния процесс может перейти в состояние \textbf{“зомби“}. Этот процесс нужен для того, чтобы родительский процесс мог получить код возврата этого процесса. Это промежуточное состояние --- процесса уже нет, но он технически еще есть. Тем не менее, это самое что ни на есть валидное состояние. 

\textbf{В какой момент процесс окончательно исчезает из таблицы процессов?}

После того как его родительский процесс вызовет wait (2), waitpid (2), waitid(2) - позволяет завершить процесс, если он таки стал зомби. 

Традиционная проблема --- порождая процесс, необходимо в какой-то момент сказать ему wait (2), чтобы не плодить зомби. В противном случае у вас может закончится лимит на создание новых процессов. 

\textbf{Что происходит с зомби, если его родитель не вызвал wait (2), а погиб?}

Его родителем становится init. Для всех процессов, которые становятся его потомками он делает wait (2). По факту, если вы написали какой-то код, который выполняет некоторое количество процессов после чего завершился, то на самом деле зомби в системе не останутся, т.к. после того как ваш основной процесс завершился, то потомков этого процесса подхватит init и он сам вызовет для них wait (2).


      \section{Системные вызовы}
      Жизненный цикл процесса начинается с того, что его кто-то порождает. Единственный процесс, который никто не порождает, это процесс init. Порождение процесса осуществляется с помощью fork (2) или vfork (2). 

Первая стадия в жизни процесса --- \textbf{“инициализация“}. В это время ядро производит подготовительные работы к дальнейшей работе процесса. На самом деле, это не совсем состояние, но логически для пользователя оно существует.

\begin{figure}[htbp]
  \centering
  \includegraphics[width=0.7\textwidth]{./processes-and-threads/processes/lifecycle/proc-lifecycle.png}
\end{figure}

Когда инициализация процесса завершается --- он оказывается в стадии \textbf{“готов“}. С этого момента процесс находится в ожидании момента, когда его выберет планировщик процессов и даст ему какой-то квант процессорного времени.

При получении процессорного времени процесс переходит в состояние \textbf{“выполняется“}. Выполняться процесс может в режиме ядра --- при осуществлении системных вызовов, прерываний и в режиме задачи --- выполнять инструкции процессора. По окончании квоты времени, процесс может снова вернуться в состояние \textbf{“готов“}.

Также из состояния “выполняется“ процесс может перейти еще в два состояния --- \textbf{“остановлен“} и \textbf{“ожидает“}. Если он остановлен, то он остановлен пользователем (например, получен сигнал SIGSTOP). Ожидание наступает тогда, когда процессу нужны какие-то ресурсы. Как только условие, которого ждет процессор, выполняется, он переходит в состояние \textbf{“готов“}, то есть ожидает своей очереди на выполнение. 

Из каждого состояния процесс может перейти в состояние \textbf{“зомби“}. Этот процесс нужен для того, чтобы родительский процесс мог получить код возврата этого процесса. Это промежуточное состояние --- процесса уже нет, но он технически еще есть. Тем не менее, это самое что ни на есть валидное состояние. 

\textbf{В какой момент процесс окончательно исчезает из таблицы процессов?}

После того как его родительский процесс вызовет wait (2), waitpid (2), waitid(2) - позволяет завершить процесс, если он таки стал зомби. 

Традиционная проблема --- порождая процесс, необходимо в какой-то момент сказать ему wait (2), чтобы не плодить зомби. В противном случае у вас может закончится лимит на создание новых процессов. 

\textbf{Что происходит с зомби, если его родитель не вызвал wait (2), а погиб?}

Его родителем становится init. Для всех процессов, которые становятся его потомками он делает wait (2). По факту, если вы написали какой-то код, который выполняет некоторое количество процессов после чего завершился, то на самом деле зомби в системе не останутся, т.к. после того как ваш основной процесс завершился, то потомков этого процесса подхватит init и он сам вызовет для них wait (2).


      \section{Ввод-вывод}
      Ввод-вывод включает в себя управление обменом данных между памятью и периферией(диски,терминалы). Взаимодействие ядра с аппаратными компонентами происходит через драйверы устройств. С помощью драйвера происходит управление одним или несколькими устройствами, представляя собой интерфейс между устройством и остальной частью ядра.


      \section{Процессы и потоки}
      Потоки и процессы являются связанными понятиями. Представляют из себя последовательности инструкций, выполняющиеся в определенном порядке. Процесс всегда будет состоять хотя бы из одного потока. Хоть поток и должен выполняться внутри процесса, необходимо различать их концепции. Процесс группируют ресурсы, в то время как, потоки - это объекты, которые поочередно исполняются на ЦП.


   \chapter{Работа в ОС UNIX}

      \section{Терминал}
      Жизненный цикл процесса начинается с того, что его кто-то порождает. Единственный процесс, который никто не порождает, это процесс init. Порождение процесса осуществляется с помощью fork (2) или vfork (2). 

Первая стадия в жизни процесса --- \textbf{“инициализация“}. В это время ядро производит подготовительные работы к дальнейшей работе процесса. На самом деле, это не совсем состояние, но логически для пользователя оно существует.

\begin{figure}[htbp]
  \centering
  \includegraphics[width=0.7\textwidth]{./processes-and-threads/processes/lifecycle/proc-lifecycle.png}
\end{figure}

Когда инициализация процесса завершается --- он оказывается в стадии \textbf{“готов“}. С этого момента процесс находится в ожидании момента, когда его выберет планировщик процессов и даст ему какой-то квант процессорного времени.

При получении процессорного времени процесс переходит в состояние \textbf{“выполняется“}. Выполняться процесс может в режиме ядра --- при осуществлении системных вызовов, прерываний и в режиме задачи --- выполнять инструкции процессора. По окончании квоты времени, процесс может снова вернуться в состояние \textbf{“готов“}.

Также из состояния “выполняется“ процесс может перейти еще в два состояния --- \textbf{“остановлен“} и \textbf{“ожидает“}. Если он остановлен, то он остановлен пользователем (например, получен сигнал SIGSTOP). Ожидание наступает тогда, когда процессу нужны какие-то ресурсы. Как только условие, которого ждет процессор, выполняется, он переходит в состояние \textbf{“готов“}, то есть ожидает своей очереди на выполнение. 

Из каждого состояния процесс может перейти в состояние \textbf{“зомби“}. Этот процесс нужен для того, чтобы родительский процесс мог получить код возврата этого процесса. Это промежуточное состояние --- процесса уже нет, но он технически еще есть. Тем не менее, это самое что ни на есть валидное состояние. 

\textbf{В какой момент процесс окончательно исчезает из таблицы процессов?}

После того как его родительский процесс вызовет wait (2), waitpid (2), waitid(2) - позволяет завершить процесс, если он таки стал зомби. 

Традиционная проблема --- порождая процесс, необходимо в какой-то момент сказать ему wait (2), чтобы не плодить зомби. В противном случае у вас может закончится лимит на создание новых процессов. 

\textbf{Что происходит с зомби, если его родитель не вызвал wait (2), а погиб?}

Его родителем становится init. Для всех процессов, которые становятся его потомками он делает wait (2). По факту, если вы написали какой-то код, который выполняет некоторое количество процессов после чего завершился, то на самом деле зомби в системе не останутся, т.к. после того как ваш основной процесс завершился, то потомков этого процесса подхватит init и он сам вызовет для них wait (2).


         \subsection{Канонический и неканонический режимы ввода}
         Жизненный цикл процесса начинается с того, что его кто-то порождает. Единственный процесс, который никто не порождает, это процесс init. Порождение процесса осуществляется с помощью fork (2) или vfork (2). 

Первая стадия в жизни процесса --- \textbf{“инициализация“}. В это время ядро производит подготовительные работы к дальнейшей работе процесса. На самом деле, это не совсем состояние, но логически для пользователя оно существует.

\begin{figure}[htbp]
  \centering
  \includegraphics[width=0.7\textwidth]{./processes-and-threads/processes/lifecycle/proc-lifecycle.png}
\end{figure}

Когда инициализация процесса завершается --- он оказывается в стадии \textbf{“готов“}. С этого момента процесс находится в ожидании момента, когда его выберет планировщик процессов и даст ему какой-то квант процессорного времени.

При получении процессорного времени процесс переходит в состояние \textbf{“выполняется“}. Выполняться процесс может в режиме ядра --- при осуществлении системных вызовов, прерываний и в режиме задачи --- выполнять инструкции процессора. По окончании квоты времени, процесс может снова вернуться в состояние \textbf{“готов“}.

Также из состояния “выполняется“ процесс может перейти еще в два состояния --- \textbf{“остановлен“} и \textbf{“ожидает“}. Если он остановлен, то он остановлен пользователем (например, получен сигнал SIGSTOP). Ожидание наступает тогда, когда процессу нужны какие-то ресурсы. Как только условие, которого ждет процессор, выполняется, он переходит в состояние \textbf{“готов“}, то есть ожидает своей очереди на выполнение. 

Из каждого состояния процесс может перейти в состояние \textbf{“зомби“}. Этот процесс нужен для того, чтобы родительский процесс мог получить код возврата этого процесса. Это промежуточное состояние --- процесса уже нет, но он технически еще есть. Тем не менее, это самое что ни на есть валидное состояние. 

\textbf{В какой момент процесс окончательно исчезает из таблицы процессов?}

После того как его родительский процесс вызовет wait (2), waitpid (2), waitid(2) - позволяет завершить процесс, если он таки стал зомби. 

Традиционная проблема --- порождая процесс, необходимо в какой-то момент сказать ему wait (2), чтобы не плодить зомби. В противном случае у вас может закончится лимит на создание новых процессов. 

\textbf{Что происходит с зомби, если его родитель не вызвал wait (2), а погиб?}

Его родителем становится init. Для всех процессов, которые становятся его потомками он делает wait (2). По факту, если вы написали какой-то код, который выполняет некоторое количество процессов после чего завершился, то на самом деле зомби в системе не останутся, т.к. после того как ваш основной процесс завершился, то потомков этого процесса подхватит init и он сам вызовет для них wait (2).


      \section{Командный интерпретатор}
      Жизненный цикл процесса начинается с того, что его кто-то порождает. Единственный процесс, который никто не порождает, это процесс init. Порождение процесса осуществляется с помощью fork (2) или vfork (2). 

Первая стадия в жизни процесса --- \textbf{“инициализация“}. В это время ядро производит подготовительные работы к дальнейшей работе процесса. На самом деле, это не совсем состояние, но логически для пользователя оно существует.

\begin{figure}[htbp]
  \centering
  \includegraphics[width=0.7\textwidth]{./processes-and-threads/processes/lifecycle/proc-lifecycle.png}
\end{figure}

Когда инициализация процесса завершается --- он оказывается в стадии \textbf{“готов“}. С этого момента процесс находится в ожидании момента, когда его выберет планировщик процессов и даст ему какой-то квант процессорного времени.

При получении процессорного времени процесс переходит в состояние \textbf{“выполняется“}. Выполняться процесс может в режиме ядра --- при осуществлении системных вызовов, прерываний и в режиме задачи --- выполнять инструкции процессора. По окончании квоты времени, процесс может снова вернуться в состояние \textbf{“готов“}.

Также из состояния “выполняется“ процесс может перейти еще в два состояния --- \textbf{“остановлен“} и \textbf{“ожидает“}. Если он остановлен, то он остановлен пользователем (например, получен сигнал SIGSTOP). Ожидание наступает тогда, когда процессу нужны какие-то ресурсы. Как только условие, которого ждет процессор, выполняется, он переходит в состояние \textbf{“готов“}, то есть ожидает своей очереди на выполнение. 

Из каждого состояния процесс может перейти в состояние \textbf{“зомби“}. Этот процесс нужен для того, чтобы родительский процесс мог получить код возврата этого процесса. Это промежуточное состояние --- процесса уже нет, но он технически еще есть. Тем не менее, это самое что ни на есть валидное состояние. 

\textbf{В какой момент процесс окончательно исчезает из таблицы процессов?}

После того как его родительский процесс вызовет wait (2), waitpid (2), waitid(2) - позволяет завершить процесс, если он таки стал зомби. 

Традиционная проблема --- порождая процесс, необходимо в какой-то момент сказать ему wait (2), чтобы не плодить зомби. В противном случае у вас может закончится лимит на создание новых процессов. 

\textbf{Что происходит с зомби, если его родитель не вызвал wait (2), а погиб?}

Его родителем становится init. Для всех процессов, которые становятся его потомками он делает wait (2). По факту, если вы написали какой-то код, который выполняет некоторое количество процессов после чего завершился, то на самом деле зомби в системе не останутся, т.к. после того как ваш основной процесс завершился, то потомков этого процесса подхватит init и он сам вызовет для них wait (2).


      \section{Программа, утилита, команда}
      Жизненный цикл процесса начинается с того, что его кто-то порождает. Единственный процесс, который никто не порождает, это процесс init. Порождение процесса осуществляется с помощью fork (2) или vfork (2). 

Первая стадия в жизни процесса --- \textbf{“инициализация“}. В это время ядро производит подготовительные работы к дальнейшей работе процесса. На самом деле, это не совсем состояние, но логически для пользователя оно существует.

\begin{figure}[htbp]
  \centering
  \includegraphics[width=0.7\textwidth]{./processes-and-threads/processes/lifecycle/proc-lifecycle.png}
\end{figure}

Когда инициализация процесса завершается --- он оказывается в стадии \textbf{“готов“}. С этого момента процесс находится в ожидании момента, когда его выберет планировщик процессов и даст ему какой-то квант процессорного времени.

При получении процессорного времени процесс переходит в состояние \textbf{“выполняется“}. Выполняться процесс может в режиме ядра --- при осуществлении системных вызовов, прерываний и в режиме задачи --- выполнять инструкции процессора. По окончании квоты времени, процесс может снова вернуться в состояние \textbf{“готов“}.

Также из состояния “выполняется“ процесс может перейти еще в два состояния --- \textbf{“остановлен“} и \textbf{“ожидает“}. Если он остановлен, то он остановлен пользователем (например, получен сигнал SIGSTOP). Ожидание наступает тогда, когда процессу нужны какие-то ресурсы. Как только условие, которого ждет процессор, выполняется, он переходит в состояние \textbf{“готов“}, то есть ожидает своей очереди на выполнение. 

Из каждого состояния процесс может перейти в состояние \textbf{“зомби“}. Этот процесс нужен для того, чтобы родительский процесс мог получить код возврата этого процесса. Это промежуточное состояние --- процесса уже нет, но он технически еще есть. Тем не менее, это самое что ни на есть валидное состояние. 

\textbf{В какой момент процесс окончательно исчезает из таблицы процессов?}

После того как его родительский процесс вызовет wait (2), waitpid (2), waitid(2) - позволяет завершить процесс, если он таки стал зомби. 

Традиционная проблема --- порождая процесс, необходимо в какой-то момент сказать ему wait (2), чтобы не плодить зомби. В противном случае у вас может закончится лимит на создание новых процессов. 

\textbf{Что происходит с зомби, если его родитель не вызвал wait (2), а погиб?}

Его родителем становится init. Для всех процессов, которые становятся его потомками он делает wait (2). По факту, если вы написали какой-то код, который выполняет некоторое количество процессов после чего завершился, то на самом деле зомби в системе не останутся, т.к. после того как ваш основной процесс завершился, то потомков этого процесса подхватит init и он сам вызовет для них wait (2).



      % ------------------------------
      % Часть: Командный интерпретарор shell

\clearpage \part{Командный интерпретатор shell}
%Жизненный цикл процесса начинается с того, что его кто-то порождает. Единственный процесс, который никто не порождает, это процесс init. Порождение процесса осуществляется с помощью fork (2) или vfork (2). 

Первая стадия в жизни процесса --- \textbf{“инициализация“}. В это время ядро производит подготовительные работы к дальнейшей работе процесса. На самом деле, это не совсем состояние, но логически для пользователя оно существует.

\begin{figure}[htbp]
  \centering
  \includegraphics[width=0.7\textwidth]{./processes-and-threads/processes/lifecycle/proc-lifecycle.png}
\end{figure}

Когда инициализация процесса завершается --- он оказывается в стадии \textbf{“готов“}. С этого момента процесс находится в ожидании момента, когда его выберет планировщик процессов и даст ему какой-то квант процессорного времени.

При получении процессорного времени процесс переходит в состояние \textbf{“выполняется“}. Выполняться процесс может в режиме ядра --- при осуществлении системных вызовов, прерываний и в режиме задачи --- выполнять инструкции процессора. По окончании квоты времени, процесс может снова вернуться в состояние \textbf{“готов“}.

Также из состояния “выполняется“ процесс может перейти еще в два состояния --- \textbf{“остановлен“} и \textbf{“ожидает“}. Если он остановлен, то он остановлен пользователем (например, получен сигнал SIGSTOP). Ожидание наступает тогда, когда процессу нужны какие-то ресурсы. Как только условие, которого ждет процессор, выполняется, он переходит в состояние \textbf{“готов“}, то есть ожидает своей очереди на выполнение. 

Из каждого состояния процесс может перейти в состояние \textbf{“зомби“}. Этот процесс нужен для того, чтобы родительский процесс мог получить код возврата этого процесса. Это промежуточное состояние --- процесса уже нет, но он технически еще есть. Тем не менее, это самое что ни на есть валидное состояние. 

\textbf{В какой момент процесс окончательно исчезает из таблицы процессов?}

После того как его родительский процесс вызовет wait (2), waitpid (2), waitid(2) - позволяет завершить процесс, если он таки стал зомби. 

Традиционная проблема --- порождая процесс, необходимо в какой-то момент сказать ему wait (2), чтобы не плодить зомби. В противном случае у вас может закончится лимит на создание новых процессов. 

\textbf{Что происходит с зомби, если его родитель не вызвал wait (2), а погиб?}

Его родителем становится init. Для всех процессов, которые становятся его потомками он делает wait (2). По факту, если вы написали какой-то код, который выполняет некоторое количество процессов после чего завершился, то на самом деле зомби в системе не останутся, т.к. после того как ваш основной процесс завершился, то потомков этого процесса подхватит init и он сам вызовет для них wait (2).


   \chapter{Основные понятия}

      \section{Командный интерпретатор}
      Жизненный цикл процесса начинается с того, что его кто-то порождает. Единственный процесс, который никто не порождает, это процесс init. Порождение процесса осуществляется с помощью fork (2) или vfork (2). 

Первая стадия в жизни процесса --- \textbf{“инициализация“}. В это время ядро производит подготовительные работы к дальнейшей работе процесса. На самом деле, это не совсем состояние, но логически для пользователя оно существует.

\begin{figure}[htbp]
  \centering
  \includegraphics[width=0.7\textwidth]{./processes-and-threads/processes/lifecycle/proc-lifecycle.png}
\end{figure}

Когда инициализация процесса завершается --- он оказывается в стадии \textbf{“готов“}. С этого момента процесс находится в ожидании момента, когда его выберет планировщик процессов и даст ему какой-то квант процессорного времени.

При получении процессорного времени процесс переходит в состояние \textbf{“выполняется“}. Выполняться процесс может в режиме ядра --- при осуществлении системных вызовов, прерываний и в режиме задачи --- выполнять инструкции процессора. По окончании квоты времени, процесс может снова вернуться в состояние \textbf{“готов“}.

Также из состояния “выполняется“ процесс может перейти еще в два состояния --- \textbf{“остановлен“} и \textbf{“ожидает“}. Если он остановлен, то он остановлен пользователем (например, получен сигнал SIGSTOP). Ожидание наступает тогда, когда процессу нужны какие-то ресурсы. Как только условие, которого ждет процессор, выполняется, он переходит в состояние \textbf{“готов“}, то есть ожидает своей очереди на выполнение. 

Из каждого состояния процесс может перейти в состояние \textbf{“зомби“}. Этот процесс нужен для того, чтобы родительский процесс мог получить код возврата этого процесса. Это промежуточное состояние --- процесса уже нет, но он технически еще есть. Тем не менее, это самое что ни на есть валидное состояние. 

\textbf{В какой момент процесс окончательно исчезает из таблицы процессов?}

После того как его родительский процесс вызовет wait (2), waitpid (2), waitid(2) - позволяет завершить процесс, если он таки стал зомби. 

Традиционная проблема --- порождая процесс, необходимо в какой-то момент сказать ему wait (2), чтобы не плодить зомби. В противном случае у вас может закончится лимит на создание новых процессов. 

\textbf{Что происходит с зомби, если его родитель не вызвал wait (2), а погиб?}

Его родителем становится init. Для всех процессов, которые становятся его потомками он делает wait (2). По факту, если вы написали какой-то код, который выполняет некоторое количество процессов после чего завершился, то на самом деле зомби в системе не останутся, т.к. после того как ваш основной процесс завершился, то потомков этого процесса подхватит init и он сам вызовет для них wait (2).


         \subsection{Приглашения командной строки (prompt string)}
         Жизненный цикл процесса начинается с того, что его кто-то порождает. Единственный процесс, который никто не порождает, это процесс init. Порождение процесса осуществляется с помощью fork (2) или vfork (2). 

Первая стадия в жизни процесса --- \textbf{“инициализация“}. В это время ядро производит подготовительные работы к дальнейшей работе процесса. На самом деле, это не совсем состояние, но логически для пользователя оно существует.

\begin{figure}[htbp]
  \centering
  \includegraphics[width=0.7\textwidth]{./processes-and-threads/processes/lifecycle/proc-lifecycle.png}
\end{figure}

Когда инициализация процесса завершается --- он оказывается в стадии \textbf{“готов“}. С этого момента процесс находится в ожидании момента, когда его выберет планировщик процессов и даст ему какой-то квант процессорного времени.

При получении процессорного времени процесс переходит в состояние \textbf{“выполняется“}. Выполняться процесс может в режиме ядра --- при осуществлении системных вызовов, прерываний и в режиме задачи --- выполнять инструкции процессора. По окончании квоты времени, процесс может снова вернуться в состояние \textbf{“готов“}.

Также из состояния “выполняется“ процесс может перейти еще в два состояния --- \textbf{“остановлен“} и \textbf{“ожидает“}. Если он остановлен, то он остановлен пользователем (например, получен сигнал SIGSTOP). Ожидание наступает тогда, когда процессу нужны какие-то ресурсы. Как только условие, которого ждет процессор, выполняется, он переходит в состояние \textbf{“готов“}, то есть ожидает своей очереди на выполнение. 

Из каждого состояния процесс может перейти в состояние \textbf{“зомби“}. Этот процесс нужен для того, чтобы родительский процесс мог получить код возврата этого процесса. Это промежуточное состояние --- процесса уже нет, но он технически еще есть. Тем не менее, это самое что ни на есть валидное состояние. 

\textbf{В какой момент процесс окончательно исчезает из таблицы процессов?}

После того как его родительский процесс вызовет wait (2), waitpid (2), waitid(2) - позволяет завершить процесс, если он таки стал зомби. 

Традиционная проблема --- порождая процесс, необходимо в какой-то момент сказать ему wait (2), чтобы не плодить зомби. В противном случае у вас может закончится лимит на создание новых процессов. 

\textbf{Что происходит с зомби, если его родитель не вызвал wait (2), а погиб?}

Его родителем становится init. Для всех процессов, которые становятся его потомками он делает wait (2). По факту, если вы написали какой-то код, который выполняет некоторое количество процессов после чего завершился, то на самом деле зомби в системе не останутся, т.к. после того как ваш основной процесс завершился, то потомков этого процесса подхватит init и он сам вызовет для них wait (2).
			

      \section{Принципы исполнения инструкций}

         \subsection{IFS}
         Жизненный цикл процесса начинается с того, что его кто-то порождает. Единственный процесс, который никто не порождает, это процесс init. Порождение процесса осуществляется с помощью fork (2) или vfork (2). 

Первая стадия в жизни процесса --- \textbf{“инициализация“}. В это время ядро производит подготовительные работы к дальнейшей работе процесса. На самом деле, это не совсем состояние, но логически для пользователя оно существует.

\begin{figure}[htbp]
  \centering
  \includegraphics[width=0.7\textwidth]{./processes-and-threads/processes/lifecycle/proc-lifecycle.png}
\end{figure}

Когда инициализация процесса завершается --- он оказывается в стадии \textbf{“готов“}. С этого момента процесс находится в ожидании момента, когда его выберет планировщик процессов и даст ему какой-то квант процессорного времени.

При получении процессорного времени процесс переходит в состояние \textbf{“выполняется“}. Выполняться процесс может в режиме ядра --- при осуществлении системных вызовов, прерываний и в режиме задачи --- выполнять инструкции процессора. По окончании квоты времени, процесс может снова вернуться в состояние \textbf{“готов“}.

Также из состояния “выполняется“ процесс может перейти еще в два состояния --- \textbf{“остановлен“} и \textbf{“ожидает“}. Если он остановлен, то он остановлен пользователем (например, получен сигнал SIGSTOP). Ожидание наступает тогда, когда процессу нужны какие-то ресурсы. Как только условие, которого ждет процессор, выполняется, он переходит в состояние \textbf{“готов“}, то есть ожидает своей очереди на выполнение. 

Из каждого состояния процесс может перейти в состояние \textbf{“зомби“}. Этот процесс нужен для того, чтобы родительский процесс мог получить код возврата этого процесса. Это промежуточное состояние --- процесса уже нет, но он технически еще есть. Тем не менее, это самое что ни на есть валидное состояние. 

\textbf{В какой момент процесс окончательно исчезает из таблицы процессов?}

После того как его родительский процесс вызовет wait (2), waitpid (2), waitid(2) - позволяет завершить процесс, если он таки стал зомби. 

Традиционная проблема --- порождая процесс, необходимо в какой-то момент сказать ему wait (2), чтобы не плодить зомби. В противном случае у вас может закончится лимит на создание новых процессов. 

\textbf{Что происходит с зомби, если его родитель не вызвал wait (2), а погиб?}

Его родителем становится init. Для всех процессов, которые становятся его потомками он делает wait (2). По факту, если вы написали какой-то код, который выполняет некоторое количество процессов после чего завершился, то на самом деле зомби в системе не останутся, т.к. после того как ваш основной процесс завершился, то потомков этого процесса подхватит init и он сам вызовет для них wait (2).


         \subsection{Группировка команд и запуск команд в subshell}
         Жизненный цикл процесса начинается с того, что его кто-то порождает. Единственный процесс, который никто не порождает, это процесс init. Порождение процесса осуществляется с помощью fork (2) или vfork (2). 

Первая стадия в жизни процесса --- \textbf{“инициализация“}. В это время ядро производит подготовительные работы к дальнейшей работе процесса. На самом деле, это не совсем состояние, но логически для пользователя оно существует.

\begin{figure}[htbp]
  \centering
  \includegraphics[width=0.7\textwidth]{./processes-and-threads/processes/lifecycle/proc-lifecycle.png}
\end{figure}

Когда инициализация процесса завершается --- он оказывается в стадии \textbf{“готов“}. С этого момента процесс находится в ожидании момента, когда его выберет планировщик процессов и даст ему какой-то квант процессорного времени.

При получении процессорного времени процесс переходит в состояние \textbf{“выполняется“}. Выполняться процесс может в режиме ядра --- при осуществлении системных вызовов, прерываний и в режиме задачи --- выполнять инструкции процессора. По окончании квоты времени, процесс может снова вернуться в состояние \textbf{“готов“}.

Также из состояния “выполняется“ процесс может перейти еще в два состояния --- \textbf{“остановлен“} и \textbf{“ожидает“}. Если он остановлен, то он остановлен пользователем (например, получен сигнал SIGSTOP). Ожидание наступает тогда, когда процессу нужны какие-то ресурсы. Как только условие, которого ждет процессор, выполняется, он переходит в состояние \textbf{“готов“}, то есть ожидает своей очереди на выполнение. 

Из каждого состояния процесс может перейти в состояние \textbf{“зомби“}. Этот процесс нужен для того, чтобы родительский процесс мог получить код возврата этого процесса. Это промежуточное состояние --- процесса уже нет, но он технически еще есть. Тем не менее, это самое что ни на есть валидное состояние. 

\textbf{В какой момент процесс окончательно исчезает из таблицы процессов?}

После того как его родительский процесс вызовет wait (2), waitpid (2), waitid(2) - позволяет завершить процесс, если он таки стал зомби. 

Традиционная проблема --- порождая процесс, необходимо в какой-то момент сказать ему wait (2), чтобы не плодить зомби. В противном случае у вас может закончится лимит на создание новых процессов. 

\textbf{Что происходит с зомби, если его родитель не вызвал wait (2), а погиб?}

Его родителем становится init. Для всех процессов, которые становятся его потомками он делает wait (2). По факту, если вы написали какой-то код, который выполняет некоторое количество процессов после чего завершился, то на самом деле зомби в системе не останутся, т.к. после того как ваш основной процесс завершился, то потомков этого процесса подхватит init и он сам вызовет для них wait (2).


         \subsection{Разделители команд}
         Для реализации работы сразу с несколькими командами был придумал механизм разделения команд между собой с помощью специальных операторов называемых разделителями команд.\\ 
Стоит сразу же отметить, что разделители команд можно комбинировать в любом варианте, в любом количестве. 

\begin{important}
	Помните, что команды интерпретируются слева направо.
\end{important}

Рассмотрим следующие разделители команд:

\begin{myenv}{“;”}{разделитель для последовательного выполнения команд.}
\end{myenv}

\begin{shCode}{Например}
		ag@helios:/home/ag$ mkdir somedir ; ls -l ; touch somefile \end{shCode}

Последовательно выполнит команды, независимо от результата их выполнения.


\begin{myenv}{“|”}{разделитель для создания конвейера - неименованного канала канала между двумя
	командами. То есть выходной поток команды до предстоящей “|” будет направлен на
вход команде, стоящей после “|”.}
\end{myenv}

	\begin{shCode}{Например}
		ag@helios:/home/ag$ ls -l | wc;  \end{shCode}

	Подсчитает количество строк, слов, символов в выводе информации о содержимом домашнего каталога.

\textcolor{gray}{Момент, о котором часто забывают при работе с конвейером. Программы в нем запускаются и выполняются параллельно. Канал, это вещь без буфера, т.е. данные должны одновременно читаться и писаться. Это приводит к тому, что для каждой программы необходимо создавать свой процесс.}

~\\
\begin{myenv}{“\&”}{оператор, предназначенный для запуска команд в фоновом режиме, но может также использоваться как разделитель.}
\end{myenv}

	\begin{shCode}{Например}
		ag@helios:/home/ag$ find / -name somename & echo "Hello" \end{shCode}

Запустит команду find / -name somename в отдельном фоновом процессе, а команда echo “Hello” будет запущена в обычном режиме и использует терминал (Выведет “Hello”).

~\\
Очень часто команды необходимо объединять разумно, то есть сложнее, чем способами описанными выше. Например, когда нам нужно, чтобы одна команда выполнилась только в случае успешного завершения другой команды. Это возможно с помощью операторов логической конъюнкции “И” и дизъюнкции “ИЛИ”.

\textcolor{gray}{Немного о том, как shell понимает, завершилась ли команда успешно. Для этого существует такое понятие как \textbf{код возврата} --- число характеризующие успешность выполнения команды.}

\textcolor{gray}{Код возврата успеха в shell, это всегда - 0. Если же команда завершилась неуспешно, мы можем посмотреть ее код возврата в переменной \$?, чтобы узнать более точную причину ошибки, а не просто факт ее возникновения.}


\begin{myenv}{“\&\&”}{разделитель “И”. Команда стоящая после “\&\&” выполнится только в случае успешного выполнения команды, стоящей до “\&\&”.}
\end{myenv}

\begin{shCode}{Например}
		ag@helios:/home/ag$ rm file1 && echo “Hello” \end{shCode}
“Hello“ будет выведено только в случае, если file1 был успешно удален.

\begin{myenv}{“||”}{разделитель “ИЛИ”. Команда стоящая после “||” выполнится только в случае
неуспешного выполнения команды, стоящей до “||”.}
\end{myenv}

\begin{shCode}{Например}
		ag@helios:/home/ag$ rm file1 || rm file2 && echo "Hello"  \end{shCode}
“Hello“ будет выведено только в случае, если был успешно удален хотя бы один из файлов file1, file2.


         \subsection{Glob-джокеры}
         Жизненный цикл процесса начинается с того, что его кто-то порождает. Единственный процесс, который никто не порождает, это процесс init. Порождение процесса осуществляется с помощью fork (2) или vfork (2). 

Первая стадия в жизни процесса --- \textbf{“инициализация“}. В это время ядро производит подготовительные работы к дальнейшей работе процесса. На самом деле, это не совсем состояние, но логически для пользователя оно существует.

\begin{figure}[htbp]
  \centering
  \includegraphics[width=0.7\textwidth]{./processes-and-threads/processes/lifecycle/proc-lifecycle.png}
\end{figure}

Когда инициализация процесса завершается --- он оказывается в стадии \textbf{“готов“}. С этого момента процесс находится в ожидании момента, когда его выберет планировщик процессов и даст ему какой-то квант процессорного времени.

При получении процессорного времени процесс переходит в состояние \textbf{“выполняется“}. Выполняться процесс может в режиме ядра --- при осуществлении системных вызовов, прерываний и в режиме задачи --- выполнять инструкции процессора. По окончании квоты времени, процесс может снова вернуться в состояние \textbf{“готов“}.

Также из состояния “выполняется“ процесс может перейти еще в два состояния --- \textbf{“остановлен“} и \textbf{“ожидает“}. Если он остановлен, то он остановлен пользователем (например, получен сигнал SIGSTOP). Ожидание наступает тогда, когда процессу нужны какие-то ресурсы. Как только условие, которого ждет процессор, выполняется, он переходит в состояние \textbf{“готов“}, то есть ожидает своей очереди на выполнение. 

Из каждого состояния процесс может перейти в состояние \textbf{“зомби“}. Этот процесс нужен для того, чтобы родительский процесс мог получить код возврата этого процесса. Это промежуточное состояние --- процесса уже нет, но он технически еще есть. Тем не менее, это самое что ни на есть валидное состояние. 

\textbf{В какой момент процесс окончательно исчезает из таблицы процессов?}

После того как его родительский процесс вызовет wait (2), waitpid (2), waitid(2) - позволяет завершить процесс, если он таки стал зомби. 

Традиционная проблема --- порождая процесс, необходимо в какой-то момент сказать ему wait (2), чтобы не плодить зомби. В противном случае у вас может закончится лимит на создание новых процессов. 

\textbf{Что происходит с зомби, если его родитель не вызвал wait (2), а погиб?}

Его родителем становится init. Для всех процессов, которые становятся его потомками он делает wait (2). По факту, если вы написали какой-то код, который выполняет некоторое количество процессов после чего завершился, то на самом деле зомби в системе не останутся, т.к. после того как ваш основной процесс завершился, то потомков этого процесса подхватит init и он сам вызовет для них wait (2).


         \subsection{Переменные}
         Внутри имен переменных shell могут использоваться буквы заглавного и прописного регистра, нижнее подчеркивание и любые цифры. Однако, первую позицию цифра занимать не может --- переменные, начинающиеся с цифры, являются специальными переменными для позиционных параметров. О них мы поговорим в другом разделе.

Для присваивания переменной значения используется оператор присваивания “=“. В качестве значения shell-переменной устанавливается строка символов. Чтобы получить значение переменной, используется символ “\$“.

\begin{shCode}{Например}
	ag@helios:/home/ag$ myvar=word
	ag@helios:/home/ag$ echo $myvar
	word \end{shCode}
		
Присвоит переменной myvar значение word и выведет его на экран.



            \subsubsection{Обрезка переменных}
            Жизненный цикл процесса начинается с того, что его кто-то порождает. Единственный процесс, который никто не порождает, это процесс init. Порождение процесса осуществляется с помощью fork (2) или vfork (2). 

Первая стадия в жизни процесса --- \textbf{“инициализация“}. В это время ядро производит подготовительные работы к дальнейшей работе процесса. На самом деле, это не совсем состояние, но логически для пользователя оно существует.

\begin{figure}[htbp]
  \centering
  \includegraphics[width=0.7\textwidth]{./processes-and-threads/processes/lifecycle/proc-lifecycle.png}
\end{figure}

Когда инициализация процесса завершается --- он оказывается в стадии \textbf{“готов“}. С этого момента процесс находится в ожидании момента, когда его выберет планировщик процессов и даст ему какой-то квант процессорного времени.

При получении процессорного времени процесс переходит в состояние \textbf{“выполняется“}. Выполняться процесс может в режиме ядра --- при осуществлении системных вызовов, прерываний и в режиме задачи --- выполнять инструкции процессора. По окончании квоты времени, процесс может снова вернуться в состояние \textbf{“готов“}.

Также из состояния “выполняется“ процесс может перейти еще в два состояния --- \textbf{“остановлен“} и \textbf{“ожидает“}. Если он остановлен, то он остановлен пользователем (например, получен сигнал SIGSTOP). Ожидание наступает тогда, когда процессу нужны какие-то ресурсы. Как только условие, которого ждет процессор, выполняется, он переходит в состояние \textbf{“готов“}, то есть ожидает своей очереди на выполнение. 

Из каждого состояния процесс может перейти в состояние \textbf{“зомби“}. Этот процесс нужен для того, чтобы родительский процесс мог получить код возврата этого процесса. Это промежуточное состояние --- процесса уже нет, но он технически еще есть. Тем не менее, это самое что ни на есть валидное состояние. 

\textbf{В какой момент процесс окончательно исчезает из таблицы процессов?}

После того как его родительский процесс вызовет wait (2), waitpid (2), waitid(2) - позволяет завершить процесс, если он таки стал зомби. 

Традиционная проблема --- порождая процесс, необходимо в какой-то момент сказать ему wait (2), чтобы не плодить зомби. В противном случае у вас может закончится лимит на создание новых процессов. 

\textbf{Что происходит с зомби, если его родитель не вызвал wait (2), а погиб?}

Его родителем становится init. Для всех процессов, которые становятся его потомками он делает wait (2). По факту, если вы написали какой-то код, который выполняет некоторое количество процессов после чего завершился, то на самом деле зомби в системе не останутся, т.к. после того как ваш основной процесс завершился, то потомков этого процесса подхватит init и он сам вызовет для них wait (2).


         \subsection{Как shell интерпретирует команду}
         Жизненный цикл процесса начинается с того, что его кто-то порождает. Единственный процесс, который никто не порождает, это процесс init. Порождение процесса осуществляется с помощью fork (2) или vfork (2). 

Первая стадия в жизни процесса --- \textbf{“инициализация“}. В это время ядро производит подготовительные работы к дальнейшей работе процесса. На самом деле, это не совсем состояние, но логически для пользователя оно существует.

\begin{figure}[htbp]
  \centering
  \includegraphics[width=0.7\textwidth]{./processes-and-threads/processes/lifecycle/proc-lifecycle.png}
\end{figure}

Когда инициализация процесса завершается --- он оказывается в стадии \textbf{“готов“}. С этого момента процесс находится в ожидании момента, когда его выберет планировщик процессов и даст ему какой-то квант процессорного времени.

При получении процессорного времени процесс переходит в состояние \textbf{“выполняется“}. Выполняться процесс может в режиме ядра --- при осуществлении системных вызовов, прерываний и в режиме задачи --- выполнять инструкции процессора. По окончании квоты времени, процесс может снова вернуться в состояние \textbf{“готов“}.

Также из состояния “выполняется“ процесс может перейти еще в два состояния --- \textbf{“остановлен“} и \textbf{“ожидает“}. Если он остановлен, то он остановлен пользователем (например, получен сигнал SIGSTOP). Ожидание наступает тогда, когда процессу нужны какие-то ресурсы. Как только условие, которого ждет процессор, выполняется, он переходит в состояние \textbf{“готов“}, то есть ожидает своей очереди на выполнение. 

Из каждого состояния процесс может перейти в состояние \textbf{“зомби“}. Этот процесс нужен для того, чтобы родительский процесс мог получить код возврата этого процесса. Это промежуточное состояние --- процесса уже нет, но он технически еще есть. Тем не менее, это самое что ни на есть валидное состояние. 

\textbf{В какой момент процесс окончательно исчезает из таблицы процессов?}

После того как его родительский процесс вызовет wait (2), waitpid (2), waitid(2) - позволяет завершить процесс, если он таки стал зомби. 

Традиционная проблема --- порождая процесс, необходимо в какой-то момент сказать ему wait (2), чтобы не плодить зомби. В противном случае у вас может закончится лимит на создание новых процессов. 

\textbf{Что происходит с зомби, если его родитель не вызвал wait (2), а погиб?}

Его родителем становится init. Для всех процессов, которые становятся его потомками он делает wait (2). По факту, если вы написали какой-то код, который выполняет некоторое количество процессов после чего завершился, то на самом деле зомби в системе не останутся, т.к. после того как ваш основной процесс завершился, то потомков этого процесса подхватит init и он сам вызовет для них wait (2).


   \chapter{Скрипты shell}
   Жизненный цикл процесса начинается с того, что его кто-то порождает. Единственный процесс, который никто не порождает, это процесс init. Порождение процесса осуществляется с помощью fork (2) или vfork (2). 

Первая стадия в жизни процесса --- \textbf{“инициализация“}. В это время ядро производит подготовительные работы к дальнейшей работе процесса. На самом деле, это не совсем состояние, но логически для пользователя оно существует.

\begin{figure}[htbp]
  \centering
  \includegraphics[width=0.7\textwidth]{./processes-and-threads/processes/lifecycle/proc-lifecycle.png}
\end{figure}

Когда инициализация процесса завершается --- он оказывается в стадии \textbf{“готов“}. С этого момента процесс находится в ожидании момента, когда его выберет планировщик процессов и даст ему какой-то квант процессорного времени.

При получении процессорного времени процесс переходит в состояние \textbf{“выполняется“}. Выполняться процесс может в режиме ядра --- при осуществлении системных вызовов, прерываний и в режиме задачи --- выполнять инструкции процессора. По окончании квоты времени, процесс может снова вернуться в состояние \textbf{“готов“}.

Также из состояния “выполняется“ процесс может перейти еще в два состояния --- \textbf{“остановлен“} и \textbf{“ожидает“}. Если он остановлен, то он остановлен пользователем (например, получен сигнал SIGSTOP). Ожидание наступает тогда, когда процессу нужны какие-то ресурсы. Как только условие, которого ждет процессор, выполняется, он переходит в состояние \textbf{“готов“}, то есть ожидает своей очереди на выполнение. 

Из каждого состояния процесс может перейти в состояние \textbf{“зомби“}. Этот процесс нужен для того, чтобы родительский процесс мог получить код возврата этого процесса. Это промежуточное состояние --- процесса уже нет, но он технически еще есть. Тем не менее, это самое что ни на есть валидное состояние. 

\textbf{В какой момент процесс окончательно исчезает из таблицы процессов?}

После того как его родительский процесс вызовет wait (2), waitpid (2), waitid(2) - позволяет завершить процесс, если он таки стал зомби. 

Традиционная проблема --- порождая процесс, необходимо в какой-то момент сказать ему wait (2), чтобы не плодить зомби. В противном случае у вас может закончится лимит на создание новых процессов. 

\textbf{Что происходит с зомби, если его родитель не вызвал wait (2), а погиб?}

Его родителем становится init. Для всех процессов, которые становятся его потомками он делает wait (2). По факту, если вы написали какой-то код, который выполняет некоторое количество процессов после чего завершился, то на самом деле зомби в системе не останутся, т.к. после того как ваш основной процесс завершился, то потомков этого процесса подхватит init и он сам вызовет для них wait (2).


      \section{Именование и запуск скриптов}
      Жизненный цикл процесса начинается с того, что его кто-то порождает. Единственный процесс, который никто не порождает, это процесс init. Порождение процесса осуществляется с помощью fork (2) или vfork (2). 

Первая стадия в жизни процесса --- \textbf{“инициализация“}. В это время ядро производит подготовительные работы к дальнейшей работе процесса. На самом деле, это не совсем состояние, но логически для пользователя оно существует.

\begin{figure}[htbp]
  \centering
  \includegraphics[width=0.7\textwidth]{./processes-and-threads/processes/lifecycle/proc-lifecycle.png}
\end{figure}

Когда инициализация процесса завершается --- он оказывается в стадии \textbf{“готов“}. С этого момента процесс находится в ожидании момента, когда его выберет планировщик процессов и даст ему какой-то квант процессорного времени.

При получении процессорного времени процесс переходит в состояние \textbf{“выполняется“}. Выполняться процесс может в режиме ядра --- при осуществлении системных вызовов, прерываний и в режиме задачи --- выполнять инструкции процессора. По окончании квоты времени, процесс может снова вернуться в состояние \textbf{“готов“}.

Также из состояния “выполняется“ процесс может перейти еще в два состояния --- \textbf{“остановлен“} и \textbf{“ожидает“}. Если он остановлен, то он остановлен пользователем (например, получен сигнал SIGSTOP). Ожидание наступает тогда, когда процессу нужны какие-то ресурсы. Как только условие, которого ждет процессор, выполняется, он переходит в состояние \textbf{“готов“}, то есть ожидает своей очереди на выполнение. 

Из каждого состояния процесс может перейти в состояние \textbf{“зомби“}. Этот процесс нужен для того, чтобы родительский процесс мог получить код возврата этого процесса. Это промежуточное состояние --- процесса уже нет, но он технически еще есть. Тем не менее, это самое что ни на есть валидное состояние. 

\textbf{В какой момент процесс окончательно исчезает из таблицы процессов?}

После того как его родительский процесс вызовет wait (2), waitpid (2), waitid(2) - позволяет завершить процесс, если он таки стал зомби. 

Традиционная проблема --- порождая процесс, необходимо в какой-то момент сказать ему wait (2), чтобы не плодить зомби. В противном случае у вас может закончится лимит на создание новых процессов. 

\textbf{Что происходит с зомби, если его родитель не вызвал wait (2), а погиб?}

Его родителем становится init. Для всех процессов, которые становятся его потомками он делает wait (2). По факту, если вы написали какой-то код, который выполняет некоторое количество процессов после чего завершился, то на самом деле зомби в системе не останутся, т.к. после того как ваш основной процесс завершился, то потомков этого процесса подхватит init и он сам вызовет для них wait (2).


      \section{Позиционные параметры}
      Жизненный цикл процесса начинается с того, что его кто-то порождает. Единственный процесс, который никто не порождает, это процесс init. Порождение процесса осуществляется с помощью fork (2) или vfork (2). 

Первая стадия в жизни процесса --- \textbf{“инициализация“}. В это время ядро производит подготовительные работы к дальнейшей работе процесса. На самом деле, это не совсем состояние, но логически для пользователя оно существует.

\begin{figure}[htbp]
  \centering
  \includegraphics[width=0.7\textwidth]{./processes-and-threads/processes/lifecycle/proc-lifecycle.png}
\end{figure}

Когда инициализация процесса завершается --- он оказывается в стадии \textbf{“готов“}. С этого момента процесс находится в ожидании момента, когда его выберет планировщик процессов и даст ему какой-то квант процессорного времени.

При получении процессорного времени процесс переходит в состояние \textbf{“выполняется“}. Выполняться процесс может в режиме ядра --- при осуществлении системных вызовов, прерываний и в режиме задачи --- выполнять инструкции процессора. По окончании квоты времени, процесс может снова вернуться в состояние \textbf{“готов“}.

Также из состояния “выполняется“ процесс может перейти еще в два состояния --- \textbf{“остановлен“} и \textbf{“ожидает“}. Если он остановлен, то он остановлен пользователем (например, получен сигнал SIGSTOP). Ожидание наступает тогда, когда процессу нужны какие-то ресурсы. Как только условие, которого ждет процессор, выполняется, он переходит в состояние \textbf{“готов“}, то есть ожидает своей очереди на выполнение. 

Из каждого состояния процесс может перейти в состояние \textbf{“зомби“}. Этот процесс нужен для того, чтобы родительский процесс мог получить код возврата этого процесса. Это промежуточное состояние --- процесса уже нет, но он технически еще есть. Тем не менее, это самое что ни на есть валидное состояние. 

\textbf{В какой момент процесс окончательно исчезает из таблицы процессов?}

После того как его родительский процесс вызовет wait (2), waitpid (2), waitid(2) - позволяет завершить процесс, если он таки стал зомби. 

Традиционная проблема --- порождая процесс, необходимо в какой-то момент сказать ему wait (2), чтобы не плодить зомби. В противном случае у вас может закончится лимит на создание новых процессов. 

\textbf{Что происходит с зомби, если его родитель не вызвал wait (2), а погиб?}

Его родителем становится init. Для всех процессов, которые становятся его потомками он делает wait (2). По факту, если вы написали какой-то код, который выполняет некоторое количество процессов после чего завершился, то на самом деле зомби в системе не останутся, т.к. после того как ваш основной процесс завершился, то потомков этого процесса подхватит init и он сам вызовет для них wait (2).


      \section{Специальные переменные}
      Говоря о позиционных параметрах, нельзя забывать две переменных окружения ---  “@” и “*”. Это такие переменные, которые  позволяют обратиться ко всем позиционным параметрам, начиная с первого.

При написании без двойных кавычек (echo \$* и echo \$@) эти переменные выдают одинаковый результат - подставляют вместо себя позиционные параметры разделенные пробелом. Разница станет значительной лишь при использовании двойных кавычек:

echo “\$*” 	выведет одну большую строку в кавычках, разделенную первым символом из IFS.
 
echo “\$@” 	    выдаст набор разделенных пробелом позиционных параметров (по сути массив,с которым мы сможем работать). Этот момент всегда нужно помнить. 

Рассмотрим пример с установкой значения IF в :

\begin{shCode}{Файл myscript}
#!/bin/sh
IFS=:
echo "$*"
echo "$@" \end{shCode}

\begin{shCode}{shell}
	ag@helios:/home/ag$ ./myscript test 67 value
	test:67:value
	test 67 value \end{shCode}

На самом деле, специальных переменных довольно много, рассмотрим некоторые из них:

\begin{myenv}{\$-}{содержит ключи, передаваемые shell (фактически опции задаваемые с помощью set).}
\end{myenv}

\begin{myenv}{\$\_}{содержит последний аргумент предыдущей выполненной команды.}
\end{myenv}

\begin{myenv}{\$\#}{содержит число позиционных параметров.}
\end{myenv}

\begin{myenv}{\$?}{содержит код возврата предыдущей команды.}
\end{myenv}

\begin{myenv}{\$!}{содержит идентификатор процесса, который был последним свернут в бэкграунд группу. Если вы запустили два процесса, то в переменной \$! будет PID последнего процесса, который был запущен.}
\end{myenv}


   \chapter{Встроенные возможности и конструкции shell}
   Жизненный цикл процесса начинается с того, что его кто-то порождает. Единственный процесс, который никто не порождает, это процесс init. Порождение процесса осуществляется с помощью fork (2) или vfork (2). 

Первая стадия в жизни процесса --- \textbf{“инициализация“}. В это время ядро производит подготовительные работы к дальнейшей работе процесса. На самом деле, это не совсем состояние, но логически для пользователя оно существует.

\begin{figure}[htbp]
  \centering
  \includegraphics[width=0.7\textwidth]{./processes-and-threads/processes/lifecycle/proc-lifecycle.png}
\end{figure}

Когда инициализация процесса завершается --- он оказывается в стадии \textbf{“готов“}. С этого момента процесс находится в ожидании момента, когда его выберет планировщик процессов и даст ему какой-то квант процессорного времени.

При получении процессорного времени процесс переходит в состояние \textbf{“выполняется“}. Выполняться процесс может в режиме ядра --- при осуществлении системных вызовов, прерываний и в режиме задачи --- выполнять инструкции процессора. По окончании квоты времени, процесс может снова вернуться в состояние \textbf{“готов“}.

Также из состояния “выполняется“ процесс может перейти еще в два состояния --- \textbf{“остановлен“} и \textbf{“ожидает“}. Если он остановлен, то он остановлен пользователем (например, получен сигнал SIGSTOP). Ожидание наступает тогда, когда процессу нужны какие-то ресурсы. Как только условие, которого ждет процессор, выполняется, он переходит в состояние \textbf{“готов“}, то есть ожидает своей очереди на выполнение. 

Из каждого состояния процесс может перейти в состояние \textbf{“зомби“}. Этот процесс нужен для того, чтобы родительский процесс мог получить код возврата этого процесса. Это промежуточное состояние --- процесса уже нет, но он технически еще есть. Тем не менее, это самое что ни на есть валидное состояние. 

\textbf{В какой момент процесс окончательно исчезает из таблицы процессов?}

После того как его родительский процесс вызовет wait (2), waitpid (2), waitid(2) - позволяет завершить процесс, если он таки стал зомби. 

Традиционная проблема --- порождая процесс, необходимо в какой-то момент сказать ему wait (2), чтобы не плодить зомби. В противном случае у вас может закончится лимит на создание новых процессов. 

\textbf{Что происходит с зомби, если его родитель не вызвал wait (2), а погиб?}

Его родителем становится init. Для всех процессов, которые становятся его потомками он делает wait (2). По факту, если вы написали какой-то код, который выполняет некоторое количество процессов после чего завершился, то на самом деле зомби в системе не останутся, т.к. после того как ваш основной процесс завершился, то потомков этого процесса подхватит init и он сам вызовет для них wait (2).


      \section{Математические операции}
      Командный интерпретатор поддерживает набор следующих математических операций:

\begin{center}
	\begin{tabular}{c|l}
		\textbf{Обозначение} & \textbf{Наименование} \\
		\hline
		+ 	& 	Сложение \\
		\hline
		- 	& 	Вычитание \\
		\hline
		* 	& 	Умножение \\
		\hline
		/ 	& 	Деление \\
		\hline
		\% 	& 	Остаток от деления \\
		\hline
		<{<} 	& 	Сдвиг влево \\
		\hline
		>{>} 	& 	Сдвиг вправо \\
		\hline
		\& 	& 	Логическое И \\
		\hline
		| 	& 	Логическое ИЛИ \\
		\hline
		\textasciicircum 	& 	Исключающее ИЛИ
	\end{tabular}
\end{center}


      \section{Условные кострукции, циклы, функции, комментарии}

         \subsection{Условные кострукции}
         Для ветвления программ в shell существуют условные конструкции IF и CASE.

\begin{shCode}{IF...THEN...ELSE}
	IF ( <condition> ) 
		THEN <actions>
	ELIF ( <alternative condition> ) 
		THEN <actions>
	ELSE 
		<actions>
	FI
\end{shCode}

В качестве условия выполнения --- <condition> --- принимает набор команд (Например, через  \&\&). Действие после THEN будет выполнено только в случае, если код возврата последней команды --- успех.

\begin{important}
	Помним, что при написании IF и THEN в одной строке они должны разделяться “;”, так как являются разными командами
\end{important}

\begin{shCode}{То есть:}
	IF ( <condition>) ; THEN <Actions>
	ELIF ( <alternative condition> ) ; THEN <actions>
	ELSE <actions>
	FI
\end{shCode}


\begin{shCode}{CASE}
	CASE (<string>) IN
     pattern-1)      
		<actions>
          	;;
     pattern-2)      
     	<actions>
          	;;
     pattern | pattern)
     	<actions>
          	;;
	ESAC
\end{shCode}

Принимает строку и в случае соответствие ее одному из установленных шаблонов совершает необходимые действия. По сути, аналог switch.

\begin{important}
	Не забываем о возможностях использования glob-джокерах в CASE. Так * будет обозначать не вошедшие в шаблоны случаи.
\end{important}

Заметка: завершающие слова в IF и CASE - просто написание их же в обратном порядке (IF - FI, CASE - ESAC)



         \subsection{Циклы}
         Довольно часто условные конструкции можно опускать за счет использования “\&\&” или “||”, а вот такие вещи как циклы имитировать просто так не получится

Всего в shell три вида циклов - while, for и until, рассмотрим их подробнее.

\begin{shCode}{WHILE}
	WHILE ( <condition> )
		DO
	 		<actions>
		DONE
\end{shCode}
Код внутри do - done будет выполняться пока условие истинно

\begin{shCode}{UNTIL}
	UNTIL ( <condition> )
		DO
			<actions>
		DONE
\end{shCode}
Код внутри do - done будет выполняться пока условие ложно

\begin{shCode}{FOR}
	FOR var IN ( <list of values> )
		DO
			<actions>
		DONE
\end{shCode}
Обеспечивает выполнения столько раз, сколько значений в списке значений, при этом переменная var последовательно принимает значения из списка значений при каждой новой итерации

Такая форма for в принципе не вызывает никаких вопросов. Но что будет, если опустить список значений? Дело в том, что это позволит неявно пройти по позиционным параметрам переданным shell, то есть var будет принимать уже их значения

\textbf{BREAK и CONTINUE}

Вспоминая си, у нас есть возможность перейти к следующей итерации цикла с помощью continue и закончить, с помощью break. В shell есть похожие слова. 

Отличие BREAK в shell в том, что он  позволяет выходить не только из последнего цикла, в  котором он написан, но и из любого уровня вложенности циклов. 

BREAK N, где N --- количество уровней циклов, из которых мы выйдем.

Ключевое слово CONTINUE позволяет переходить заново к проверке условия (в начало цикла), то есть работает также как, например, в языке Си. 


         \subsection{Функции}
         Когда мы пишем shell скрипты, то можно воспользоваться такой вещью как функция, чтобы ее задать необходимо использовать следующий синтаксис:

\begin{shCode}{ }
	function func {
		echo "Inspiration unlocks the future."
	}
\end{shCode}

Ключевое слово function не является обязательным, его можно опускать.

Перечислять аргументы в прототипе не нужно потому как все аргументы, которые вы передадите этой функции будут доступны как позиционные параметры.

\begin{shCode}{Пример функции и ее вызова с параметрами}
		function func {
			echo $1 $2
		}
	
		func $myvar1 $myvar2
\end{shCode}


         \subsection{Комментарии}
         Жизненный цикл процесса начинается с того, что его кто-то порождает. Единственный процесс, который никто не порождает, это процесс init. Порождение процесса осуществляется с помощью fork (2) или vfork (2). 

Первая стадия в жизни процесса --- \textbf{“инициализация“}. В это время ядро производит подготовительные работы к дальнейшей работе процесса. На самом деле, это не совсем состояние, но логически для пользователя оно существует.

\begin{figure}[htbp]
  \centering
  \includegraphics[width=0.7\textwidth]{./processes-and-threads/processes/lifecycle/proc-lifecycle.png}
\end{figure}

Когда инициализация процесса завершается --- он оказывается в стадии \textbf{“готов“}. С этого момента процесс находится в ожидании момента, когда его выберет планировщик процессов и даст ему какой-то квант процессорного времени.

При получении процессорного времени процесс переходит в состояние \textbf{“выполняется“}. Выполняться процесс может в режиме ядра --- при осуществлении системных вызовов, прерываний и в режиме задачи --- выполнять инструкции процессора. По окончании квоты времени, процесс может снова вернуться в состояние \textbf{“готов“}.

Также из состояния “выполняется“ процесс может перейти еще в два состояния --- \textbf{“остановлен“} и \textbf{“ожидает“}. Если он остановлен, то он остановлен пользователем (например, получен сигнал SIGSTOP). Ожидание наступает тогда, когда процессу нужны какие-то ресурсы. Как только условие, которого ждет процессор, выполняется, он переходит в состояние \textbf{“готов“}, то есть ожидает своей очереди на выполнение. 

Из каждого состояния процесс может перейти в состояние \textbf{“зомби“}. Этот процесс нужен для того, чтобы родительский процесс мог получить код возврата этого процесса. Это промежуточное состояние --- процесса уже нет, но он технически еще есть. Тем не менее, это самое что ни на есть валидное состояние. 

\textbf{В какой момент процесс окончательно исчезает из таблицы процессов?}

После того как его родительский процесс вызовет wait (2), waitpid (2), waitid(2) - позволяет завершить процесс, если он таки стал зомби. 

Традиционная проблема --- порождая процесс, необходимо в какой-то момент сказать ему wait (2), чтобы не плодить зомби. В противном случае у вас может закончится лимит на создание новых процессов. 

\textbf{Что происходит с зомби, если его родитель не вызвал wait (2), а погиб?}

Его родителем становится init. Для всех процессов, которые становятся его потомками он делает wait (2). По факту, если вы написали какой-то код, который выполняет некоторое количество процессов после чего завершился, то на самом деле зомби в системе не останутся, т.к. после того как ваш основной процесс завершился, то потомков этого процесса подхватит init и он сам вызовет для них wait (2).


      \chapter{Потоки ввода-вывода}
      Жизненный цикл процесса начинается с того, что его кто-то порождает. Единственный процесс, который никто не порождает, это процесс init. Порождение процесса осуществляется с помощью fork (2) или vfork (2). 

Первая стадия в жизни процесса --- \textbf{“инициализация“}. В это время ядро производит подготовительные работы к дальнейшей работе процесса. На самом деле, это не совсем состояние, но логически для пользователя оно существует.

\begin{figure}[htbp]
  \centering
  \includegraphics[width=0.7\textwidth]{./processes-and-threads/processes/lifecycle/proc-lifecycle.png}
\end{figure}

Когда инициализация процесса завершается --- он оказывается в стадии \textbf{“готов“}. С этого момента процесс находится в ожидании момента, когда его выберет планировщик процессов и даст ему какой-то квант процессорного времени.

При получении процессорного времени процесс переходит в состояние \textbf{“выполняется“}. Выполняться процесс может в режиме ядра --- при осуществлении системных вызовов, прерываний и в режиме задачи --- выполнять инструкции процессора. По окончании квоты времени, процесс может снова вернуться в состояние \textbf{“готов“}.

Также из состояния “выполняется“ процесс может перейти еще в два состояния --- \textbf{“остановлен“} и \textbf{“ожидает“}. Если он остановлен, то он остановлен пользователем (например, получен сигнал SIGSTOP). Ожидание наступает тогда, когда процессу нужны какие-то ресурсы. Как только условие, которого ждет процессор, выполняется, он переходит в состояние \textbf{“готов“}, то есть ожидает своей очереди на выполнение. 

Из каждого состояния процесс может перейти в состояние \textbf{“зомби“}. Этот процесс нужен для того, чтобы родительский процесс мог получить код возврата этого процесса. Это промежуточное состояние --- процесса уже нет, но он технически еще есть. Тем не менее, это самое что ни на есть валидное состояние. 

\textbf{В какой момент процесс окончательно исчезает из таблицы процессов?}

После того как его родительский процесс вызовет wait (2), waitpid (2), waitid(2) - позволяет завершить процесс, если он таки стал зомби. 

Традиционная проблема --- порождая процесс, необходимо в какой-то момент сказать ему wait (2), чтобы не плодить зомби. В противном случае у вас может закончится лимит на создание новых процессов. 

\textbf{Что происходит с зомби, если его родитель не вызвал wait (2), а погиб?}

Его родителем становится init. Для всех процессов, которые становятся его потомками он делает wait (2). По факту, если вы написали какой-то код, который выполняет некоторое количество процессов после чего завершился, то на самом деле зомби в системе не останутся, т.к. после того как ваш основной процесс завершился, то потомков этого процесса подхватит init и он сам вызовет для них wait (2).


         \section{Стандартные потоки ввода-вывода}
         Жизненный цикл процесса начинается с того, что его кто-то порождает. Единственный процесс, который никто не порождает, это процесс init. Порождение процесса осуществляется с помощью fork (2) или vfork (2). 

Первая стадия в жизни процесса --- \textbf{“инициализация“}. В это время ядро производит подготовительные работы к дальнейшей работе процесса. На самом деле, это не совсем состояние, но логически для пользователя оно существует.

\begin{figure}[htbp]
  \centering
  \includegraphics[width=0.7\textwidth]{./processes-and-threads/processes/lifecycle/proc-lifecycle.png}
\end{figure}

Когда инициализация процесса завершается --- он оказывается в стадии \textbf{“готов“}. С этого момента процесс находится в ожидании момента, когда его выберет планировщик процессов и даст ему какой-то квант процессорного времени.

При получении процессорного времени процесс переходит в состояние \textbf{“выполняется“}. Выполняться процесс может в режиме ядра --- при осуществлении системных вызовов, прерываний и в режиме задачи --- выполнять инструкции процессора. По окончании квоты времени, процесс может снова вернуться в состояние \textbf{“готов“}.

Также из состояния “выполняется“ процесс может перейти еще в два состояния --- \textbf{“остановлен“} и \textbf{“ожидает“}. Если он остановлен, то он остановлен пользователем (например, получен сигнал SIGSTOP). Ожидание наступает тогда, когда процессу нужны какие-то ресурсы. Как только условие, которого ждет процессор, выполняется, он переходит в состояние \textbf{“готов“}, то есть ожидает своей очереди на выполнение. 

Из каждого состояния процесс может перейти в состояние \textbf{“зомби“}. Этот процесс нужен для того, чтобы родительский процесс мог получить код возврата этого процесса. Это промежуточное состояние --- процесса уже нет, но он технически еще есть. Тем не менее, это самое что ни на есть валидное состояние. 

\textbf{В какой момент процесс окончательно исчезает из таблицы процессов?}

После того как его родительский процесс вызовет wait (2), waitpid (2), waitid(2) - позволяет завершить процесс, если он таки стал зомби. 

Традиционная проблема --- порождая процесс, необходимо в какой-то момент сказать ему wait (2), чтобы не плодить зомби. В противном случае у вас может закончится лимит на создание новых процессов. 

\textbf{Что происходит с зомби, если его родитель не вызвал wait (2), а погиб?}

Его родителем становится init. Для всех процессов, которые становятся его потомками он делает wait (2). По факту, если вы написали какой-то код, который выполняет некоторое количество процессов после чего завершился, то на самом деле зомби в системе не останутся, т.к. после того как ваш основной процесс завершился, то потомков этого процесса подхватит init и он сам вызовет для них wait (2).


         \section{Перенаправление потоков}
         Жизненный цикл процесса начинается с того, что его кто-то порождает. Единственный процесс, который никто не порождает, это процесс init. Порождение процесса осуществляется с помощью fork (2) или vfork (2). 

Первая стадия в жизни процесса --- \textbf{“инициализация“}. В это время ядро производит подготовительные работы к дальнейшей работе процесса. На самом деле, это не совсем состояние, но логически для пользователя оно существует.

\begin{figure}[htbp]
  \centering
  \includegraphics[width=0.7\textwidth]{./processes-and-threads/processes/lifecycle/proc-lifecycle.png}
\end{figure}

Когда инициализация процесса завершается --- он оказывается в стадии \textbf{“готов“}. С этого момента процесс находится в ожидании момента, когда его выберет планировщик процессов и даст ему какой-то квант процессорного времени.

При получении процессорного времени процесс переходит в состояние \textbf{“выполняется“}. Выполняться процесс может в режиме ядра --- при осуществлении системных вызовов, прерываний и в режиме задачи --- выполнять инструкции процессора. По окончании квоты времени, процесс может снова вернуться в состояние \textbf{“готов“}.

Также из состояния “выполняется“ процесс может перейти еще в два состояния --- \textbf{“остановлен“} и \textbf{“ожидает“}. Если он остановлен, то он остановлен пользователем (например, получен сигнал SIGSTOP). Ожидание наступает тогда, когда процессу нужны какие-то ресурсы. Как только условие, которого ждет процессор, выполняется, он переходит в состояние \textbf{“готов“}, то есть ожидает своей очереди на выполнение. 

Из каждого состояния процесс может перейти в состояние \textbf{“зомби“}. Этот процесс нужен для того, чтобы родительский процесс мог получить код возврата этого процесса. Это промежуточное состояние --- процесса уже нет, но он технически еще есть. Тем не менее, это самое что ни на есть валидное состояние. 

\textbf{В какой момент процесс окончательно исчезает из таблицы процессов?}

После того как его родительский процесс вызовет wait (2), waitpid (2), waitid(2) - позволяет завершить процесс, если он таки стал зомби. 

Традиционная проблема --- порождая процесс, необходимо в какой-то момент сказать ему wait (2), чтобы не плодить зомби. В противном случае у вас может закончится лимит на создание новых процессов. 

\textbf{Что происходит с зомби, если его родитель не вызвал wait (2), а погиб?}

Его родителем становится init. Для всех процессов, которые становятся его потомками он делает wait (2). По факту, если вы написали какой-то код, который выполняет некоторое количество процессов после чего завершился, то на самом деле зомби в системе не останутся, т.к. после того как ваш основной процесс завершился, то потомков этого процесса подхватит init и он сам вызовет для них wait (2).


      \chapter{set и env}

         \section{Команда set}
         \begin{defi}{set}
	встроенная в shell команда, она предназначена для установки или сброса ключей и позиционных параметров.
\end{defi}

\begin{shCode}{Например}
	ag@helios:/home/ag$ set word1 word2 word3 \end{shCode}
Установит значения 1-го, 2-го и 3-го позиционных параметров, соответственно.

Процесс экспортирования необратим. Единственный способ сделать такую переменную неэкспортируемой это удалить саму переменную т.е сделать операцию unset.

Приведем некоторые режимы set:
\begin{itemize}
	\item \textbf{set -v} -- выводит на терминал текст исходной команды, перед ее выполнением;
	\item \textbf{set -x} -- выводит на терминал текст интерпретированной команды, перед ее выполнением;
	\item \textbf{set -vi/emacs} -- устанавливает в shell режим управления как в соответствующих редакторах;
	\item \textbf{set -o ignore EOF} -- игнорировать конец файла;
	\item \textbf{set -o no exec} -- запретить выполнение команд в текущем shell;
	\item \textbf{set -o no clobber} -- аналог >|.
\end{itemize}

\begin{important}
	Символ - перед опцией означет включение режима, символ + -- выключение
\end{important}


         \section{Утилита env}
         Жизненный цикл процесса начинается с того, что его кто-то порождает. Единственный процесс, который никто не порождает, это процесс init. Порождение процесса осуществляется с помощью fork (2) или vfork (2). 

Первая стадия в жизни процесса --- \textbf{“инициализация“}. В это время ядро производит подготовительные работы к дальнейшей работе процесса. На самом деле, это не совсем состояние, но логически для пользователя оно существует.

\begin{figure}[htbp]
  \centering
  \includegraphics[width=0.7\textwidth]{./processes-and-threads/processes/lifecycle/proc-lifecycle.png}
\end{figure}

Когда инициализация процесса завершается --- он оказывается в стадии \textbf{“готов“}. С этого момента процесс находится в ожидании момента, когда его выберет планировщик процессов и даст ему какой-то квант процессорного времени.

При получении процессорного времени процесс переходит в состояние \textbf{“выполняется“}. Выполняться процесс может в режиме ядра --- при осуществлении системных вызовов, прерываний и в режиме задачи --- выполнять инструкции процессора. По окончании квоты времени, процесс может снова вернуться в состояние \textbf{“готов“}.

Также из состояния “выполняется“ процесс может перейти еще в два состояния --- \textbf{“остановлен“} и \textbf{“ожидает“}. Если он остановлен, то он остановлен пользователем (например, получен сигнал SIGSTOP). Ожидание наступает тогда, когда процессу нужны какие-то ресурсы. Как только условие, которого ждет процессор, выполняется, он переходит в состояние \textbf{“готов“}, то есть ожидает своей очереди на выполнение. 

Из каждого состояния процесс может перейти в состояние \textbf{“зомби“}. Этот процесс нужен для того, чтобы родительский процесс мог получить код возврата этого процесса. Это промежуточное состояние --- процесса уже нет, но он технически еще есть. Тем не менее, это самое что ни на есть валидное состояние. 

\textbf{В какой момент процесс окончательно исчезает из таблицы процессов?}

После того как его родительский процесс вызовет wait (2), waitpid (2), waitid(2) - позволяет завершить процесс, если он таки стал зомби. 

Традиционная проблема --- порождая процесс, необходимо в какой-то момент сказать ему wait (2), чтобы не плодить зомби. В противном случае у вас может закончится лимит на создание новых процессов. 

\textbf{Что происходит с зомби, если его родитель не вызвал wait (2), а погиб?}

Его родителем становится init. Для всех процессов, которые становятся его потомками он делает wait (2). По факту, если вы написали какой-то код, который выполняет некоторое количество процессов после чего завершился, то на самом деле зомби в системе не останутся, т.к. после того как ваш основной процесс завершился, то потомков этого процесса подхватит init и он сам вызовет для них wait (2).




         % ------------------------------
         % Часть: Язык Си в системном программировании

\clearpage \part{Язык Си в системном программировании}
Целью этой части не является изучение языка Си, она носит только вспомогательный характер. В основном предназначена для освежения и восполнения пробелов в знаниях о языке Си, изученном в рамках других курсов.

В этой части также будут рассмотрены некоторые особенности системного программирования на языке Си. В том числе в операционной системе UNIX.


   \chapter{Программы на языке Си}
   Жизненный цикл процесса начинается с того, что его кто-то порождает. Единственный процесс, который никто не порождает, это процесс init. Порождение процесса осуществляется с помощью fork (2) или vfork (2). 

Первая стадия в жизни процесса --- \textbf{“инициализация“}. В это время ядро производит подготовительные работы к дальнейшей работе процесса. На самом деле, это не совсем состояние, но логически для пользователя оно существует.

\begin{figure}[htbp]
  \centering
  \includegraphics[width=0.7\textwidth]{./processes-and-threads/processes/lifecycle/proc-lifecycle.png}
\end{figure}

Когда инициализация процесса завершается --- он оказывается в стадии \textbf{“готов“}. С этого момента процесс находится в ожидании момента, когда его выберет планировщик процессов и даст ему какой-то квант процессорного времени.

При получении процессорного времени процесс переходит в состояние \textbf{“выполняется“}. Выполняться процесс может в режиме ядра --- при осуществлении системных вызовов, прерываний и в режиме задачи --- выполнять инструкции процессора. По окончании квоты времени, процесс может снова вернуться в состояние \textbf{“готов“}.

Также из состояния “выполняется“ процесс может перейти еще в два состояния --- \textbf{“остановлен“} и \textbf{“ожидает“}. Если он остановлен, то он остановлен пользователем (например, получен сигнал SIGSTOP). Ожидание наступает тогда, когда процессу нужны какие-то ресурсы. Как только условие, которого ждет процессор, выполняется, он переходит в состояние \textbf{“готов“}, то есть ожидает своей очереди на выполнение. 

Из каждого состояния процесс может перейти в состояние \textbf{“зомби“}. Этот процесс нужен для того, чтобы родительский процесс мог получить код возврата этого процесса. Это промежуточное состояние --- процесса уже нет, но он технически еще есть. Тем не менее, это самое что ни на есть валидное состояние. 

\textbf{В какой момент процесс окончательно исчезает из таблицы процессов?}

После того как его родительский процесс вызовет wait (2), waitpid (2), waitid(2) - позволяет завершить процесс, если он таки стал зомби. 

Традиционная проблема --- порождая процесс, необходимо в какой-то момент сказать ему wait (2), чтобы не плодить зомби. В противном случае у вас может закончится лимит на создание новых процессов. 

\textbf{Что происходит с зомби, если его родитель не вызвал wait (2), а погиб?}

Его родителем становится init. Для всех процессов, которые становятся его потомками он делает wait (2). По факту, если вы написали какой-то код, который выполняет некоторое количество процессов после чего завершился, то на самом деле зомби в системе не останутся, т.к. после того как ваш основной процесс завершился, то потомков этого процесса подхватит init и он сам вызовет для них wait (2).


      \section{Заголовочные файлы}
      Жизненный цикл процесса начинается с того, что его кто-то порождает. Единственный процесс, который никто не порождает, это процесс init. Порождение процесса осуществляется с помощью fork (2) или vfork (2). 

Первая стадия в жизни процесса --- \textbf{“инициализация“}. В это время ядро производит подготовительные работы к дальнейшей работе процесса. На самом деле, это не совсем состояние, но логически для пользователя оно существует.

\begin{figure}[htbp]
  \centering
  \includegraphics[width=0.7\textwidth]{./processes-and-threads/processes/lifecycle/proc-lifecycle.png}
\end{figure}

Когда инициализация процесса завершается --- он оказывается в стадии \textbf{“готов“}. С этого момента процесс находится в ожидании момента, когда его выберет планировщик процессов и даст ему какой-то квант процессорного времени.

При получении процессорного времени процесс переходит в состояние \textbf{“выполняется“}. Выполняться процесс может в режиме ядра --- при осуществлении системных вызовов, прерываний и в режиме задачи --- выполнять инструкции процессора. По окончании квоты времени, процесс может снова вернуться в состояние \textbf{“готов“}.

Также из состояния “выполняется“ процесс может перейти еще в два состояния --- \textbf{“остановлен“} и \textbf{“ожидает“}. Если он остановлен, то он остановлен пользователем (например, получен сигнал SIGSTOP). Ожидание наступает тогда, когда процессу нужны какие-то ресурсы. Как только условие, которого ждет процессор, выполняется, он переходит в состояние \textbf{“готов“}, то есть ожидает своей очереди на выполнение. 

Из каждого состояния процесс может перейти в состояние \textbf{“зомби“}. Этот процесс нужен для того, чтобы родительский процесс мог получить код возврата этого процесса. Это промежуточное состояние --- процесса уже нет, но он технически еще есть. Тем не менее, это самое что ни на есть валидное состояние. 

\textbf{В какой момент процесс окончательно исчезает из таблицы процессов?}

После того как его родительский процесс вызовет wait (2), waitpid (2), waitid(2) - позволяет завершить процесс, если он таки стал зомби. 

Традиционная проблема --- порождая процесс, необходимо в какой-то момент сказать ему wait (2), чтобы не плодить зомби. В противном случае у вас может закончится лимит на создание новых процессов. 

\textbf{Что происходит с зомби, если его родитель не вызвал wait (2), а погиб?}

Его родителем становится init. Для всех процессов, которые становятся его потомками он делает wait (2). По факту, если вы написали какой-то код, который выполняет некоторое количество процессов после чего завершился, то на самом деле зомби в системе не останутся, т.к. после того как ваш основной процесс завершился, то потомков этого процесса подхватит init и он сам вызовет для них wait (2).


         \subsection{Часто используемые заголовочные файлы UNIX}
         Приведем заголовочные файлы, наиболее часто используемые в системном программировании в операционной системе UNIX.

	\begin{center}
		\begin{tabular}{l|l}	
			\textbf{Имя} & \textbf{Содержимое} \\
			\hline
			unistd.h 	& объявления UNIX \\
			\hline
			stdio.h  	& стандартный ввод/вывод \\
			\hline
			fcntl.h  	& операции с файлами (например: open) \\
			\hline
			sys/types.h	& системные типы \\
			\hline
			sys/stat.h	& системные статусы \\
			\hline
			errno.h	& errno и и директивы с определением ошибок \\
		\end{tabular}
		\end{center}
		
Более подробно узнать об их назначении можно узнать использовав команду \textbf{man}.

	\begin{shCode}{Например}
		ag@helios:/home/ag$ man stdlib 
		ag@helios:/home/ag$ man stdlib.h \end{shCode}


      \section{main-функция}
      Жизненный цикл процесса начинается с того, что его кто-то порождает. Единственный процесс, который никто не порождает, это процесс init. Порождение процесса осуществляется с помощью fork (2) или vfork (2). 

Первая стадия в жизни процесса --- \textbf{“инициализация“}. В это время ядро производит подготовительные работы к дальнейшей работе процесса. На самом деле, это не совсем состояние, но логически для пользователя оно существует.

\begin{figure}[htbp]
  \centering
  \includegraphics[width=0.7\textwidth]{./processes-and-threads/processes/lifecycle/proc-lifecycle.png}
\end{figure}

Когда инициализация процесса завершается --- он оказывается в стадии \textbf{“готов“}. С этого момента процесс находится в ожидании момента, когда его выберет планировщик процессов и даст ему какой-то квант процессорного времени.

При получении процессорного времени процесс переходит в состояние \textbf{“выполняется“}. Выполняться процесс может в режиме ядра --- при осуществлении системных вызовов, прерываний и в режиме задачи --- выполнять инструкции процессора. По окончании квоты времени, процесс может снова вернуться в состояние \textbf{“готов“}.

Также из состояния “выполняется“ процесс может перейти еще в два состояния --- \textbf{“остановлен“} и \textbf{“ожидает“}. Если он остановлен, то он остановлен пользователем (например, получен сигнал SIGSTOP). Ожидание наступает тогда, когда процессу нужны какие-то ресурсы. Как только условие, которого ждет процессор, выполняется, он переходит в состояние \textbf{“готов“}, то есть ожидает своей очереди на выполнение. 

Из каждого состояния процесс может перейти в состояние \textbf{“зомби“}. Этот процесс нужен для того, чтобы родительский процесс мог получить код возврата этого процесса. Это промежуточное состояние --- процесса уже нет, но он технически еще есть. Тем не менее, это самое что ни на есть валидное состояние. 

\textbf{В какой момент процесс окончательно исчезает из таблицы процессов?}

После того как его родительский процесс вызовет wait (2), waitpid (2), waitid(2) - позволяет завершить процесс, если он таки стал зомби. 

Традиционная проблема --- порождая процесс, необходимо в какой-то момент сказать ему wait (2), чтобы не плодить зомби. В противном случае у вас может закончится лимит на создание новых процессов. 

\textbf{Что происходит с зомби, если его родитель не вызвал wait (2), а погиб?}

Его родителем становится init. Для всех процессов, которые становятся его потомками он делает wait (2). По факту, если вы написали какой-то код, который выполняет некоторое количество процессов после чего завершился, то на самом деле зомби в системе не останутся, т.к. после того как ваш основной процесс завершился, то потомков этого процесса подхватит init и он сам вызовет для них wait (2).


         \subsection{Аргументы main-функции}
         Жизненный цикл процесса начинается с того, что его кто-то порождает. Единственный процесс, который никто не порождает, это процесс init. Порождение процесса осуществляется с помощью fork (2) или vfork (2). 

Первая стадия в жизни процесса --- \textbf{“инициализация“}. В это время ядро производит подготовительные работы к дальнейшей работе процесса. На самом деле, это не совсем состояние, но логически для пользователя оно существует.

\begin{figure}[htbp]
  \centering
  \includegraphics[width=0.7\textwidth]{./processes-and-threads/processes/lifecycle/proc-lifecycle.png}
\end{figure}

Когда инициализация процесса завершается --- он оказывается в стадии \textbf{“готов“}. С этого момента процесс находится в ожидании момента, когда его выберет планировщик процессов и даст ему какой-то квант процессорного времени.

При получении процессорного времени процесс переходит в состояние \textbf{“выполняется“}. Выполняться процесс может в режиме ядра --- при осуществлении системных вызовов, прерываний и в режиме задачи --- выполнять инструкции процессора. По окончании квоты времени, процесс может снова вернуться в состояние \textbf{“готов“}.

Также из состояния “выполняется“ процесс может перейти еще в два состояния --- \textbf{“остановлен“} и \textbf{“ожидает“}. Если он остановлен, то он остановлен пользователем (например, получен сигнал SIGSTOP). Ожидание наступает тогда, когда процессу нужны какие-то ресурсы. Как только условие, которого ждет процессор, выполняется, он переходит в состояние \textbf{“готов“}, то есть ожидает своей очереди на выполнение. 

Из каждого состояния процесс может перейти в состояние \textbf{“зомби“}. Этот процесс нужен для того, чтобы родительский процесс мог получить код возврата этого процесса. Это промежуточное состояние --- процесса уже нет, но он технически еще есть. Тем не менее, это самое что ни на есть валидное состояние. 

\textbf{В какой момент процесс окончательно исчезает из таблицы процессов?}

После того как его родительский процесс вызовет wait (2), waitpid (2), waitid(2) - позволяет завершить процесс, если он таки стал зомби. 

Традиционная проблема --- порождая процесс, необходимо в какой-то момент сказать ему wait (2), чтобы не плодить зомби. В противном случае у вас может закончится лимит на создание новых процессов. 

\textbf{Что происходит с зомби, если его родитель не вызвал wait (2), а погиб?}

Его родителем становится init. Для всех процессов, которые становятся его потомками он делает wait (2). По факту, если вы написали какой-то код, который выполняет некоторое количество процессов после чего завершился, то на самом деле зомби в системе не останутся, т.к. после того как ваш основной процесс завершился, то потомков этого процесса подхватит init и он сам вызовет для них wait (2).


      \section{Код возврата}
      Жизненный цикл процесса начинается с того, что его кто-то порождает. Единственный процесс, который никто не порождает, это процесс init. Порождение процесса осуществляется с помощью fork (2) или vfork (2). 

Первая стадия в жизни процесса --- \textbf{“инициализация“}. В это время ядро производит подготовительные работы к дальнейшей работе процесса. На самом деле, это не совсем состояние, но логически для пользователя оно существует.

\begin{figure}[htbp]
  \centering
  \includegraphics[width=0.7\textwidth]{./processes-and-threads/processes/lifecycle/proc-lifecycle.png}
\end{figure}

Когда инициализация процесса завершается --- он оказывается в стадии \textbf{“готов“}. С этого момента процесс находится в ожидании момента, когда его выберет планировщик процессов и даст ему какой-то квант процессорного времени.

При получении процессорного времени процесс переходит в состояние \textbf{“выполняется“}. Выполняться процесс может в режиме ядра --- при осуществлении системных вызовов, прерываний и в режиме задачи --- выполнять инструкции процессора. По окончании квоты времени, процесс может снова вернуться в состояние \textbf{“готов“}.

Также из состояния “выполняется“ процесс может перейти еще в два состояния --- \textbf{“остановлен“} и \textbf{“ожидает“}. Если он остановлен, то он остановлен пользователем (например, получен сигнал SIGSTOP). Ожидание наступает тогда, когда процессу нужны какие-то ресурсы. Как только условие, которого ждет процессор, выполняется, он переходит в состояние \textbf{“готов“}, то есть ожидает своей очереди на выполнение. 

Из каждого состояния процесс может перейти в состояние \textbf{“зомби“}. Этот процесс нужен для того, чтобы родительский процесс мог получить код возврата этого процесса. Это промежуточное состояние --- процесса уже нет, но он технически еще есть. Тем не менее, это самое что ни на есть валидное состояние. 

\textbf{В какой момент процесс окончательно исчезает из таблицы процессов?}

После того как его родительский процесс вызовет wait (2), waitpid (2), waitid(2) - позволяет завершить процесс, если он таки стал зомби. 

Традиционная проблема --- порождая процесс, необходимо в какой-то момент сказать ему wait (2), чтобы не плодить зомби. В противном случае у вас может закончится лимит на создание новых процессов. 

\textbf{Что происходит с зомби, если его родитель не вызвал wait (2), а погиб?}

Его родителем становится init. Для всех процессов, которые становятся его потомками он делает wait (2). По факту, если вы написали какой-то код, который выполняет некоторое количество процессов после чего завершился, то на самом деле зомби в системе не останутся, т.к. после того как ваш основной процесс завершился, то потомков этого процесса подхватит init и он сам вызовет для них wait (2).


      \section{Ошибки и стандартизация ошибок}
      Теперь поговорим об ошибках, возникающих при работе с системными вызовами, и о том, как можно узнать, почему тот или иной вызов не завершился успешно.

Системный вызов является либо вызовом инструкции int, либо вызовом инструкции syscall. При выполнение этой инструкции, система возвращает статус системного вызова через регистр АХ вызова (то есть код возврата). Помним, что неотрицательный код возврата, это успешное выполнение, отрицательный (чаще всего -1) --- ошибка. 

Однако, нам часто хочется узнать более точную причину произошедшей ошибки, а не просто факт ее возникновения. Для осуществления этой возможности и была придумана специальная \textbf{переменная errno}. Эта переменнтая содержится в заголовочном файле errno.h и имеет тип external int.

Помимо errno в <errno.h> хранятся также и директивы с определением ошибок и их кодов.


\begin{CCode}{Пример содержимого errno.h}
	...
	#define ENODEV      19  /* No such device */
	#define ENOTDIR     20  /* Not a directory */
	#define EISDIR      21  /* Is a directory */ 
	... \end{CCode}

\textbf{Как это работает?}

Дело в том, что в случае возникновения ошибки, системный вызов не только возвращает отрицательное значение. Он также устанавливает значение errno в зависимости от того, по какой причине эта самая ошибка случилась (если ошибки не произошло, значение errno не изменяется). 

\textbf{В чем еще преимущество errno?}

В разных ОС коды возврата системных вызовов они чуть-чуть отличаются друг от друга, поэтому если наша программа анализировала содержимое AX регистра, то мы бы устали для каждой ОС делать исключения. Libc предлагает нам унифицировать не только ввод-вывод, но и взаимодействие системных вызовов. 

Как это работает? Чтобы программистам на языке Си было проще, код возврата функции -- обертки системного вызова приводится к форме отрицательного числа уже средствами самой libC. 


         \subsection{perror (3) и strerror (3)}
         Жизненный цикл процесса начинается с того, что его кто-то порождает. Единственный процесс, который никто не порождает, это процесс init. Порождение процесса осуществляется с помощью fork (2) или vfork (2). 

Первая стадия в жизни процесса --- \textbf{“инициализация“}. В это время ядро производит подготовительные работы к дальнейшей работе процесса. На самом деле, это не совсем состояние, но логически для пользователя оно существует.

\begin{figure}[htbp]
  \centering
  \includegraphics[width=0.7\textwidth]{./processes-and-threads/processes/lifecycle/proc-lifecycle.png}
\end{figure}

Когда инициализация процесса завершается --- он оказывается в стадии \textbf{“готов“}. С этого момента процесс находится в ожидании момента, когда его выберет планировщик процессов и даст ему какой-то квант процессорного времени.

При получении процессорного времени процесс переходит в состояние \textbf{“выполняется“}. Выполняться процесс может в режиме ядра --- при осуществлении системных вызовов, прерываний и в режиме задачи --- выполнять инструкции процессора. По окончании квоты времени, процесс может снова вернуться в состояние \textbf{“готов“}.

Также из состояния “выполняется“ процесс может перейти еще в два состояния --- \textbf{“остановлен“} и \textbf{“ожидает“}. Если он остановлен, то он остановлен пользователем (например, получен сигнал SIGSTOP). Ожидание наступает тогда, когда процессу нужны какие-то ресурсы. Как только условие, которого ждет процессор, выполняется, он переходит в состояние \textbf{“готов“}, то есть ожидает своей очереди на выполнение. 

Из каждого состояния процесс может перейти в состояние \textbf{“зомби“}. Этот процесс нужен для того, чтобы родительский процесс мог получить код возврата этого процесса. Это промежуточное состояние --- процесса уже нет, но он технически еще есть. Тем не менее, это самое что ни на есть валидное состояние. 

\textbf{В какой момент процесс окончательно исчезает из таблицы процессов?}

После того как его родительский процесс вызовет wait (2), waitpid (2), waitid(2) - позволяет завершить процесс, если он таки стал зомби. 

Традиционная проблема --- порождая процесс, необходимо в какой-то момент сказать ему wait (2), чтобы не плодить зомби. В противном случае у вас может закончится лимит на создание новых процессов. 

\textbf{Что происходит с зомби, если его родитель не вызвал wait (2), а погиб?}

Его родителем становится init. Для всех процессов, которые становятся его потомками он делает wait (2). По факту, если вы написали какой-то код, который выполняет некоторое количество процессов после чего завершился, то на самом деле зомби в системе не останутся, т.к. после того как ваш основной процесс завершился, то потомков этого процесса подхватит init и он сам вызовет для них wait (2).


   \chapter{Компиляция программы}
   Жизненный цикл процесса начинается с того, что его кто-то порождает. Единственный процесс, который никто не порождает, это процесс init. Порождение процесса осуществляется с помощью fork (2) или vfork (2). 

Первая стадия в жизни процесса --- \textbf{“инициализация“}. В это время ядро производит подготовительные работы к дальнейшей работе процесса. На самом деле, это не совсем состояние, но логически для пользователя оно существует.

\begin{figure}[htbp]
  \centering
  \includegraphics[width=0.7\textwidth]{./processes-and-threads/processes/lifecycle/proc-lifecycle.png}
\end{figure}

Когда инициализация процесса завершается --- он оказывается в стадии \textbf{“готов“}. С этого момента процесс находится в ожидании момента, когда его выберет планировщик процессов и даст ему какой-то квант процессорного времени.

При получении процессорного времени процесс переходит в состояние \textbf{“выполняется“}. Выполняться процесс может в режиме ядра --- при осуществлении системных вызовов, прерываний и в режиме задачи --- выполнять инструкции процессора. По окончании квоты времени, процесс может снова вернуться в состояние \textbf{“готов“}.

Также из состояния “выполняется“ процесс может перейти еще в два состояния --- \textbf{“остановлен“} и \textbf{“ожидает“}. Если он остановлен, то он остановлен пользователем (например, получен сигнал SIGSTOP). Ожидание наступает тогда, когда процессу нужны какие-то ресурсы. Как только условие, которого ждет процессор, выполняется, он переходит в состояние \textbf{“готов“}, то есть ожидает своей очереди на выполнение. 

Из каждого состояния процесс может перейти в состояние \textbf{“зомби“}. Этот процесс нужен для того, чтобы родительский процесс мог получить код возврата этого процесса. Это промежуточное состояние --- процесса уже нет, но он технически еще есть. Тем не менее, это самое что ни на есть валидное состояние. 

\textbf{В какой момент процесс окончательно исчезает из таблицы процессов?}

После того как его родительский процесс вызовет wait (2), waitpid (2), waitid(2) - позволяет завершить процесс, если он таки стал зомби. 

Традиционная проблема --- порождая процесс, необходимо в какой-то момент сказать ему wait (2), чтобы не плодить зомби. В противном случае у вас может закончится лимит на создание новых процессов. 

\textbf{Что происходит с зомби, если его родитель не вызвал wait (2), а погиб?}

Его родителем становится init. Для всех процессов, которые становятся его потомками он делает wait (2). По факту, если вы написали какой-то код, который выполняет некоторое количество процессов после чего завершился, то на самом деле зомби в системе не останутся, т.к. после того как ваш основной процесс завершился, то потомков этого процесса подхватит init и он сам вызовет для них wait (2).


      \section{Компиляторы}
      Представим некоторые компиляторы подходящие для языка Си:

\begin{itemize}
		
	\item
	\begin{myenv}{cc}{C compiler}
	\end{myenv}
			
	\item
	\begin{myenv}{gcc}{GNU project C and C++ compiler}
	\end{myenv}
			
	\item
	\begin{myenv}{clang}{C and Objective-C compiler}
	\end{myenv}
			
\end{itemize}


      \section{Makefile и утилита make}
      Для автоматизации сборки программ существуют две полезные вещи --- утилита make и Makefile.

Утилита make предназначена для автоматизации преобразования файлов из одной формы в другую. Правила преобразования задаются в скрипте с именем Makefile, который должен находиться в корне рабочей директории проекта.

Рассмотрим структуру Makefile на примере.

\begin{shCode}{Makefile}
	CC=gcc
	CFLAGS=-m64

	all: $(PROJS)
		@echo Done!
	main:
		$(CC) $(CFLAGS) -o $@ $(@:=.c) \end{shCode}

В начале объявляются переменные, затем конструкция вида:

<цель> : <зависимости> \\
<список команд>
			
\begin{shCode}{Утилита make}
	ag@helios:/home/ag $ make
	gcc -m64 -o main main.c
	Done!
	ag@helios:/home/ag $ ./main
	Inspiration unlocks the future \end{shCode}

При команде \textbf{make <цель>} будут выполнены команды соответствующие метке <цель>, а также последовательно все зависимости(другие метки), если они не являются файлами.

В нашем примере make по умолчанию перейдет к цели all, увидит зависимость main и перейдет на эту метку. Далее так как зависимостей у main нет, make выполнит команду “gcc -m64 -o main main.c“, а затем вернется к метке all и выполнит команду echo.

Makefile имеет специальные собственные переменные.

Приведем некоторые из них:

\begin{itemize}
	\item \$@  --- имя текущей цели;

	\item \$<  --- имя первой зависимости Makefile;

	\item \$\textasciicircum  --- имена всех зависимостей.
\end{itemize}



      % ------------------------------
      % Часть: Файлы и файловая система

\clearpage \part{Файлы и файловая подсистема}

   \chapter{Все есть файл! (Кроме потоков и ядра)}
   Жизненный цикл процесса начинается с того, что его кто-то порождает. Единственный процесс, который никто не порождает, это процесс init. Порождение процесса осуществляется с помощью fork (2) или vfork (2). 

Первая стадия в жизни процесса --- \textbf{“инициализация“}. В это время ядро производит подготовительные работы к дальнейшей работе процесса. На самом деле, это не совсем состояние, но логически для пользователя оно существует.

\begin{figure}[htbp]
  \centering
  \includegraphics[width=0.7\textwidth]{./processes-and-threads/processes/lifecycle/proc-lifecycle.png}
\end{figure}

Когда инициализация процесса завершается --- он оказывается в стадии \textbf{“готов“}. С этого момента процесс находится в ожидании момента, когда его выберет планировщик процессов и даст ему какой-то квант процессорного времени.

При получении процессорного времени процесс переходит в состояние \textbf{“выполняется“}. Выполняться процесс может в режиме ядра --- при осуществлении системных вызовов, прерываний и в режиме задачи --- выполнять инструкции процессора. По окончании квоты времени, процесс может снова вернуться в состояние \textbf{“готов“}.

Также из состояния “выполняется“ процесс может перейти еще в два состояния --- \textbf{“остановлен“} и \textbf{“ожидает“}. Если он остановлен, то он остановлен пользователем (например, получен сигнал SIGSTOP). Ожидание наступает тогда, когда процессу нужны какие-то ресурсы. Как только условие, которого ждет процессор, выполняется, он переходит в состояние \textbf{“готов“}, то есть ожидает своей очереди на выполнение. 

Из каждого состояния процесс может перейти в состояние \textbf{“зомби“}. Этот процесс нужен для того, чтобы родительский процесс мог получить код возврата этого процесса. Это промежуточное состояние --- процесса уже нет, но он технически еще есть. Тем не менее, это самое что ни на есть валидное состояние. 

\textbf{В какой момент процесс окончательно исчезает из таблицы процессов?}

После того как его родительский процесс вызовет wait (2), waitpid (2), waitid(2) - позволяет завершить процесс, если он таки стал зомби. 

Традиционная проблема --- порождая процесс, необходимо в какой-то момент сказать ему wait (2), чтобы не плодить зомби. В противном случае у вас может закончится лимит на создание новых процессов. 

\textbf{Что происходит с зомби, если его родитель не вызвал wait (2), а погиб?}

Его родителем становится init. Для всех процессов, которые становятся его потомками он делает wait (2). По факту, если вы написали какой-то код, который выполняет некоторое количество процессов после чего завершился, то на самом деле зомби в системе не останутся, т.к. после того как ваш основной процесс завершился, то потомков этого процесса подхватит init и он сам вызовет для них wait (2).


      \section{inode}
      Жизненный цикл процесса начинается с того, что его кто-то порождает. Единственный процесс, который никто не порождает, это процесс init. Порождение процесса осуществляется с помощью fork (2) или vfork (2). 

Первая стадия в жизни процесса --- \textbf{“инициализация“}. В это время ядро производит подготовительные работы к дальнейшей работе процесса. На самом деле, это не совсем состояние, но логически для пользователя оно существует.

\begin{figure}[htbp]
  \centering
  \includegraphics[width=0.7\textwidth]{./processes-and-threads/processes/lifecycle/proc-lifecycle.png}
\end{figure}

Когда инициализация процесса завершается --- он оказывается в стадии \textbf{“готов“}. С этого момента процесс находится в ожидании момента, когда его выберет планировщик процессов и даст ему какой-то квант процессорного времени.

При получении процессорного времени процесс переходит в состояние \textbf{“выполняется“}. Выполняться процесс может в режиме ядра --- при осуществлении системных вызовов, прерываний и в режиме задачи --- выполнять инструкции процессора. По окончании квоты времени, процесс может снова вернуться в состояние \textbf{“готов“}.

Также из состояния “выполняется“ процесс может перейти еще в два состояния --- \textbf{“остановлен“} и \textbf{“ожидает“}. Если он остановлен, то он остановлен пользователем (например, получен сигнал SIGSTOP). Ожидание наступает тогда, когда процессу нужны какие-то ресурсы. Как только условие, которого ждет процессор, выполняется, он переходит в состояние \textbf{“готов“}, то есть ожидает своей очереди на выполнение. 

Из каждого состояния процесс может перейти в состояние \textbf{“зомби“}. Этот процесс нужен для того, чтобы родительский процесс мог получить код возврата этого процесса. Это промежуточное состояние --- процесса уже нет, но он технически еще есть. Тем не менее, это самое что ни на есть валидное состояние. 

\textbf{В какой момент процесс окончательно исчезает из таблицы процессов?}

После того как его родительский процесс вызовет wait (2), waitpid (2), waitid(2) - позволяет завершить процесс, если он таки стал зомби. 

Традиционная проблема --- порождая процесс, необходимо в какой-то момент сказать ему wait (2), чтобы не плодить зомби. В противном случае у вас может закончится лимит на создание новых процессов. 

\textbf{Что происходит с зомби, если его родитель не вызвал wait (2), а погиб?}

Его родителем становится init. Для всех процессов, которые становятся его потомками он делает wait (2). По факту, если вы написали какой-то код, который выполняет некоторое количество процессов после чего завершился, то на самом деле зомби в системе не останутся, т.к. после того как ваш основной процесс завершился, то потомков этого процесса подхватит init и он сам вызовет для них wait (2).


         \section{Структура stat}
         Жизненный цикл процесса начинается с того, что его кто-то порождает. Единственный процесс, который никто не порождает, это процесс init. Порождение процесса осуществляется с помощью fork (2) или vfork (2). 

Первая стадия в жизни процесса --- \textbf{“инициализация“}. В это время ядро производит подготовительные работы к дальнейшей работе процесса. На самом деле, это не совсем состояние, но логически для пользователя оно существует.

\begin{figure}[htbp]
  \centering
  \includegraphics[width=0.7\textwidth]{./processes-and-threads/processes/lifecycle/proc-lifecycle.png}
\end{figure}

Когда инициализация процесса завершается --- он оказывается в стадии \textbf{“готов“}. С этого момента процесс находится в ожидании момента, когда его выберет планировщик процессов и даст ему какой-то квант процессорного времени.

При получении процессорного времени процесс переходит в состояние \textbf{“выполняется“}. Выполняться процесс может в режиме ядра --- при осуществлении системных вызовов, прерываний и в режиме задачи --- выполнять инструкции процессора. По окончании квоты времени, процесс может снова вернуться в состояние \textbf{“готов“}.

Также из состояния “выполняется“ процесс может перейти еще в два состояния --- \textbf{“остановлен“} и \textbf{“ожидает“}. Если он остановлен, то он остановлен пользователем (например, получен сигнал SIGSTOP). Ожидание наступает тогда, когда процессу нужны какие-то ресурсы. Как только условие, которого ждет процессор, выполняется, он переходит в состояние \textbf{“готов“}, то есть ожидает своей очереди на выполнение. 

Из каждого состояния процесс может перейти в состояние \textbf{“зомби“}. Этот процесс нужен для того, чтобы родительский процесс мог получить код возврата этого процесса. Это промежуточное состояние --- процесса уже нет, но он технически еще есть. Тем не менее, это самое что ни на есть валидное состояние. 

\textbf{В какой момент процесс окончательно исчезает из таблицы процессов?}

После того как его родительский процесс вызовет wait (2), waitpid (2), waitid(2) - позволяет завершить процесс, если он таки стал зомби. 

Традиционная проблема --- порождая процесс, необходимо в какой-то момент сказать ему wait (2), чтобы не плодить зомби. В противном случае у вас может закончится лимит на создание новых процессов. 

\textbf{Что происходит с зомби, если его родитель не вызвал wait (2), а погиб?}

Его родителем становится init. Для всех процессов, которые становятся его потомками он делает wait (2). По факту, если вы написали какой-то код, который выполняет некоторое количество процессов после чего завершился, то на самом деле зомби в системе не останутся, т.к. после того как ваш основной процесс завершился, то потомков этого процесса подхватит init и он сам вызовет для них wait (2).


      \section{Файловый дескриптор}
      Жизненный цикл процесса начинается с того, что его кто-то порождает. Единственный процесс, который никто не порождает, это процесс init. Порождение процесса осуществляется с помощью fork (2) или vfork (2). 

Первая стадия в жизни процесса --- \textbf{“инициализация“}. В это время ядро производит подготовительные работы к дальнейшей работе процесса. На самом деле, это не совсем состояние, но логически для пользователя оно существует.

\begin{figure}[htbp]
  \centering
  \includegraphics[width=0.7\textwidth]{./processes-and-threads/processes/lifecycle/proc-lifecycle.png}
\end{figure}

Когда инициализация процесса завершается --- он оказывается в стадии \textbf{“готов“}. С этого момента процесс находится в ожидании момента, когда его выберет планировщик процессов и даст ему какой-то квант процессорного времени.

При получении процессорного времени процесс переходит в состояние \textbf{“выполняется“}. Выполняться процесс может в режиме ядра --- при осуществлении системных вызовов, прерываний и в режиме задачи --- выполнять инструкции процессора. По окончании квоты времени, процесс может снова вернуться в состояние \textbf{“готов“}.

Также из состояния “выполняется“ процесс может перейти еще в два состояния --- \textbf{“остановлен“} и \textbf{“ожидает“}. Если он остановлен, то он остановлен пользователем (например, получен сигнал SIGSTOP). Ожидание наступает тогда, когда процессу нужны какие-то ресурсы. Как только условие, которого ждет процессор, выполняется, он переходит в состояние \textbf{“готов“}, то есть ожидает своей очереди на выполнение. 

Из каждого состояния процесс может перейти в состояние \textbf{“зомби“}. Этот процесс нужен для того, чтобы родительский процесс мог получить код возврата этого процесса. Это промежуточное состояние --- процесса уже нет, но он технически еще есть. Тем не менее, это самое что ни на есть валидное состояние. 

\textbf{В какой момент процесс окончательно исчезает из таблицы процессов?}

После того как его родительский процесс вызовет wait (2), waitpid (2), waitid(2) - позволяет завершить процесс, если он таки стал зомби. 

Традиционная проблема --- порождая процесс, необходимо в какой-то момент сказать ему wait (2), чтобы не плодить зомби. В противном случае у вас может закончится лимит на создание новых процессов. 

\textbf{Что происходит с зомби, если его родитель не вызвал wait (2), а погиб?}

Его родителем становится init. Для всех процессов, которые становятся его потомками он делает wait (2). По факту, если вы написали какой-то код, который выполняет некоторое количество процессов после чего завершился, то на самом деле зомби в системе не останутся, т.к. после того как ваш основной процесс завершился, то потомков этого процесса подхватит init и он сам вызовет для них wait (2).


         \subsection{Работа с файловым дескриптором}
         Жизненный цикл процесса начинается с того, что его кто-то порождает. Единственный процесс, который никто не порождает, это процесс init. Порождение процесса осуществляется с помощью fork (2) или vfork (2). 

Первая стадия в жизни процесса --- \textbf{“инициализация“}. В это время ядро производит подготовительные работы к дальнейшей работе процесса. На самом деле, это не совсем состояние, но логически для пользователя оно существует.

\begin{figure}[htbp]
  \centering
  \includegraphics[width=0.7\textwidth]{./processes-and-threads/processes/lifecycle/proc-lifecycle.png}
\end{figure}

Когда инициализация процесса завершается --- он оказывается в стадии \textbf{“готов“}. С этого момента процесс находится в ожидании момента, когда его выберет планировщик процессов и даст ему какой-то квант процессорного времени.

При получении процессорного времени процесс переходит в состояние \textbf{“выполняется“}. Выполняться процесс может в режиме ядра --- при осуществлении системных вызовов, прерываний и в режиме задачи --- выполнять инструкции процессора. По окончании квоты времени, процесс может снова вернуться в состояние \textbf{“готов“}.

Также из состояния “выполняется“ процесс может перейти еще в два состояния --- \textbf{“остановлен“} и \textbf{“ожидает“}. Если он остановлен, то он остановлен пользователем (например, получен сигнал SIGSTOP). Ожидание наступает тогда, когда процессу нужны какие-то ресурсы. Как только условие, которого ждет процессор, выполняется, он переходит в состояние \textbf{“готов“}, то есть ожидает своей очереди на выполнение. 

Из каждого состояния процесс может перейти в состояние \textbf{“зомби“}. Этот процесс нужен для того, чтобы родительский процесс мог получить код возврата этого процесса. Это промежуточное состояние --- процесса уже нет, но он технически еще есть. Тем не менее, это самое что ни на есть валидное состояние. 

\textbf{В какой момент процесс окончательно исчезает из таблицы процессов?}

После того как его родительский процесс вызовет wait (2), waitpid (2), waitid(2) - позволяет завершить процесс, если он таки стал зомби. 

Традиционная проблема --- порождая процесс, необходимо в какой-то момент сказать ему wait (2), чтобы не плодить зомби. В противном случае у вас может закончится лимит на создание новых процессов. 

\textbf{Что происходит с зомби, если его родитель не вызвал wait (2), а погиб?}

Его родителем становится init. Для всех процессов, которые становятся его потомками он делает wait (2). По факту, если вы написали какой-то код, который выполняет некоторое количество процессов после чего завершился, то на самом деле зомби в системе не останутся, т.к. после того как ваш основной процесс завершился, то потомков этого процесса подхватит init и он сам вызовет для них wait (2).


         \subsection{fcntl}
         Жизненный цикл процесса начинается с того, что его кто-то порождает. Единственный процесс, который никто не порождает, это процесс init. Порождение процесса осуществляется с помощью fork (2) или vfork (2). 

Первая стадия в жизни процесса --- \textbf{“инициализация“}. В это время ядро производит подготовительные работы к дальнейшей работе процесса. На самом деле, это не совсем состояние, но логически для пользователя оно существует.

\begin{figure}[htbp]
  \centering
  \includegraphics[width=0.7\textwidth]{./processes-and-threads/processes/lifecycle/proc-lifecycle.png}
\end{figure}

Когда инициализация процесса завершается --- он оказывается в стадии \textbf{“готов“}. С этого момента процесс находится в ожидании момента, когда его выберет планировщик процессов и даст ему какой-то квант процессорного времени.

При получении процессорного времени процесс переходит в состояние \textbf{“выполняется“}. Выполняться процесс может в режиме ядра --- при осуществлении системных вызовов, прерываний и в режиме задачи --- выполнять инструкции процессора. По окончании квоты времени, процесс может снова вернуться в состояние \textbf{“готов“}.

Также из состояния “выполняется“ процесс может перейти еще в два состояния --- \textbf{“остановлен“} и \textbf{“ожидает“}. Если он остановлен, то он остановлен пользователем (например, получен сигнал SIGSTOP). Ожидание наступает тогда, когда процессу нужны какие-то ресурсы. Как только условие, которого ждет процессор, выполняется, он переходит в состояние \textbf{“готов“}, то есть ожидает своей очереди на выполнение. 

Из каждого состояния процесс может перейти в состояние \textbf{“зомби“}. Этот процесс нужен для того, чтобы родительский процесс мог получить код возврата этого процесса. Это промежуточное состояние --- процесса уже нет, но он технически еще есть. Тем не менее, это самое что ни на есть валидное состояние. 

\textbf{В какой момент процесс окончательно исчезает из таблицы процессов?}

После того как его родительский процесс вызовет wait (2), waitpid (2), waitid(2) - позволяет завершить процесс, если он таки стал зомби. 

Традиционная проблема --- порождая процесс, необходимо в какой-то момент сказать ему wait (2), чтобы не плодить зомби. В противном случае у вас может закончится лимит на создание новых процессов. 

\textbf{Что происходит с зомби, если его родитель не вызвал wait (2), а погиб?}

Его родителем становится init. Для всех процессов, которые становятся его потомками он делает wait (2). По факту, если вы написали какой-то код, который выполняет некоторое количество процессов после чего завершился, то на самом деле зомби в системе не останутся, т.к. после того как ваш основной процесс завершился, то потомков этого процесса подхватит init и он сам вызовет для них wait (2).


      \section{Типы файлов}
      Жизненный цикл процесса начинается с того, что его кто-то порождает. Единственный процесс, который никто не порождает, это процесс init. Порождение процесса осуществляется с помощью fork (2) или vfork (2). 

Первая стадия в жизни процесса --- \textbf{“инициализация“}. В это время ядро производит подготовительные работы к дальнейшей работе процесса. На самом деле, это не совсем состояние, но логически для пользователя оно существует.

\begin{figure}[htbp]
  \centering
  \includegraphics[width=0.7\textwidth]{./processes-and-threads/processes/lifecycle/proc-lifecycle.png}
\end{figure}

Когда инициализация процесса завершается --- он оказывается в стадии \textbf{“готов“}. С этого момента процесс находится в ожидании момента, когда его выберет планировщик процессов и даст ему какой-то квант процессорного времени.

При получении процессорного времени процесс переходит в состояние \textbf{“выполняется“}. Выполняться процесс может в режиме ядра --- при осуществлении системных вызовов, прерываний и в режиме задачи --- выполнять инструкции процессора. По окончании квоты времени, процесс может снова вернуться в состояние \textbf{“готов“}.

Также из состояния “выполняется“ процесс может перейти еще в два состояния --- \textbf{“остановлен“} и \textbf{“ожидает“}. Если он остановлен, то он остановлен пользователем (например, получен сигнал SIGSTOP). Ожидание наступает тогда, когда процессу нужны какие-то ресурсы. Как только условие, которого ждет процессор, выполняется, он переходит в состояние \textbf{“готов“}, то есть ожидает своей очереди на выполнение. 

Из каждого состояния процесс может перейти в состояние \textbf{“зомби“}. Этот процесс нужен для того, чтобы родительский процесс мог получить код возврата этого процесса. Это промежуточное состояние --- процесса уже нет, но он технически еще есть. Тем не менее, это самое что ни на есть валидное состояние. 

\textbf{В какой момент процесс окончательно исчезает из таблицы процессов?}

После того как его родительский процесс вызовет wait (2), waitpid (2), waitid(2) - позволяет завершить процесс, если он таки стал зомби. 

Традиционная проблема --- порождая процесс, необходимо в какой-то момент сказать ему wait (2), чтобы не плодить зомби. В противном случае у вас может закончится лимит на создание новых процессов. 

\textbf{Что происходит с зомби, если его родитель не вызвал wait (2), а погиб?}

Его родителем становится init. Для всех процессов, которые становятся его потомками он делает wait (2). По факту, если вы написали какой-то код, который выполняет некоторое количество процессов после чего завершился, то на самом деле зомби в системе не останутся, т.к. после того как ваш основной процесс завершился, то потомков этого процесса подхватит init и он сам вызовет для них wait (2).


   \chapter{Файловый ввод-вывод}

      \section{Открытие и закрытие файла}
      Жизненный цикл процесса начинается с того, что его кто-то порождает. Единственный процесс, который никто не порождает, это процесс init. Порождение процесса осуществляется с помощью fork (2) или vfork (2). 

Первая стадия в жизни процесса --- \textbf{“инициализация“}. В это время ядро производит подготовительные работы к дальнейшей работе процесса. На самом деле, это не совсем состояние, но логически для пользователя оно существует.

\begin{figure}[htbp]
  \centering
  \includegraphics[width=0.7\textwidth]{./processes-and-threads/processes/lifecycle/proc-lifecycle.png}
\end{figure}

Когда инициализация процесса завершается --- он оказывается в стадии \textbf{“готов“}. С этого момента процесс находится в ожидании момента, когда его выберет планировщик процессов и даст ему какой-то квант процессорного времени.

При получении процессорного времени процесс переходит в состояние \textbf{“выполняется“}. Выполняться процесс может в режиме ядра --- при осуществлении системных вызовов, прерываний и в режиме задачи --- выполнять инструкции процессора. По окончании квоты времени, процесс может снова вернуться в состояние \textbf{“готов“}.

Также из состояния “выполняется“ процесс может перейти еще в два состояния --- \textbf{“остановлен“} и \textbf{“ожидает“}. Если он остановлен, то он остановлен пользователем (например, получен сигнал SIGSTOP). Ожидание наступает тогда, когда процессу нужны какие-то ресурсы. Как только условие, которого ждет процессор, выполняется, он переходит в состояние \textbf{“готов“}, то есть ожидает своей очереди на выполнение. 

Из каждого состояния процесс может перейти в состояние \textbf{“зомби“}. Этот процесс нужен для того, чтобы родительский процесс мог получить код возврата этого процесса. Это промежуточное состояние --- процесса уже нет, но он технически еще есть. Тем не менее, это самое что ни на есть валидное состояние. 

\textbf{В какой момент процесс окончательно исчезает из таблицы процессов?}

После того как его родительский процесс вызовет wait (2), waitpid (2), waitid(2) - позволяет завершить процесс, если он таки стал зомби. 

Традиционная проблема --- порождая процесс, необходимо в какой-то момент сказать ему wait (2), чтобы не плодить зомби. В противном случае у вас может закончится лимит на создание новых процессов. 

\textbf{Что происходит с зомби, если его родитель не вызвал wait (2), а погиб?}

Его родителем становится init. Для всех процессов, которые становятся его потомками он делает wait (2). По факту, если вы написали какой-то код, который выполняет некоторое количество процессов после чего завершился, то на самом деле зомби в системе не останутся, т.к. после того как ваш основной процесс завершился, то потомков этого процесса подхватит init и он сам вызовет для них wait (2).


      \section{Перемещение внутри файла}
      Жизненный цикл процесса начинается с того, что его кто-то порождает. Единственный процесс, который никто не порождает, это процесс init. Порождение процесса осуществляется с помощью fork (2) или vfork (2). 

Первая стадия в жизни процесса --- \textbf{“инициализация“}. В это время ядро производит подготовительные работы к дальнейшей работе процесса. На самом деле, это не совсем состояние, но логически для пользователя оно существует.

\begin{figure}[htbp]
  \centering
  \includegraphics[width=0.7\textwidth]{./processes-and-threads/processes/lifecycle/proc-lifecycle.png}
\end{figure}

Когда инициализация процесса завершается --- он оказывается в стадии \textbf{“готов“}. С этого момента процесс находится в ожидании момента, когда его выберет планировщик процессов и даст ему какой-то квант процессорного времени.

При получении процессорного времени процесс переходит в состояние \textbf{“выполняется“}. Выполняться процесс может в режиме ядра --- при осуществлении системных вызовов, прерываний и в режиме задачи --- выполнять инструкции процессора. По окончании квоты времени, процесс может снова вернуться в состояние \textbf{“готов“}.

Также из состояния “выполняется“ процесс может перейти еще в два состояния --- \textbf{“остановлен“} и \textbf{“ожидает“}. Если он остановлен, то он остановлен пользователем (например, получен сигнал SIGSTOP). Ожидание наступает тогда, когда процессу нужны какие-то ресурсы. Как только условие, которого ждет процессор, выполняется, он переходит в состояние \textbf{“готов“}, то есть ожидает своей очереди на выполнение. 

Из каждого состояния процесс может перейти в состояние \textbf{“зомби“}. Этот процесс нужен для того, чтобы родительский процесс мог получить код возврата этого процесса. Это промежуточное состояние --- процесса уже нет, но он технически еще есть. Тем не менее, это самое что ни на есть валидное состояние. 

\textbf{В какой момент процесс окончательно исчезает из таблицы процессов?}

После того как его родительский процесс вызовет wait (2), waitpid (2), waitid(2) - позволяет завершить процесс, если он таки стал зомби. 

Традиционная проблема --- порождая процесс, необходимо в какой-то момент сказать ему wait (2), чтобы не плодить зомби. В противном случае у вас может закончится лимит на создание новых процессов. 

\textbf{Что происходит с зомби, если его родитель не вызвал wait (2), а погиб?}

Его родителем становится init. Для всех процессов, которые становятся его потомками он делает wait (2). По факту, если вы написали какой-то код, который выполняет некоторое количество процессов после чего завершился, то на самом деле зомби в системе не останутся, т.к. после того как ваш основной процесс завершился, то потомков этого процесса подхватит init и он сам вызовет для них wait (2).


      \section{Ввод-вывод}

         \subsection{Простой ввод-вывод}
         Жизненный цикл процесса начинается с того, что его кто-то порождает. Единственный процесс, который никто не порождает, это процесс init. Порождение процесса осуществляется с помощью fork (2) или vfork (2). 

Первая стадия в жизни процесса --- \textbf{“инициализация“}. В это время ядро производит подготовительные работы к дальнейшей работе процесса. На самом деле, это не совсем состояние, но логически для пользователя оно существует.

\begin{figure}[htbp]
  \centering
  \includegraphics[width=0.7\textwidth]{./processes-and-threads/processes/lifecycle/proc-lifecycle.png}
\end{figure}

Когда инициализация процесса завершается --- он оказывается в стадии \textbf{“готов“}. С этого момента процесс находится в ожидании момента, когда его выберет планировщик процессов и даст ему какой-то квант процессорного времени.

При получении процессорного времени процесс переходит в состояние \textbf{“выполняется“}. Выполняться процесс может в режиме ядра --- при осуществлении системных вызовов, прерываний и в режиме задачи --- выполнять инструкции процессора. По окончании квоты времени, процесс может снова вернуться в состояние \textbf{“готов“}.

Также из состояния “выполняется“ процесс может перейти еще в два состояния --- \textbf{“остановлен“} и \textbf{“ожидает“}. Если он остановлен, то он остановлен пользователем (например, получен сигнал SIGSTOP). Ожидание наступает тогда, когда процессу нужны какие-то ресурсы. Как только условие, которого ждет процессор, выполняется, он переходит в состояние \textbf{“готов“}, то есть ожидает своей очереди на выполнение. 

Из каждого состояния процесс может перейти в состояние \textbf{“зомби“}. Этот процесс нужен для того, чтобы родительский процесс мог получить код возврата этого процесса. Это промежуточное состояние --- процесса уже нет, но он технически еще есть. Тем не менее, это самое что ни на есть валидное состояние. 

\textbf{В какой момент процесс окончательно исчезает из таблицы процессов?}

После того как его родительский процесс вызовет wait (2), waitpid (2), waitid(2) - позволяет завершить процесс, если он таки стал зомби. 

Традиционная проблема --- порождая процесс, необходимо в какой-то момент сказать ему wait (2), чтобы не плодить зомби. В противном случае у вас может закончится лимит на создание новых процессов. 

\textbf{Что происходит с зомби, если его родитель не вызвал wait (2), а погиб?}

Его родителем становится init. Для всех процессов, которые становятся его потомками он делает wait (2). По факту, если вы написали какой-то код, который выполняет некоторое количество процессов после чего завершился, то на самом деле зомби в системе не останутся, т.к. после того как ваш основной процесс завершился, то потомков этого процесса подхватит init и он сам вызовет для них wait (2).


         \subsection{Ввод-вывод со смещением}
         Мы уже написали о том, что есть простой ввод-вывод write и read, однако ему есть альтернатива --- ввод-вывод со смещением. 

Ввод-вывод со смещением позволяет не просто вводить и выводить в том месте, на которое указывает текущая позиция внутри файла, а позволяет осуществить ввод в конкретную позицию или прочитать оттуда данные.

Для его осуществления есть два системных вызова pread (2) и pwrite (2). Оба вызова принимают в качестве аргуметнов файловый дескриптор, буфер для чтения/записи, размер этого буфера и смещение относительно начала файла.

\begin{CCode}{pread (2)}
	ssize_t pread(
		int fildes, 	/* descriptor of file */
		void *buf,		/* buffer */
		size_t nbyte, 	/* size of buffer  */
		off_t offset	/* offset of start of file */
	); \end{CCode}
Возвращает количество считанных байт или код ошибки.

\begin{CCode}{pwrite (2)}
	ssize_t pwrite(
		int fildes, 		/* descriptor of file */
		const void *buf,	/* buffer */
		size_t nbyte,		/* size of buffer  */
		off_t offset		/* offset of start of file */
	); \end{CCode}
Возвращает количество записанных байт или код ошибки.

\textbf{Когда может понадобиться ввод или вывод со смещением?}

Если у вас есть однопоточное приложение, внутри открытого файла вы можете управлять указателем с помощью lseek. Если же у вас несколько потоков и они работают с одним и тем же файлом (например, большая база данных), а указатель все равно один, может произойти такая неприятная ситуация.

Допустим, вы поставили указатель на то место, куда хотели бы записать 0.5Гб, и начали запись. В это время другой поток что-то прочитал, то есть подвинул указатель. Теперь запись по этому указателю может привести к фатальным последствиям --- испортятся важные данные.

Системные вызовы выполняются атомарно. То есть, при выполнении любого системного вызова блокируется выполнение всего процесса целиком и полностью --- никто другой в этот момент не может сделать системный вызов из этого процесса. Это значит, что и pread и pwrite выполняются атомарно т.е. никто не сможет воспользоваться lseek и сдвинуть наш указатель. Именно этим и пользуются программисты, когда используют ввод и вывод со смещением.


         \subsection{Векторный ввод-вывод}
         Что делать, если нам нужно вывести сразу несколько буферов или прочитать сразу в несколько мест из одного файла? Это решается вводом дополнительного буфера. Это плохо тем, что это требует лишнего копирования данных внутри памяти. Для решения этой проблемы был придуман векторный (или рассеянный) ввод-вывод.

Одна из особенностей реализации векторного ввода-вывода --- структура iovec. Она хранит в себе адрес начала сегмента (iov\_base), в который мы хотим что-то записать, и количество байт (iov\_len), которые мы готовы отвести под хранение данных.

\begin{CCode}{Структура iovec}
	typedef struct iovec { 
		void *iov_base;  	/* start address */ 
		size_t iov_len; 	/* segment length */ 
	} iovec_t; \end{CCode}

\textbf{Что нам это позволяет делать?}

Теперь, создав массив из векторов ioveс, мы можем последовательно брать вектор и читать/писать данные из блоков, которые в нем описаны.
Эту процедуру за нас полностью выполняют системные вызовы readv (2) и writev (2). В качестве агрументов они принимают файловый дескриптор, указатель на первую структуру в массиве iovec и количество структур, содержащихся в массиве. 

Системный вызов readv (2) отвечает за векторный вывод и возвращает количество считанных байт.

\begin{CCode}{readv (2)}
	#include <sys/types.h>
	#include <sys/uio.h>

	ssize_t readv(
		int fildes, 				/* file descriptor */ 
		const struct iovec *iov, 	/* pointer on first structure in array */ 
		int iovcnt 					/* size of array of structres */ 
	); \end{CCode}

Системный вызов write (2) отвечает за векторный вввод и возвращает количество записанных байт.

\begin{CCode}{writev}
	#include <sys/types.h>
	#include <sys/uio.h>

	ssize_t writev(
		int fildes, 				/* file descriptor */ 
		const struct iovec *iov, 	/* pointer on first structure in array */ 
		int iovcnt 					/* size of array of structres */ 
	); \end{CCode}

\begin{CCode}{Пример}
	ssize_t bytes_written;
	int fd;
	char *buf0 = "Inspiration unlocks the future.\n";
	char *buf1 = "Now go, and don't look back.\n";
	int iovcnt;
	const struct iovec iov[2];

	iov[0].iov_base = buf0;
	iov[0].iov_len = strlen(buf0);
	iov[1].iov_base = buf1;
	iov[1].iov_len = strlen(buf1);
	
	iovcnt = sizeof(iov) / sizeof(struct iovec);
		...
	bytes_written = writev(fd, iov, iovcnt); \end{CCode}

В этом примере в файл с дескриптором fd будут записаны две строки из buf0 и buf1.


      \subsection{Конец файла}
      Жизненный цикл процесса начинается с того, что его кто-то порождает. Единственный процесс, который никто не порождает, это процесс init. Порождение процесса осуществляется с помощью fork (2) или vfork (2). 

Первая стадия в жизни процесса --- \textbf{“инициализация“}. В это время ядро производит подготовительные работы к дальнейшей работе процесса. На самом деле, это не совсем состояние, но логически для пользователя оно существует.

\begin{figure}[htbp]
  \centering
  \includegraphics[width=0.7\textwidth]{./processes-and-threads/processes/lifecycle/proc-lifecycle.png}
\end{figure}

Когда инициализация процесса завершается --- он оказывается в стадии \textbf{“готов“}. С этого момента процесс находится в ожидании момента, когда его выберет планировщик процессов и даст ему какой-то квант процессорного времени.

При получении процессорного времени процесс переходит в состояние \textbf{“выполняется“}. Выполняться процесс может в режиме ядра --- при осуществлении системных вызовов, прерываний и в режиме задачи --- выполнять инструкции процессора. По окончании квоты времени, процесс может снова вернуться в состояние \textbf{“готов“}.

Также из состояния “выполняется“ процесс может перейти еще в два состояния --- \textbf{“остановлен“} и \textbf{“ожидает“}. Если он остановлен, то он остановлен пользователем (например, получен сигнал SIGSTOP). Ожидание наступает тогда, когда процессу нужны какие-то ресурсы. Как только условие, которого ждет процессор, выполняется, он переходит в состояние \textbf{“готов“}, то есть ожидает своей очереди на выполнение. 

Из каждого состояния процесс может перейти в состояние \textbf{“зомби“}. Этот процесс нужен для того, чтобы родительский процесс мог получить код возврата этого процесса. Это промежуточное состояние --- процесса уже нет, но он технически еще есть. Тем не менее, это самое что ни на есть валидное состояние. 

\textbf{В какой момент процесс окончательно исчезает из таблицы процессов?}

После того как его родительский процесс вызовет wait (2), waitpid (2), waitid(2) - позволяет завершить процесс, если он таки стал зомби. 

Традиционная проблема --- порождая процесс, необходимо в какой-то момент сказать ему wait (2), чтобы не плодить зомби. В противном случае у вас может закончится лимит на создание новых процессов. 

\textbf{Что происходит с зомби, если его родитель не вызвал wait (2), а погиб?}

Его родителем становится init. Для всех процессов, которые становятся его потомками он делает wait (2). По факту, если вы написали какой-то код, который выполняет некоторое количество процессов после чего завершился, то на самом деле зомби в системе не останутся, т.к. после того как ваш основной процесс завершился, то потомков этого процесса подхватит init и он сам вызовет для них wait (2).


   \chapter{Директории}
   С точки зрения пользователя файлы в ОС UNIX организованы в виде древовидного пространства имен. Дерево состоит из ветвей (каталогов) начиная от (/) и заканчивается листьями (файлами).

Каталог отличается от обычного файла тем, что у него четко заранее заданная известная структура, которую можно представить в виде таблицы, содержащей имена находящихся в нем файлов и указатели на метаданные, позволяющие ОС производить действия с этими файлами.
	

      \section{Структура dirent}
      Жизненный цикл процесса начинается с того, что его кто-то порождает. Единственный процесс, который никто не порождает, это процесс init. Порождение процесса осуществляется с помощью fork (2) или vfork (2). 

Первая стадия в жизни процесса --- \textbf{“инициализация“}. В это время ядро производит подготовительные работы к дальнейшей работе процесса. На самом деле, это не совсем состояние, но логически для пользователя оно существует.

\begin{figure}[htbp]
  \centering
  \includegraphics[width=0.7\textwidth]{./processes-and-threads/processes/lifecycle/proc-lifecycle.png}
\end{figure}

Когда инициализация процесса завершается --- он оказывается в стадии \textbf{“готов“}. С этого момента процесс находится в ожидании момента, когда его выберет планировщик процессов и даст ему какой-то квант процессорного времени.

При получении процессорного времени процесс переходит в состояние \textbf{“выполняется“}. Выполняться процесс может в режиме ядра --- при осуществлении системных вызовов, прерываний и в режиме задачи --- выполнять инструкции процессора. По окончании квоты времени, процесс может снова вернуться в состояние \textbf{“готов“}.

Также из состояния “выполняется“ процесс может перейти еще в два состояния --- \textbf{“остановлен“} и \textbf{“ожидает“}. Если он остановлен, то он остановлен пользователем (например, получен сигнал SIGSTOP). Ожидание наступает тогда, когда процессу нужны какие-то ресурсы. Как только условие, которого ждет процессор, выполняется, он переходит в состояние \textbf{“готов“}, то есть ожидает своей очереди на выполнение. 

Из каждого состояния процесс может перейти в состояние \textbf{“зомби“}. Этот процесс нужен для того, чтобы родительский процесс мог получить код возврата этого процесса. Это промежуточное состояние --- процесса уже нет, но он технически еще есть. Тем не менее, это самое что ни на есть валидное состояние. 

\textbf{В какой момент процесс окончательно исчезает из таблицы процессов?}

После того как его родительский процесс вызовет wait (2), waitpid (2), waitid(2) - позволяет завершить процесс, если он таки стал зомби. 

Традиционная проблема --- порождая процесс, необходимо в какой-то момент сказать ему wait (2), чтобы не плодить зомби. В противном случае у вас может закончится лимит на создание новых процессов. 

\textbf{Что происходит с зомби, если его родитель не вызвал wait (2), а погиб?}

Его родителем становится init. Для всех процессов, которые становятся его потомками он делает wait (2). По факту, если вы написали какой-то код, который выполняет некоторое количество процессов после чего завершился, то на самом деле зомби в системе не останутся, т.к. после того как ваш основной процесс завершился, то потомков этого процесса подхватит init и он сам вызовет для них wait (2).


      \section{Получение текущей рабочей директории и ее изменение}
      Жизненный цикл процесса начинается с того, что его кто-то порождает. Единственный процесс, который никто не порождает, это процесс init. Порождение процесса осуществляется с помощью fork (2) или vfork (2). 

Первая стадия в жизни процесса --- \textbf{“инициализация“}. В это время ядро производит подготовительные работы к дальнейшей работе процесса. На самом деле, это не совсем состояние, но логически для пользователя оно существует.

\begin{figure}[htbp]
  \centering
  \includegraphics[width=0.7\textwidth]{./processes-and-threads/processes/lifecycle/proc-lifecycle.png}
\end{figure}

Когда инициализация процесса завершается --- он оказывается в стадии \textbf{“готов“}. С этого момента процесс находится в ожидании момента, когда его выберет планировщик процессов и даст ему какой-то квант процессорного времени.

При получении процессорного времени процесс переходит в состояние \textbf{“выполняется“}. Выполняться процесс может в режиме ядра --- при осуществлении системных вызовов, прерываний и в режиме задачи --- выполнять инструкции процессора. По окончании квоты времени, процесс может снова вернуться в состояние \textbf{“готов“}.

Также из состояния “выполняется“ процесс может перейти еще в два состояния --- \textbf{“остановлен“} и \textbf{“ожидает“}. Если он остановлен, то он остановлен пользователем (например, получен сигнал SIGSTOP). Ожидание наступает тогда, когда процессу нужны какие-то ресурсы. Как только условие, которого ждет процессор, выполняется, он переходит в состояние \textbf{“готов“}, то есть ожидает своей очереди на выполнение. 

Из каждого состояния процесс может перейти в состояние \textbf{“зомби“}. Этот процесс нужен для того, чтобы родительский процесс мог получить код возврата этого процесса. Это промежуточное состояние --- процесса уже нет, но он технически еще есть. Тем не менее, это самое что ни на есть валидное состояние. 

\textbf{В какой момент процесс окончательно исчезает из таблицы процессов?}

После того как его родительский процесс вызовет wait (2), waitpid (2), waitid(2) - позволяет завершить процесс, если он таки стал зомби. 

Традиционная проблема --- порождая процесс, необходимо в какой-то момент сказать ему wait (2), чтобы не плодить зомби. В противном случае у вас может закончится лимит на создание новых процессов. 

\textbf{Что происходит с зомби, если его родитель не вызвал wait (2), а погиб?}

Его родителем становится init. Для всех процессов, которые становятся его потомками он делает wait (2). По факту, если вы написали какой-то код, который выполняет некоторое количество процессов после чего завершился, то на самом деле зомби в системе не останутся, т.к. после того как ваш основной процесс завершился, то потомков этого процесса подхватит init и он сам вызовет для них wait (2).
	

      \section{Создание и удаление директорий}
      Жизненный цикл процесса начинается с того, что его кто-то порождает. Единственный процесс, который никто не порождает, это процесс init. Порождение процесса осуществляется с помощью fork (2) или vfork (2). 

Первая стадия в жизни процесса --- \textbf{“инициализация“}. В это время ядро производит подготовительные работы к дальнейшей работе процесса. На самом деле, это не совсем состояние, но логически для пользователя оно существует.

\begin{figure}[htbp]
  \centering
  \includegraphics[width=0.7\textwidth]{./processes-and-threads/processes/lifecycle/proc-lifecycle.png}
\end{figure}

Когда инициализация процесса завершается --- он оказывается в стадии \textbf{“готов“}. С этого момента процесс находится в ожидании момента, когда его выберет планировщик процессов и даст ему какой-то квант процессорного времени.

При получении процессорного времени процесс переходит в состояние \textbf{“выполняется“}. Выполняться процесс может в режиме ядра --- при осуществлении системных вызовов, прерываний и в режиме задачи --- выполнять инструкции процессора. По окончании квоты времени, процесс может снова вернуться в состояние \textbf{“готов“}.

Также из состояния “выполняется“ процесс может перейти еще в два состояния --- \textbf{“остановлен“} и \textbf{“ожидает“}. Если он остановлен, то он остановлен пользователем (например, получен сигнал SIGSTOP). Ожидание наступает тогда, когда процессу нужны какие-то ресурсы. Как только условие, которого ждет процессор, выполняется, он переходит в состояние \textbf{“готов“}, то есть ожидает своей очереди на выполнение. 

Из каждого состояния процесс может перейти в состояние \textbf{“зомби“}. Этот процесс нужен для того, чтобы родительский процесс мог получить код возврата этого процесса. Это промежуточное состояние --- процесса уже нет, но он технически еще есть. Тем не менее, это самое что ни на есть валидное состояние. 

\textbf{В какой момент процесс окончательно исчезает из таблицы процессов?}

После того как его родительский процесс вызовет wait (2), waitpid (2), waitid(2) - позволяет завершить процесс, если он таки стал зомби. 

Традиционная проблема --- порождая процесс, необходимо в какой-то момент сказать ему wait (2), чтобы не плодить зомби. В противном случае у вас может закончится лимит на создание новых процессов. 

\textbf{Что происходит с зомби, если его родитель не вызвал wait (2), а погиб?}

Его родителем становится init. Для всех процессов, которые становятся его потомками он делает wait (2). По факту, если вы написали какой-то код, который выполняет некоторое количество процессов после чего завершился, то на самом деле зомби в системе не останутся, т.к. после того как ваш основной процесс завершился, то потомков этого процесса подхватит init и он сам вызовет для них wait (2).
	


   \chapter{Ссылки}

      \section{Жесткие ссылки}
      Жизненный цикл процесса начинается с того, что его кто-то порождает. Единственный процесс, который никто не порождает, это процесс init. Порождение процесса осуществляется с помощью fork (2) или vfork (2). 

Первая стадия в жизни процесса --- \textbf{“инициализация“}. В это время ядро производит подготовительные работы к дальнейшей работе процесса. На самом деле, это не совсем состояние, но логически для пользователя оно существует.

\begin{figure}[htbp]
  \centering
  \includegraphics[width=0.7\textwidth]{./processes-and-threads/processes/lifecycle/proc-lifecycle.png}
\end{figure}

Когда инициализация процесса завершается --- он оказывается в стадии \textbf{“готов“}. С этого момента процесс находится в ожидании момента, когда его выберет планировщик процессов и даст ему какой-то квант процессорного времени.

При получении процессорного времени процесс переходит в состояние \textbf{“выполняется“}. Выполняться процесс может в режиме ядра --- при осуществлении системных вызовов, прерываний и в режиме задачи --- выполнять инструкции процессора. По окончании квоты времени, процесс может снова вернуться в состояние \textbf{“готов“}.

Также из состояния “выполняется“ процесс может перейти еще в два состояния --- \textbf{“остановлен“} и \textbf{“ожидает“}. Если он остановлен, то он остановлен пользователем (например, получен сигнал SIGSTOP). Ожидание наступает тогда, когда процессу нужны какие-то ресурсы. Как только условие, которого ждет процессор, выполняется, он переходит в состояние \textbf{“готов“}, то есть ожидает своей очереди на выполнение. 

Из каждого состояния процесс может перейти в состояние \textbf{“зомби“}. Этот процесс нужен для того, чтобы родительский процесс мог получить код возврата этого процесса. Это промежуточное состояние --- процесса уже нет, но он технически еще есть. Тем не менее, это самое что ни на есть валидное состояние. 

\textbf{В какой момент процесс окончательно исчезает из таблицы процессов?}

После того как его родительский процесс вызовет wait (2), waitpid (2), waitid(2) - позволяет завершить процесс, если он таки стал зомби. 

Традиционная проблема --- порождая процесс, необходимо в какой-то момент сказать ему wait (2), чтобы не плодить зомби. В противном случае у вас может закончится лимит на создание новых процессов. 

\textbf{Что происходит с зомби, если его родитель не вызвал wait (2), а погиб?}

Его родителем становится init. Для всех процессов, которые становятся его потомками он делает wait (2). По факту, если вы написали какой-то код, который выполняет некоторое количество процессов после чего завершился, то на самом деле зомби в системе не останутся, т.к. после того как ваш основной процесс завершился, то потомков этого процесса подхватит init и он сам вызовет для них wait (2).


      \section{Символьные ссылки}
      Жизненный цикл процесса начинается с того, что его кто-то порождает. Единственный процесс, который никто не порождает, это процесс init. Порождение процесса осуществляется с помощью fork (2) или vfork (2). 

Первая стадия в жизни процесса --- \textbf{“инициализация“}. В это время ядро производит подготовительные работы к дальнейшей работе процесса. На самом деле, это не совсем состояние, но логически для пользователя оно существует.

\begin{figure}[htbp]
  \centering
  \includegraphics[width=0.7\textwidth]{./processes-and-threads/processes/lifecycle/proc-lifecycle.png}
\end{figure}

Когда инициализация процесса завершается --- он оказывается в стадии \textbf{“готов“}. С этого момента процесс находится в ожидании момента, когда его выберет планировщик процессов и даст ему какой-то квант процессорного времени.

При получении процессорного времени процесс переходит в состояние \textbf{“выполняется“}. Выполняться процесс может в режиме ядра --- при осуществлении системных вызовов, прерываний и в режиме задачи --- выполнять инструкции процессора. По окончании квоты времени, процесс может снова вернуться в состояние \textbf{“готов“}.

Также из состояния “выполняется“ процесс может перейти еще в два состояния --- \textbf{“остановлен“} и \textbf{“ожидает“}. Если он остановлен, то он остановлен пользователем (например, получен сигнал SIGSTOP). Ожидание наступает тогда, когда процессу нужны какие-то ресурсы. Как только условие, которого ждет процессор, выполняется, он переходит в состояние \textbf{“готов“}, то есть ожидает своей очереди на выполнение. 

Из каждого состояния процесс может перейти в состояние \textbf{“зомби“}. Этот процесс нужен для того, чтобы родительский процесс мог получить код возврата этого процесса. Это промежуточное состояние --- процесса уже нет, но он технически еще есть. Тем не менее, это самое что ни на есть валидное состояние. 

\textbf{В какой момент процесс окончательно исчезает из таблицы процессов?}

После того как его родительский процесс вызовет wait (2), waitpid (2), waitid(2) - позволяет завершить процесс, если он таки стал зомби. 

Традиционная проблема --- порождая процесс, необходимо в какой-то момент сказать ему wait (2), чтобы не плодить зомби. В противном случае у вас может закончится лимит на создание новых процессов. 

\textbf{Что происходит с зомби, если его родитель не вызвал wait (2), а погиб?}

Его родителем становится init. Для всех процессов, которые становятся его потомками он делает wait (2). По факту, если вы написали какой-то код, который выполняет некоторое количество процессов после чего завершился, то на самом деле зомби в системе не останутся, т.к. после того как ваш основной процесс завершился, то потомков этого процесса подхватит init и он сам вызовет для них wait (2).


   \chapter{Файлы устройств}
   Жизненный цикл процесса начинается с того, что его кто-то порождает. Единственный процесс, который никто не порождает, это процесс init. Порождение процесса осуществляется с помощью fork (2) или vfork (2). 

Первая стадия в жизни процесса --- \textbf{“инициализация“}. В это время ядро производит подготовительные работы к дальнейшей работе процесса. На самом деле, это не совсем состояние, но логически для пользователя оно существует.

\begin{figure}[htbp]
  \centering
  \includegraphics[width=0.7\textwidth]{./processes-and-threads/processes/lifecycle/proc-lifecycle.png}
\end{figure}

Когда инициализация процесса завершается --- он оказывается в стадии \textbf{“готов“}. С этого момента процесс находится в ожидании момента, когда его выберет планировщик процессов и даст ему какой-то квант процессорного времени.

При получении процессорного времени процесс переходит в состояние \textbf{“выполняется“}. Выполняться процесс может в режиме ядра --- при осуществлении системных вызовов, прерываний и в режиме задачи --- выполнять инструкции процессора. По окончании квоты времени, процесс может снова вернуться в состояние \textbf{“готов“}.

Также из состояния “выполняется“ процесс может перейти еще в два состояния --- \textbf{“остановлен“} и \textbf{“ожидает“}. Если он остановлен, то он остановлен пользователем (например, получен сигнал SIGSTOP). Ожидание наступает тогда, когда процессу нужны какие-то ресурсы. Как только условие, которого ждет процессор, выполняется, он переходит в состояние \textbf{“готов“}, то есть ожидает своей очереди на выполнение. 

Из каждого состояния процесс может перейти в состояние \textbf{“зомби“}. Этот процесс нужен для того, чтобы родительский процесс мог получить код возврата этого процесса. Это промежуточное состояние --- процесса уже нет, но он технически еще есть. Тем не менее, это самое что ни на есть валидное состояние. 

\textbf{В какой момент процесс окончательно исчезает из таблицы процессов?}

После того как его родительский процесс вызовет wait (2), waitpid (2), waitid(2) - позволяет завершить процесс, если он таки стал зомби. 

Традиционная проблема --- порождая процесс, необходимо в какой-то момент сказать ему wait (2), чтобы не плодить зомби. В противном случае у вас может закончится лимит на создание новых процессов. 

\textbf{Что происходит с зомби, если его родитель не вызвал wait (2), а погиб?}

Его родителем становится init. Для всех процессов, которые становятся его потомками он делает wait (2). По факту, если вы написали какой-то код, который выполняет некоторое количество процессов после чего завершился, то на самом деле зомби в системе не останутся, т.к. после того как ваш основной процесс завершился, то потомков этого процесса подхватит init и он сам вызовет для них wait (2).




   % ------------------------------
   % Часть: Пользователи и группы

\clearpage \part{Пользователи и группы}

   \chapter{Пользователи и группы}
   Как мы помним, UNIX является многопользовательской операционной системой. Пользователи, занимающиеся общими задачами, могут объединяться в группы. Каждый пользователь обязательно принадлежит к одной или нескольким группам. Все команды выполняются от имени определенного пользователя, принадлежащего в момент выполнения к определенной группе.

Для изменения действующего пользователя существует утилита su.


      \section{Атрибуты пользователя}
      Жизненный цикл процесса начинается с того, что его кто-то порождает. Единственный процесс, который никто не порождает, это процесс init. Порождение процесса осуществляется с помощью fork (2) или vfork (2). 

Первая стадия в жизни процесса --- \textbf{“инициализация“}. В это время ядро производит подготовительные работы к дальнейшей работе процесса. На самом деле, это не совсем состояние, но логически для пользователя оно существует.

\begin{figure}[htbp]
  \centering
  \includegraphics[width=0.7\textwidth]{./processes-and-threads/processes/lifecycle/proc-lifecycle.png}
\end{figure}

Когда инициализация процесса завершается --- он оказывается в стадии \textbf{“готов“}. С этого момента процесс находится в ожидании момента, когда его выберет планировщик процессов и даст ему какой-то квант процессорного времени.

При получении процессорного времени процесс переходит в состояние \textbf{“выполняется“}. Выполняться процесс может в режиме ядра --- при осуществлении системных вызовов, прерываний и в режиме задачи --- выполнять инструкции процессора. По окончании квоты времени, процесс может снова вернуться в состояние \textbf{“готов“}.

Также из состояния “выполняется“ процесс может перейти еще в два состояния --- \textbf{“остановлен“} и \textbf{“ожидает“}. Если он остановлен, то он остановлен пользователем (например, получен сигнал SIGSTOP). Ожидание наступает тогда, когда процессу нужны какие-то ресурсы. Как только условие, которого ждет процессор, выполняется, он переходит в состояние \textbf{“готов“}, то есть ожидает своей очереди на выполнение. 

Из каждого состояния процесс может перейти в состояние \textbf{“зомби“}. Этот процесс нужен для того, чтобы родительский процесс мог получить код возврата этого процесса. Это промежуточное состояние --- процесса уже нет, но он технически еще есть. Тем не менее, это самое что ни на есть валидное состояние. 

\textbf{В какой момент процесс окончательно исчезает из таблицы процессов?}

После того как его родительский процесс вызовет wait (2), waitpid (2), waitid(2) - позволяет завершить процесс, если он таки стал зомби. 

Традиционная проблема --- порождая процесс, необходимо в какой-то момент сказать ему wait (2), чтобы не плодить зомби. В противном случае у вас может закончится лимит на создание новых процессов. 

\textbf{Что происходит с зомби, если его родитель не вызвал wait (2), а погиб?}

Его родителем становится init. Для всех процессов, которые становятся его потомками он делает wait (2). По факту, если вы написали какой-то код, который выполняет некоторое количество процессов после чего завершился, то на самом деле зомби в системе не останутся, т.к. после того как ваш основной процесс завершился, то потомков этого процесса подхватит init и он сам вызовет для них wait (2).


         \subsection{Атрибуты группы}
         Жизненный цикл процесса начинается с того, что его кто-то порождает. Единственный процесс, который никто не порождает, это процесс init. Порождение процесса осуществляется с помощью fork (2) или vfork (2). 

Первая стадия в жизни процесса --- \textbf{“инициализация“}. В это время ядро производит подготовительные работы к дальнейшей работе процесса. На самом деле, это не совсем состояние, но логически для пользователя оно существует.

\begin{figure}[htbp]
  \centering
  \includegraphics[width=0.7\textwidth]{./processes-and-threads/processes/lifecycle/proc-lifecycle.png}
\end{figure}

Когда инициализация процесса завершается --- он оказывается в стадии \textbf{“готов“}. С этого момента процесс находится в ожидании момента, когда его выберет планировщик процессов и даст ему какой-то квант процессорного времени.

При получении процессорного времени процесс переходит в состояние \textbf{“выполняется“}. Выполняться процесс может в режиме ядра --- при осуществлении системных вызовов, прерываний и в режиме задачи --- выполнять инструкции процессора. По окончании квоты времени, процесс может снова вернуться в состояние \textbf{“готов“}.

Также из состояния “выполняется“ процесс может перейти еще в два состояния --- \textbf{“остановлен“} и \textbf{“ожидает“}. Если он остановлен, то он остановлен пользователем (например, получен сигнал SIGSTOP). Ожидание наступает тогда, когда процессу нужны какие-то ресурсы. Как только условие, которого ждет процессор, выполняется, он переходит в состояние \textbf{“готов“}, то есть ожидает своей очереди на выполнение. 

Из каждого состояния процесс может перейти в состояние \textbf{“зомби“}. Этот процесс нужен для того, чтобы родительский процесс мог получить код возврата этого процесса. Это промежуточное состояние --- процесса уже нет, но он технически еще есть. Тем не менее, это самое что ни на есть валидное состояние. 

\textbf{В какой момент процесс окончательно исчезает из таблицы процессов?}

После того как его родительский процесс вызовет wait (2), waitpid (2), waitid(2) - позволяет завершить процесс, если он таки стал зомби. 

Традиционная проблема --- порождая процесс, необходимо в какой-то момент сказать ему wait (2), чтобы не плодить зомби. В противном случае у вас может закончится лимит на создание новых процессов. 

\textbf{Что происходит с зомби, если его родитель не вызвал wait (2), а погиб?}

Его родителем становится init. Для всех процессов, которые становятся его потомками он делает wait (2). По факту, если вы написали какой-то код, который выполняет некоторое количество процессов после чего завершился, то на самом деле зомби в системе не останутся, т.к. после того как ваш основной процесс завершился, то потомков этого процесса подхватит init и он сам вызовет для них wait (2).


         \subsection{Утилита getent}
         Жизненный цикл процесса начинается с того, что его кто-то порождает. Единственный процесс, который никто не порождает, это процесс init. Порождение процесса осуществляется с помощью fork (2) или vfork (2). 

Первая стадия в жизни процесса --- \textbf{“инициализация“}. В это время ядро производит подготовительные работы к дальнейшей работе процесса. На самом деле, это не совсем состояние, но логически для пользователя оно существует.

\begin{figure}[htbp]
  \centering
  \includegraphics[width=0.7\textwidth]{./processes-and-threads/processes/lifecycle/proc-lifecycle.png}
\end{figure}

Когда инициализация процесса завершается --- он оказывается в стадии \textbf{“готов“}. С этого момента процесс находится в ожидании момента, когда его выберет планировщик процессов и даст ему какой-то квант процессорного времени.

При получении процессорного времени процесс переходит в состояние \textbf{“выполняется“}. Выполняться процесс может в режиме ядра --- при осуществлении системных вызовов, прерываний и в режиме задачи --- выполнять инструкции процессора. По окончании квоты времени, процесс может снова вернуться в состояние \textbf{“готов“}.

Также из состояния “выполняется“ процесс может перейти еще в два состояния --- \textbf{“остановлен“} и \textbf{“ожидает“}. Если он остановлен, то он остановлен пользователем (например, получен сигнал SIGSTOP). Ожидание наступает тогда, когда процессу нужны какие-то ресурсы. Как только условие, которого ждет процессор, выполняется, он переходит в состояние \textbf{“готов“}, то есть ожидает своей очереди на выполнение. 

Из каждого состояния процесс может перейти в состояние \textbf{“зомби“}. Этот процесс нужен для того, чтобы родительский процесс мог получить код возврата этого процесса. Это промежуточное состояние --- процесса уже нет, но он технически еще есть. Тем не менее, это самое что ни на есть валидное состояние. 

\textbf{В какой момент процесс окончательно исчезает из таблицы процессов?}

После того как его родительский процесс вызовет wait (2), waitpid (2), waitid(2) - позволяет завершить процесс, если он таки стал зомби. 

Традиционная проблема --- порождая процесс, необходимо в какой-то момент сказать ему wait (2), чтобы не плодить зомби. В противном случае у вас может закончится лимит на создание новых процессов. 

\textbf{Что происходит с зомби, если его родитель не вызвал wait (2), а погиб?}

Его родителем становится init. Для всех процессов, которые становятся его потомками он делает wait (2). По факту, если вы написали какой-то код, который выполняет некоторое количество процессов после чего завершился, то на самом деле зомби в системе не останутся, т.к. после того как ваш основной процесс завершился, то потомков этого процесса подхватит init и он сам вызовет для них wait (2).




         % ------------------------------
         % Часть: Права и владельцы

\clearpage \part{Владельцы файлов, права и режимы доступа}

   \chapter{Права и режимы доступа}
   Жизненный цикл процесса начинается с того, что его кто-то порождает. Единственный процесс, который никто не порождает, это процесс init. Порождение процесса осуществляется с помощью fork (2) или vfork (2). 

Первая стадия в жизни процесса --- \textbf{“инициализация“}. В это время ядро производит подготовительные работы к дальнейшей работе процесса. На самом деле, это не совсем состояние, но логически для пользователя оно существует.

\begin{figure}[htbp]
  \centering
  \includegraphics[width=0.7\textwidth]{./processes-and-threads/processes/lifecycle/proc-lifecycle.png}
\end{figure}

Когда инициализация процесса завершается --- он оказывается в стадии \textbf{“готов“}. С этого момента процесс находится в ожидании момента, когда его выберет планировщик процессов и даст ему какой-то квант процессорного времени.

При получении процессорного времени процесс переходит в состояние \textbf{“выполняется“}. Выполняться процесс может в режиме ядра --- при осуществлении системных вызовов, прерываний и в режиме задачи --- выполнять инструкции процессора. По окончании квоты времени, процесс может снова вернуться в состояние \textbf{“готов“}.

Также из состояния “выполняется“ процесс может перейти еще в два состояния --- \textbf{“остановлен“} и \textbf{“ожидает“}. Если он остановлен, то он остановлен пользователем (например, получен сигнал SIGSTOP). Ожидание наступает тогда, когда процессу нужны какие-то ресурсы. Как только условие, которого ждет процессор, выполняется, он переходит в состояние \textbf{“готов“}, то есть ожидает своей очереди на выполнение. 

Из каждого состояния процесс может перейти в состояние \textbf{“зомби“}. Этот процесс нужен для того, чтобы родительский процесс мог получить код возврата этого процесса. Это промежуточное состояние --- процесса уже нет, но он технически еще есть. Тем не менее, это самое что ни на есть валидное состояние. 

\textbf{В какой момент процесс окончательно исчезает из таблицы процессов?}

После того как его родительский процесс вызовет wait (2), waitpid (2), waitid(2) - позволяет завершить процесс, если он таки стал зомби. 

Традиционная проблема --- порождая процесс, необходимо в какой-то момент сказать ему wait (2), чтобы не плодить зомби. В противном случае у вас может закончится лимит на создание новых процессов. 

\textbf{Что происходит с зомби, если его родитель не вызвал wait (2), а погиб?}

Его родителем становится init. Для всех процессов, которые становятся его потомками он делает wait (2). По факту, если вы написали какой-то код, который выполняет некоторое количество процессов после чего завершился, то на самом деле зомби в системе не останутся, т.к. после того как ваш основной процесс завершился, то потомков этого процесса подхватит init и он сам вызовет для них wait (2).


      \section{Права доступа}

         \subsection{Проверка прав доступа}
         Жизненный цикл процесса начинается с того, что его кто-то порождает. Единственный процесс, который никто не порождает, это процесс init. Порождение процесса осуществляется с помощью fork (2) или vfork (2). 

Первая стадия в жизни процесса --- \textbf{“инициализация“}. В это время ядро производит подготовительные работы к дальнейшей работе процесса. На самом деле, это не совсем состояние, но логически для пользователя оно существует.

\begin{figure}[htbp]
  \centering
  \includegraphics[width=0.7\textwidth]{./processes-and-threads/processes/lifecycle/proc-lifecycle.png}
\end{figure}

Когда инициализация процесса завершается --- он оказывается в стадии \textbf{“готов“}. С этого момента процесс находится в ожидании момента, когда его выберет планировщик процессов и даст ему какой-то квант процессорного времени.

При получении процессорного времени процесс переходит в состояние \textbf{“выполняется“}. Выполняться процесс может в режиме ядра --- при осуществлении системных вызовов, прерываний и в режиме задачи --- выполнять инструкции процессора. По окончании квоты времени, процесс может снова вернуться в состояние \textbf{“готов“}.

Также из состояния “выполняется“ процесс может перейти еще в два состояния --- \textbf{“остановлен“} и \textbf{“ожидает“}. Если он остановлен, то он остановлен пользователем (например, получен сигнал SIGSTOP). Ожидание наступает тогда, когда процессу нужны какие-то ресурсы. Как только условие, которого ждет процессор, выполняется, он переходит в состояние \textbf{“готов“}, то есть ожидает своей очереди на выполнение. 

Из каждого состояния процесс может перейти в состояние \textbf{“зомби“}. Этот процесс нужен для того, чтобы родительский процесс мог получить код возврата этого процесса. Это промежуточное состояние --- процесса уже нет, но он технически еще есть. Тем не менее, это самое что ни на есть валидное состояние. 

\textbf{В какой момент процесс окончательно исчезает из таблицы процессов?}

После того как его родительский процесс вызовет wait (2), waitpid (2), waitid(2) - позволяет завершить процесс, если он таки стал зомби. 

Традиционная проблема --- порождая процесс, необходимо в какой-то момент сказать ему wait (2), чтобы не плодить зомби. В противном случае у вас может закончится лимит на создание новых процессов. 

\textbf{Что происходит с зомби, если его родитель не вызвал wait (2), а погиб?}

Его родителем становится init. Для всех процессов, которые становятся его потомками он делает wait (2). По факту, если вы написали какой-то код, который выполняет некоторое количество процессов после чего завершился, то на самом деле зомби в системе не останутся, т.к. после того как ваш основной процесс завершился, то потомков этого процесса подхватит init и он сам вызовет для них wait (2).


         \subsection{Изменение прав доступа}
         Жизненный цикл процесса начинается с того, что его кто-то порождает. Единственный процесс, который никто не порождает, это процесс init. Порождение процесса осуществляется с помощью fork (2) или vfork (2). 

Первая стадия в жизни процесса --- \textbf{“инициализация“}. В это время ядро производит подготовительные работы к дальнейшей работе процесса. На самом деле, это не совсем состояние, но логически для пользователя оно существует.

\begin{figure}[htbp]
  \centering
  \includegraphics[width=0.7\textwidth]{./processes-and-threads/processes/lifecycle/proc-lifecycle.png}
\end{figure}

Когда инициализация процесса завершается --- он оказывается в стадии \textbf{“готов“}. С этого момента процесс находится в ожидании момента, когда его выберет планировщик процессов и даст ему какой-то квант процессорного времени.

При получении процессорного времени процесс переходит в состояние \textbf{“выполняется“}. Выполняться процесс может в режиме ядра --- при осуществлении системных вызовов, прерываний и в режиме задачи --- выполнять инструкции процессора. По окончании квоты времени, процесс может снова вернуться в состояние \textbf{“готов“}.

Также из состояния “выполняется“ процесс может перейти еще в два состояния --- \textbf{“остановлен“} и \textbf{“ожидает“}. Если он остановлен, то он остановлен пользователем (например, получен сигнал SIGSTOP). Ожидание наступает тогда, когда процессу нужны какие-то ресурсы. Как только условие, которого ждет процессор, выполняется, он переходит в состояние \textbf{“готов“}, то есть ожидает своей очереди на выполнение. 

Из каждого состояния процесс может перейти в состояние \textbf{“зомби“}. Этот процесс нужен для того, чтобы родительский процесс мог получить код возврата этого процесса. Это промежуточное состояние --- процесса уже нет, но он технически еще есть. Тем не менее, это самое что ни на есть валидное состояние. 

\textbf{В какой момент процесс окончательно исчезает из таблицы процессов?}

После того как его родительский процесс вызовет wait (2), waitpid (2), waitid(2) - позволяет завершить процесс, если он таки стал зомби. 

Традиционная проблема --- порождая процесс, необходимо в какой-то момент сказать ему wait (2), чтобы не плодить зомби. В противном случае у вас может закончится лимит на создание новых процессов. 

\textbf{Что происходит с зомби, если его родитель не вызвал wait (2), а погиб?}

Его родителем становится init. Для всех процессов, которые становятся его потомками он делает wait (2). По факту, если вы написали какой-то код, который выполняет некоторое количество процессов после чего завершился, то на самом деле зомби в системе не останутся, т.к. после того как ваш основной процесс завершился, то потомков этого процесса подхватит init и он сам вызовет для них wait (2).


         \subsection{Маска создания файла}
         Жизненный цикл процесса начинается с того, что его кто-то порождает. Единственный процесс, который никто не порождает, это процесс init. Порождение процесса осуществляется с помощью fork (2) или vfork (2). 

Первая стадия в жизни процесса --- \textbf{“инициализация“}. В это время ядро производит подготовительные работы к дальнейшей работе процесса. На самом деле, это не совсем состояние, но логически для пользователя оно существует.

\begin{figure}[htbp]
  \centering
  \includegraphics[width=0.7\textwidth]{./processes-and-threads/processes/lifecycle/proc-lifecycle.png}
\end{figure}

Когда инициализация процесса завершается --- он оказывается в стадии \textbf{“готов“}. С этого момента процесс находится в ожидании момента, когда его выберет планировщик процессов и даст ему какой-то квант процессорного времени.

При получении процессорного времени процесс переходит в состояние \textbf{“выполняется“}. Выполняться процесс может в режиме ядра --- при осуществлении системных вызовов, прерываний и в режиме задачи --- выполнять инструкции процессора. По окончании квоты времени, процесс может снова вернуться в состояние \textbf{“готов“}.

Также из состояния “выполняется“ процесс может перейти еще в два состояния --- \textbf{“остановлен“} и \textbf{“ожидает“}. Если он остановлен, то он остановлен пользователем (например, получен сигнал SIGSTOP). Ожидание наступает тогда, когда процессу нужны какие-то ресурсы. Как только условие, которого ждет процессор, выполняется, он переходит в состояние \textbf{“готов“}, то есть ожидает своей очереди на выполнение. 

Из каждого состояния процесс может перейти в состояние \textbf{“зомби“}. Этот процесс нужен для того, чтобы родительский процесс мог получить код возврата этого процесса. Это промежуточное состояние --- процесса уже нет, но он технически еще есть. Тем не менее, это самое что ни на есть валидное состояние. 

\textbf{В какой момент процесс окончательно исчезает из таблицы процессов?}

После того как его родительский процесс вызовет wait (2), waitpid (2), waitid(2) - позволяет завершить процесс, если он таки стал зомби. 

Традиционная проблема --- порождая процесс, необходимо в какой-то момент сказать ему wait (2), чтобы не плодить зомби. В противном случае у вас может закончится лимит на создание новых процессов. 

\textbf{Что происходит с зомби, если его родитель не вызвал wait (2), а погиб?}

Его родителем становится init. Для всех процессов, которые становятся его потомками он делает wait (2). По факту, если вы написали какой-то код, который выполняет некоторое количество процессов после чего завершился, то на самом деле зомби в системе не останутся, т.к. после того как ваш основной процесс завершился, то потомков этого процесса подхватит init и он сам вызовет для них wait (2).



   \chapter{Владельцы файлов}

      \section{Изменение владельца файла}
      С помощью системных вызовов chown (2) можно изменить владельца файла. Оба системных вызова принимают в качестве второго и третьего аргументов идентификатор владельца и идентификатор группы. Единственная разница между ними --- chown (2) принимает первым аргументом путь к файлу, а fchown (2) --- файловый дескриптор.

\begin{CCode}{chown(2)}
	int chown(
		const char *path, 
		uid_t owner,
		gid_t group
	); \end{CCode}

\begin{CCode}{fchown(2)}
	int fchown(
		int fildes, 
		uid_t owner,
		gid_t group
); \end{CCode}

Оба системных вызова возвращают 0 в случае успеха, в ином случае --- код ошибки.



      % ------------------------------
      % Часть: Память

\clearpage \part{Память}

   \chapter{Как устроена память}
   Жизненный цикл процесса начинается с того, что его кто-то порождает. Единственный процесс, который никто не порождает, это процесс init. Порождение процесса осуществляется с помощью fork (2) или vfork (2). 

Первая стадия в жизни процесса --- \textbf{“инициализация“}. В это время ядро производит подготовительные работы к дальнейшей работе процесса. На самом деле, это не совсем состояние, но логически для пользователя оно существует.

\begin{figure}[htbp]
  \centering
  \includegraphics[width=0.7\textwidth]{./processes-and-threads/processes/lifecycle/proc-lifecycle.png}
\end{figure}

Когда инициализация процесса завершается --- он оказывается в стадии \textbf{“готов“}. С этого момента процесс находится в ожидании момента, когда его выберет планировщик процессов и даст ему какой-то квант процессорного времени.

При получении процессорного времени процесс переходит в состояние \textbf{“выполняется“}. Выполняться процесс может в режиме ядра --- при осуществлении системных вызовов, прерываний и в режиме задачи --- выполнять инструкции процессора. По окончании квоты времени, процесс может снова вернуться в состояние \textbf{“готов“}.

Также из состояния “выполняется“ процесс может перейти еще в два состояния --- \textbf{“остановлен“} и \textbf{“ожидает“}. Если он остановлен, то он остановлен пользователем (например, получен сигнал SIGSTOP). Ожидание наступает тогда, когда процессу нужны какие-то ресурсы. Как только условие, которого ждет процессор, выполняется, он переходит в состояние \textbf{“готов“}, то есть ожидает своей очереди на выполнение. 

Из каждого состояния процесс может перейти в состояние \textbf{“зомби“}. Этот процесс нужен для того, чтобы родительский процесс мог получить код возврата этого процесса. Это промежуточное состояние --- процесса уже нет, но он технически еще есть. Тем не менее, это самое что ни на есть валидное состояние. 

\textbf{В какой момент процесс окончательно исчезает из таблицы процессов?}

После того как его родительский процесс вызовет wait (2), waitpid (2), waitid(2) - позволяет завершить процесс, если он таки стал зомби. 

Традиционная проблема --- порождая процесс, необходимо в какой-то момент сказать ему wait (2), чтобы не плодить зомби. В противном случае у вас может закончится лимит на создание новых процессов. 

\textbf{Что происходит с зомби, если его родитель не вызвал wait (2), а погиб?}

Его родителем становится init. Для всех процессов, которые становятся его потомками он делает wait (2). По факту, если вы написали какой-то код, который выполняет некоторое количество процессов после чего завершился, то на самом деле зомби в системе не останутся, т.к. после того как ваш основной процесс завершился, то потомков этого процесса подхватит init и он сам вызовет для них wait (2).


      \section{Аллокация памяти}
      Жизненный цикл процесса начинается с того, что его кто-то порождает. Единственный процесс, который никто не порождает, это процесс init. Порождение процесса осуществляется с помощью fork (2) или vfork (2). 

Первая стадия в жизни процесса --- \textbf{“инициализация“}. В это время ядро производит подготовительные работы к дальнейшей работе процесса. На самом деле, это не совсем состояние, но логически для пользователя оно существует.

\begin{figure}[htbp]
  \centering
  \includegraphics[width=0.7\textwidth]{./processes-and-threads/processes/lifecycle/proc-lifecycle.png}
\end{figure}

Когда инициализация процесса завершается --- он оказывается в стадии \textbf{“готов“}. С этого момента процесс находится в ожидании момента, когда его выберет планировщик процессов и даст ему какой-то квант процессорного времени.

При получении процессорного времени процесс переходит в состояние \textbf{“выполняется“}. Выполняться процесс может в режиме ядра --- при осуществлении системных вызовов, прерываний и в режиме задачи --- выполнять инструкции процессора. По окончании квоты времени, процесс может снова вернуться в состояние \textbf{“готов“}.

Также из состояния “выполняется“ процесс может перейти еще в два состояния --- \textbf{“остановлен“} и \textbf{“ожидает“}. Если он остановлен, то он остановлен пользователем (например, получен сигнал SIGSTOP). Ожидание наступает тогда, когда процессу нужны какие-то ресурсы. Как только условие, которого ждет процессор, выполняется, он переходит в состояние \textbf{“готов“}, то есть ожидает своей очереди на выполнение. 

Из каждого состояния процесс может перейти в состояние \textbf{“зомби“}. Этот процесс нужен для того, чтобы родительский процесс мог получить код возврата этого процесса. Это промежуточное состояние --- процесса уже нет, но он технически еще есть. Тем не менее, это самое что ни на есть валидное состояние. 

\textbf{В какой момент процесс окончательно исчезает из таблицы процессов?}

После того как его родительский процесс вызовет wait (2), waitpid (2), waitid(2) - позволяет завершить процесс, если он таки стал зомби. 

Традиционная проблема --- порождая процесс, необходимо в какой-то момент сказать ему wait (2), чтобы не плодить зомби. В противном случае у вас может закончится лимит на создание новых процессов. 

\textbf{Что происходит с зомби, если его родитель не вызвал wait (2), а погиб?}

Его родителем становится init. Для всех процессов, которые становятся его потомками он делает wait (2). По факту, если вы написали какой-то код, который выполняет некоторое количество процессов после чего завершился, то на самом деле зомби в системе не останутся, т.к. после того как ваш основной процесс завершился, то потомков этого процесса подхватит init и он сам вызовет для них wait (2).


      % ------------------------------
      % Часть: Процессы и потоки

\clearpage \part{Процессы и потоки}

   \chapter{Процессы}
   \begin{defi}{Процесс}
	совокупность программы и метаинформации, описывающей её выполнение
\end{defi}

Когда мы говорим процесс, речь идет именно о выполняющейся программе. Когда программа не выполняется её нельзя назвать процессом.

Каждый процесс представлен в системе двумя основными структурами данных — proc и u\_block, описанными, соответственно, в файлах <sys/proc.h> и <sys/user.h>. Содержимое и формат этих структур различны для разных версий UNIX.


      \section{Атрибуты процесса}
      Жизненный цикл процесса начинается с того, что его кто-то порождает. Единственный процесс, который никто не порождает, это процесс init. Порождение процесса осуществляется с помощью fork (2) или vfork (2). 

Первая стадия в жизни процесса --- \textbf{“инициализация“}. В это время ядро производит подготовительные работы к дальнейшей работе процесса. На самом деле, это не совсем состояние, но логически для пользователя оно существует.

\begin{figure}[htbp]
  \centering
  \includegraphics[width=0.7\textwidth]{./processes-and-threads/processes/lifecycle/proc-lifecycle.png}
\end{figure}

Когда инициализация процесса завершается --- он оказывается в стадии \textbf{“готов“}. С этого момента процесс находится в ожидании момента, когда его выберет планировщик процессов и даст ему какой-то квант процессорного времени.

При получении процессорного времени процесс переходит в состояние \textbf{“выполняется“}. Выполняться процесс может в режиме ядра --- при осуществлении системных вызовов, прерываний и в режиме задачи --- выполнять инструкции процессора. По окончании квоты времени, процесс может снова вернуться в состояние \textbf{“готов“}.

Также из состояния “выполняется“ процесс может перейти еще в два состояния --- \textbf{“остановлен“} и \textbf{“ожидает“}. Если он остановлен, то он остановлен пользователем (например, получен сигнал SIGSTOP). Ожидание наступает тогда, когда процессу нужны какие-то ресурсы. Как только условие, которого ждет процессор, выполняется, он переходит в состояние \textbf{“готов“}, то есть ожидает своей очереди на выполнение. 

Из каждого состояния процесс может перейти в состояние \textbf{“зомби“}. Этот процесс нужен для того, чтобы родительский процесс мог получить код возврата этого процесса. Это промежуточное состояние --- процесса уже нет, но он технически еще есть. Тем не менее, это самое что ни на есть валидное состояние. 

\textbf{В какой момент процесс окончательно исчезает из таблицы процессов?}

После того как его родительский процесс вызовет wait (2), waitpid (2), waitid(2) - позволяет завершить процесс, если он таки стал зомби. 

Традиционная проблема --- порождая процесс, необходимо в какой-то момент сказать ему wait (2), чтобы не плодить зомби. В противном случае у вас может закончится лимит на создание новых процессов. 

\textbf{Что происходит с зомби, если его родитель не вызвал wait (2), а погиб?}

Его родителем становится init. Для всех процессов, которые становятся его потомками он делает wait (2). По факту, если вы написали какой-то код, который выполняет некоторое количество процессов после чего завершился, то на самом деле зомби в системе не останутся, т.к. после того как ваш основной процесс завершился, то потомков этого процесса подхватит init и он сам вызовет для них wait (2).


      \section{Жизненный цикл процесса}
      Жизненный цикл процесса начинается с того, что его кто-то порождает. Единственный процесс, который никто не порождает, это процесс init. Порождение процесса осуществляется с помощью fork (2) или vfork (2). 

Первая стадия в жизни процесса --- \textbf{“инициализация“}. В это время ядро производит подготовительные работы к дальнейшей работе процесса. На самом деле, это не совсем состояние, но логически для пользователя оно существует.

\begin{figure}[htbp]
  \centering
  \includegraphics[width=0.7\textwidth]{./processes-and-threads/processes/lifecycle/proc-lifecycle.png}
\end{figure}

Когда инициализация процесса завершается --- он оказывается в стадии \textbf{“готов“}. С этого момента процесс находится в ожидании момента, когда его выберет планировщик процессов и даст ему какой-то квант процессорного времени.

При получении процессорного времени процесс переходит в состояние \textbf{“выполняется“}. Выполняться процесс может в режиме ядра --- при осуществлении системных вызовов, прерываний и в режиме задачи --- выполнять инструкции процессора. По окончании квоты времени, процесс может снова вернуться в состояние \textbf{“готов“}.

Также из состояния “выполняется“ процесс может перейти еще в два состояния --- \textbf{“остановлен“} и \textbf{“ожидает“}. Если он остановлен, то он остановлен пользователем (например, получен сигнал SIGSTOP). Ожидание наступает тогда, когда процессу нужны какие-то ресурсы. Как только условие, которого ждет процессор, выполняется, он переходит в состояние \textbf{“готов“}, то есть ожидает своей очереди на выполнение. 

Из каждого состояния процесс может перейти в состояние \textbf{“зомби“}. Этот процесс нужен для того, чтобы родительский процесс мог получить код возврата этого процесса. Это промежуточное состояние --- процесса уже нет, но он технически еще есть. Тем не менее, это самое что ни на есть валидное состояние. 

\textbf{В какой момент процесс окончательно исчезает из таблицы процессов?}

После того как его родительский процесс вызовет wait (2), waitpid (2), waitid(2) - позволяет завершить процесс, если он таки стал зомби. 

Традиционная проблема --- порождая процесс, необходимо в какой-то момент сказать ему wait (2), чтобы не плодить зомби. В противном случае у вас может закончится лимит на создание новых процессов. 

\textbf{Что происходит с зомби, если его родитель не вызвал wait (2), а погиб?}

Его родителем становится init. Для всех процессов, которые становятся его потомками он делает wait (2). По факту, если вы написали какой-то код, который выполняет некоторое количество процессов после чего завершился, то на самом деле зомби в системе не останутся, т.к. после того как ваш основной процесс завершился, то потомков этого процесса подхватит init и он сам вызовет для них wait (2).


      \section{Создание процессов}

         \subsection{Семейство системных вызовов fork}
         Жизненный цикл процесса начинается с того, что его кто-то порождает. Единственный процесс, который никто не порождает, это процесс init. Порождение процесса осуществляется с помощью fork (2) или vfork (2). 

Первая стадия в жизни процесса --- \textbf{“инициализация“}. В это время ядро производит подготовительные работы к дальнейшей работе процесса. На самом деле, это не совсем состояние, но логически для пользователя оно существует.

\begin{figure}[htbp]
  \centering
  \includegraphics[width=0.7\textwidth]{./processes-and-threads/processes/lifecycle/proc-lifecycle.png}
\end{figure}

Когда инициализация процесса завершается --- он оказывается в стадии \textbf{“готов“}. С этого момента процесс находится в ожидании момента, когда его выберет планировщик процессов и даст ему какой-то квант процессорного времени.

При получении процессорного времени процесс переходит в состояние \textbf{“выполняется“}. Выполняться процесс может в режиме ядра --- при осуществлении системных вызовов, прерываний и в режиме задачи --- выполнять инструкции процессора. По окончании квоты времени, процесс может снова вернуться в состояние \textbf{“готов“}.

Также из состояния “выполняется“ процесс может перейти еще в два состояния --- \textbf{“остановлен“} и \textbf{“ожидает“}. Если он остановлен, то он остановлен пользователем (например, получен сигнал SIGSTOP). Ожидание наступает тогда, когда процессу нужны какие-то ресурсы. Как только условие, которого ждет процессор, выполняется, он переходит в состояние \textbf{“готов“}, то есть ожидает своей очереди на выполнение. 

Из каждого состояния процесс может перейти в состояние \textbf{“зомби“}. Этот процесс нужен для того, чтобы родительский процесс мог получить код возврата этого процесса. Это промежуточное состояние --- процесса уже нет, но он технически еще есть. Тем не менее, это самое что ни на есть валидное состояние. 

\textbf{В какой момент процесс окончательно исчезает из таблицы процессов?}

После того как его родительский процесс вызовет wait (2), waitpid (2), waitid(2) - позволяет завершить процесс, если он таки стал зомби. 

Традиционная проблема --- порождая процесс, необходимо в какой-то момент сказать ему wait (2), чтобы не плодить зомби. В противном случае у вас может закончится лимит на создание новых процессов. 

\textbf{Что происходит с зомби, если его родитель не вызвал wait (2), а погиб?}

Его родителем становится init. Для всех процессов, которые становятся его потомками он делает wait (2). По факту, если вы написали какой-то код, который выполняет некоторое количество процессов после чего завершился, то на самом деле зомби в системе не останутся, т.к. после того как ваш основной процесс завершился, то потомков этого процесса подхватит init и он сам вызовет для них wait (2).


         \subsection{Семейство системных вызовов exec}
         Жизненный цикл процесса начинается с того, что его кто-то порождает. Единственный процесс, который никто не порождает, это процесс init. Порождение процесса осуществляется с помощью fork (2) или vfork (2). 

Первая стадия в жизни процесса --- \textbf{“инициализация“}. В это время ядро производит подготовительные работы к дальнейшей работе процесса. На самом деле, это не совсем состояние, но логически для пользователя оно существует.

\begin{figure}[htbp]
  \centering
  \includegraphics[width=0.7\textwidth]{./processes-and-threads/processes/lifecycle/proc-lifecycle.png}
\end{figure}

Когда инициализация процесса завершается --- он оказывается в стадии \textbf{“готов“}. С этого момента процесс находится в ожидании момента, когда его выберет планировщик процессов и даст ему какой-то квант процессорного времени.

При получении процессорного времени процесс переходит в состояние \textbf{“выполняется“}. Выполняться процесс может в режиме ядра --- при осуществлении системных вызовов, прерываний и в режиме задачи --- выполнять инструкции процессора. По окончании квоты времени, процесс может снова вернуться в состояние \textbf{“готов“}.

Также из состояния “выполняется“ процесс может перейти еще в два состояния --- \textbf{“остановлен“} и \textbf{“ожидает“}. Если он остановлен, то он остановлен пользователем (например, получен сигнал SIGSTOP). Ожидание наступает тогда, когда процессу нужны какие-то ресурсы. Как только условие, которого ждет процессор, выполняется, он переходит в состояние \textbf{“готов“}, то есть ожидает своей очереди на выполнение. 

Из каждого состояния процесс может перейти в состояние \textbf{“зомби“}. Этот процесс нужен для того, чтобы родительский процесс мог получить код возврата этого процесса. Это промежуточное состояние --- процесса уже нет, но он технически еще есть. Тем не менее, это самое что ни на есть валидное состояние. 

\textbf{В какой момент процесс окончательно исчезает из таблицы процессов?}

После того как его родительский процесс вызовет wait (2), waitpid (2), waitid(2) - позволяет завершить процесс, если он таки стал зомби. 

Традиционная проблема --- порождая процесс, необходимо в какой-то момент сказать ему wait (2), чтобы не плодить зомби. В противном случае у вас может закончится лимит на создание новых процессов. 

\textbf{Что происходит с зомби, если его родитель не вызвал wait (2), а погиб?}

Его родителем становится init. Для всех процессов, которые становятся его потомками он делает wait (2). По факту, если вы написали какой-то код, который выполняет некоторое количество процессов после чего завершился, то на самом деле зомби в системе не останутся, т.к. после того как ваш основной процесс завершился, то потомков этого процесса подхватит init и он сам вызовет для них wait (2).


   \chapter{Межпроцессное взаимодействие}
   Жизненный цикл процесса начинается с того, что его кто-то порождает. Единственный процесс, который никто не порождает, это процесс init. Порождение процесса осуществляется с помощью fork (2) или vfork (2). 

Первая стадия в жизни процесса --- \textbf{“инициализация“}. В это время ядро производит подготовительные работы к дальнейшей работе процесса. На самом деле, это не совсем состояние, но логически для пользователя оно существует.

\begin{figure}[htbp]
  \centering
  \includegraphics[width=0.7\textwidth]{./processes-and-threads/processes/lifecycle/proc-lifecycle.png}
\end{figure}

Когда инициализация процесса завершается --- он оказывается в стадии \textbf{“готов“}. С этого момента процесс находится в ожидании момента, когда его выберет планировщик процессов и даст ему какой-то квант процессорного времени.

При получении процессорного времени процесс переходит в состояние \textbf{“выполняется“}. Выполняться процесс может в режиме ядра --- при осуществлении системных вызовов, прерываний и в режиме задачи --- выполнять инструкции процессора. По окончании квоты времени, процесс может снова вернуться в состояние \textbf{“готов“}.

Также из состояния “выполняется“ процесс может перейти еще в два состояния --- \textbf{“остановлен“} и \textbf{“ожидает“}. Если он остановлен, то он остановлен пользователем (например, получен сигнал SIGSTOP). Ожидание наступает тогда, когда процессу нужны какие-то ресурсы. Как только условие, которого ждет процессор, выполняется, он переходит в состояние \textbf{“готов“}, то есть ожидает своей очереди на выполнение. 

Из каждого состояния процесс может перейти в состояние \textbf{“зомби“}. Этот процесс нужен для того, чтобы родительский процесс мог получить код возврата этого процесса. Это промежуточное состояние --- процесса уже нет, но он технически еще есть. Тем не менее, это самое что ни на есть валидное состояние. 

\textbf{В какой момент процесс окончательно исчезает из таблицы процессов?}

После того как его родительский процесс вызовет wait (2), waitpid (2), waitid(2) - позволяет завершить процесс, если он таки стал зомби. 

Традиционная проблема --- порождая процесс, необходимо в какой-то момент сказать ему wait (2), чтобы не плодить зомби. В противном случае у вас может закончится лимит на создание новых процессов. 

\textbf{Что происходит с зомби, если его родитель не вызвал wait (2), а погиб?}

Его родителем становится init. Для всех процессов, которые становятся его потомками он делает wait (2). По факту, если вы написали какой-то код, который выполняет некоторое количество процессов после чего завершился, то на самом деле зомби в системе не останутся, т.к. после того как ваш основной процесс завершился, то потомков этого процесса подхватит init и он сам вызовет для них wait (2).
	

      \section{Семафоры}
      Жизненный цикл процесса начинается с того, что его кто-то порождает. Единственный процесс, который никто не порождает, это процесс init. Порождение процесса осуществляется с помощью fork (2) или vfork (2). 

Первая стадия в жизни процесса --- \textbf{“инициализация“}. В это время ядро производит подготовительные работы к дальнейшей работе процесса. На самом деле, это не совсем состояние, но логически для пользователя оно существует.

\begin{figure}[htbp]
  \centering
  \includegraphics[width=0.7\textwidth]{./processes-and-threads/processes/lifecycle/proc-lifecycle.png}
\end{figure}

Когда инициализация процесса завершается --- он оказывается в стадии \textbf{“готов“}. С этого момента процесс находится в ожидании момента, когда его выберет планировщик процессов и даст ему какой-то квант процессорного времени.

При получении процессорного времени процесс переходит в состояние \textbf{“выполняется“}. Выполняться процесс может в режиме ядра --- при осуществлении системных вызовов, прерываний и в режиме задачи --- выполнять инструкции процессора. По окончании квоты времени, процесс может снова вернуться в состояние \textbf{“готов“}.

Также из состояния “выполняется“ процесс может перейти еще в два состояния --- \textbf{“остановлен“} и \textbf{“ожидает“}. Если он остановлен, то он остановлен пользователем (например, получен сигнал SIGSTOP). Ожидание наступает тогда, когда процессу нужны какие-то ресурсы. Как только условие, которого ждет процессор, выполняется, он переходит в состояние \textbf{“готов“}, то есть ожидает своей очереди на выполнение. 

Из каждого состояния процесс может перейти в состояние \textbf{“зомби“}. Этот процесс нужен для того, чтобы родительский процесс мог получить код возврата этого процесса. Это промежуточное состояние --- процесса уже нет, но он технически еще есть. Тем не менее, это самое что ни на есть валидное состояние. 

\textbf{В какой момент процесс окончательно исчезает из таблицы процессов?}

После того как его родительский процесс вызовет wait (2), waitpid (2), waitid(2) - позволяет завершить процесс, если он таки стал зомби. 

Традиционная проблема --- порождая процесс, необходимо в какой-то момент сказать ему wait (2), чтобы не плодить зомби. В противном случае у вас может закончится лимит на создание новых процессов. 

\textbf{Что происходит с зомби, если его родитель не вызвал wait (2), а погиб?}

Его родителем становится init. Для всех процессов, которые становятся его потомками он делает wait (2). По факту, если вы написали какой-то код, который выполняет некоторое количество процессов после чего завершился, то на самом деле зомби в системе не останутся, т.к. после того как ваш основной процесс завершился, то потомков этого процесса подхватит init и он сам вызовет для них wait (2).


      \section{Каналы}

         \subsection{Неименованные каналы}
         Для создания неименованного калала существуют системные вызовы pipe (2) и pipe2 (2). Оба они принимают на вход массив из двух элементов. После успешного выполнения вызова массив содержит два файловых дескриптора: для чтения информации из канала и для записи в него соответственно. 

\begin{CCode}{pipe(2)}
	#include <unistd.h>
	
	int pipe(
		filedes[2]
	); \end{CCode}

Системный вызов pipe2 (2) принимает также некоторые флаги, влияющие на поведение канала.

Неименованные каналы можно использовать для родственных процессов. Когда процесс порождает другой процесс, дескрипторы родителя наследуются дочерним процессом, и, таким образом, осуществляется связь между двумя процессами. Один из них использует канал только для чтения, а другой только --- для записи.


         \subsection{Именованные каналы}
         Жизненный цикл процесса начинается с того, что его кто-то порождает. Единственный процесс, который никто не порождает, это процесс init. Порождение процесса осуществляется с помощью fork (2) или vfork (2). 

Первая стадия в жизни процесса --- \textbf{“инициализация“}. В это время ядро производит подготовительные работы к дальнейшей работе процесса. На самом деле, это не совсем состояние, но логически для пользователя оно существует.

\begin{figure}[htbp]
  \centering
  \includegraphics[width=0.7\textwidth]{./processes-and-threads/processes/lifecycle/proc-lifecycle.png}
\end{figure}

Когда инициализация процесса завершается --- он оказывается в стадии \textbf{“готов“}. С этого момента процесс находится в ожидании момента, когда его выберет планировщик процессов и даст ему какой-то квант процессорного времени.

При получении процессорного времени процесс переходит в состояние \textbf{“выполняется“}. Выполняться процесс может в режиме ядра --- при осуществлении системных вызовов, прерываний и в режиме задачи --- выполнять инструкции процессора. По окончании квоты времени, процесс может снова вернуться в состояние \textbf{“готов“}.

Также из состояния “выполняется“ процесс может перейти еще в два состояния --- \textbf{“остановлен“} и \textbf{“ожидает“}. Если он остановлен, то он остановлен пользователем (например, получен сигнал SIGSTOP). Ожидание наступает тогда, когда процессу нужны какие-то ресурсы. Как только условие, которого ждет процессор, выполняется, он переходит в состояние \textbf{“готов“}, то есть ожидает своей очереди на выполнение. 

Из каждого состояния процесс может перейти в состояние \textbf{“зомби“}. Этот процесс нужен для того, чтобы родительский процесс мог получить код возврата этого процесса. Это промежуточное состояние --- процесса уже нет, но он технически еще есть. Тем не менее, это самое что ни на есть валидное состояние. 

\textbf{В какой момент процесс окончательно исчезает из таблицы процессов?}

После того как его родительский процесс вызовет wait (2), waitpid (2), waitid(2) - позволяет завершить процесс, если он таки стал зомби. 

Традиционная проблема --- порождая процесс, необходимо в какой-то момент сказать ему wait (2), чтобы не плодить зомби. В противном случае у вас может закончится лимит на создание новых процессов. 

\textbf{Что происходит с зомби, если его родитель не вызвал wait (2), а погиб?}

Его родителем становится init. Для всех процессов, которые становятся его потомками он делает wait (2). По факту, если вы написали какой-то код, который выполняет некоторое количество процессов после чего завершился, то на самом деле зомби в системе не останутся, т.к. после того как ваш основной процесс завершился, то потомков этого процесса подхватит init и он сам вызовет для них wait (2).
	

      \section{Сокеты}
      Как уже было сказано, сокеты --- это двусторонние каналы передачи данных между двумя единицами соединения. Существует несколько видов сокетов:

\begin{itemize}
	\item UNIX domain 
	\item Сетевые 
	\item BSD сокеты 
	\item Прочие
\end{itemize}

Они различаются по семействам адресов. 

Сокет подразумевает, что у вас есть некий сервер, который слушает определенный сокет.


\textbf{Для работы с сокетами существуют следующие системные вызовы.}

Cистемный вызов socket(2) создает новый сокет указанного типа и возвращает номер файлового дескриптора или код ошибки. В качестве аргументов этот системный вызов принимает семейство адресов, тип соединения и протокол.

\begin{CCode}{socket(2)}
	int socket( 
		int domain, 	/* address domain */ 
		int type,		/* type */ 
		int protocol	/* protocol (ex. tcp, udp) */ 
	); \end{CCode}

Системный вызов bind(2) используется на стороне сервера для того, чтобы ассоциировать сокет с адресом (описанным структурой address), например, адресом хоста и портом. Этот системный вызов возвращает 0 или код ошибки. Вызов принимает номер файлового дескриптора сокета, имя и длину поля имени.

\begin{CCode}{bind(2)}
	int bind( 
		int s, 							/* socket's descriptor number */ 
		const struct sockaddr *name, 	/* name */ 
		int namelen 					/* size of name in bytes */ 
	); \end{CCode}

Системный вызов listen(2) используется на стороне сервера для того, чтобы TCP сокет, связанный с адресом, начал “слушать” указанный порт. Системный вызов принимает номер файлового дескриптора сокета и длину очереди. Возвращает 0 или код ошибки.

\begin{CCode}{listen(2)}
	int listen( 
		int s, 			/* socket's descriptor number */ 
		int backlog 	/* length of queue */ 
	); \end{CCode}


Системный вызов accept(2) используется на стороне сервера. Принимает входящее соединение, создает новое TCP соединение и новый сокет, связанный с подключенным. Возвращает номер дескриптора или код ошибки.

\begin{CCode}{accept(2)}
	int accept( 
		int s, 					/* socket's descriptor number */ 
		struct sockaddr *addr,	/* client's address */ 
		socklen_t *addrlen 		/* size of client's address */ 
	); \end{CCode} 


Системный вызов connect(2) используется на стороне клиента и назначает свободный порт сокету. В случае TCP сокета пытается установить новое TCP соединение. Этот вызов принимает номер файлового дескриптора сокета, имя и длину поля имени. Он возвращает 0 или код ошибки.

\begin{CCode}{connect(2)}
	int connect( 
		int s,							/* socket's descriptor number */ 
		const struct sockaddr *name, 	/* name */ 
		int namelen 					/* size of name in bytes */ 
	); \end{CCode} 


      \section{Сигналы}
      Жизненный цикл процесса начинается с того, что его кто-то порождает. Единственный процесс, который никто не порождает, это процесс init. Порождение процесса осуществляется с помощью fork (2) или vfork (2). 

Первая стадия в жизни процесса --- \textbf{“инициализация“}. В это время ядро производит подготовительные работы к дальнейшей работе процесса. На самом деле, это не совсем состояние, но логически для пользователя оно существует.

\begin{figure}[htbp]
  \centering
  \includegraphics[width=0.7\textwidth]{./processes-and-threads/processes/lifecycle/proc-lifecycle.png}
\end{figure}

Когда инициализация процесса завершается --- он оказывается в стадии \textbf{“готов“}. С этого момента процесс находится в ожидании момента, когда его выберет планировщик процессов и даст ему какой-то квант процессорного времени.

При получении процессорного времени процесс переходит в состояние \textbf{“выполняется“}. Выполняться процесс может в режиме ядра --- при осуществлении системных вызовов, прерываний и в режиме задачи --- выполнять инструкции процессора. По окончании квоты времени, процесс может снова вернуться в состояние \textbf{“готов“}.

Также из состояния “выполняется“ процесс может перейти еще в два состояния --- \textbf{“остановлен“} и \textbf{“ожидает“}. Если он остановлен, то он остановлен пользователем (например, получен сигнал SIGSTOP). Ожидание наступает тогда, когда процессу нужны какие-то ресурсы. Как только условие, которого ждет процессор, выполняется, он переходит в состояние \textbf{“готов“}, то есть ожидает своей очереди на выполнение. 

Из каждого состояния процесс может перейти в состояние \textbf{“зомби“}. Этот процесс нужен для того, чтобы родительский процесс мог получить код возврата этого процесса. Это промежуточное состояние --- процесса уже нет, но он технически еще есть. Тем не менее, это самое что ни на есть валидное состояние. 

\textbf{В какой момент процесс окончательно исчезает из таблицы процессов?}

После того как его родительский процесс вызовет wait (2), waitpid (2), waitid(2) - позволяет завершить процесс, если он таки стал зомби. 

Традиционная проблема --- порождая процесс, необходимо в какой-то момент сказать ему wait (2), чтобы не плодить зомби. В противном случае у вас может закончится лимит на создание новых процессов. 

\textbf{Что происходит с зомби, если его родитель не вызвал wait (2), а погиб?}

Его родителем становится init. Для всех процессов, которые становятся его потомками он делает wait (2). По факту, если вы написали какой-то код, который выполняет некоторое количество процессов после чего завершился, то на самом деле зомби в системе не останутся, т.к. после того как ваш основной процесс завершился, то потомков этого процесса подхватит init и он сам вызовет для них wait (2).


         \subsection{Обработка сигналов}
         На все сигналы, кроме SIG\_STOP и SIG\_KILL, вы можете самостоятельно написать свой обработчик сигналов --- собственную функцию, которая будет реагировать на пришедший сигнал так, как вы в ней описали.

Свой обработчик сигналов можно сделать разными способами.

Библиотека libc предлагает механизм, который называется signal (3). Эта функция принимает в качестве аргементов номер сигнала и новую диспозицию сигнала. В качестве диспозиции можно подать функцию, которую вы хотите назначить обработчиком сигнала. Функция возвращает предыдущую диспозицию.

\begin{CCode}{signal (3)}
	#include <signal.h>

    void (*signal(int sig, void (*disp)(int)))(int); \end{CCode}
~\\[0.5cm]

\textbf{Какие недостатки есть у signal (3)?}

Во многих реализация libC диспозиция сигнала устанавливается на действие по умолчанию каждый раз при получении сигнала.

Кроме того, у процесса в контексте есть такое понятие, как маска принимаемых сигналов. Фактически, это битовая последовательность, которая маскирует сигналы от принятия при их передаче. Если в этой битовой последовательности первый бит установлен в единицу, то, при попытке доставить сигнал процессу, операционная система увидит, что этот сигнал замаскирован, и поэтому его доставлять не нужно --- он будет просто отброшен.

Есть еще один более гибкий традиционно используемый способ обработки сигналов --- это системный вызов sigaction (2), который принимает числовой номер сигнала и два указателя на структуру sig\_action.

Первый указатель нужен для того, чтобы указать ОС, что мы хотим делать в случае прихода этого сигнала, т.е. во второй указатель помещается реакция на сигнал после выполнения системного вызова sigaction (2).

\begin{CCode}{sigaction (2)}
	#include <signal.h>

     int sigaction(int sig, const struct sigaction *restrict act,
         struct sigaction *restrict oact);  \end{CCode}

Структура sigaction включает в себя следующие поля:

\begin{itemize}
	\item void (*sa\_handler)(int) -- указатель на функцию обработчик сигнала;
	\item void (*sa\_sigaction)(int, siginfo\_t *, void *) -- указатель на функцию обработчик сигнала, если установлен флаг SA\_SIGINFO;
	\item sa\_flags -- флаги;
	\item sa\_mask -- битовая маска, которая маскирует прием/доставку сигналов процессу.
\end{itemize}
	

      \section{Очереди сообщений}
      Жизненный цикл процесса начинается с того, что его кто-то порождает. Единственный процесс, который никто не порождает, это процесс init. Порождение процесса осуществляется с помощью fork (2) или vfork (2). 

Первая стадия в жизни процесса --- \textbf{“инициализация“}. В это время ядро производит подготовительные работы к дальнейшей работе процесса. На самом деле, это не совсем состояние, но логически для пользователя оно существует.

\begin{figure}[htbp]
  \centering
  \includegraphics[width=0.7\textwidth]{./processes-and-threads/processes/lifecycle/proc-lifecycle.png}
\end{figure}

Когда инициализация процесса завершается --- он оказывается в стадии \textbf{“готов“}. С этого момента процесс находится в ожидании момента, когда его выберет планировщик процессов и даст ему какой-то квант процессорного времени.

При получении процессорного времени процесс переходит в состояние \textbf{“выполняется“}. Выполняться процесс может в режиме ядра --- при осуществлении системных вызовов, прерываний и в режиме задачи --- выполнять инструкции процессора. По окончании квоты времени, процесс может снова вернуться в состояние \textbf{“готов“}.

Также из состояния “выполняется“ процесс может перейти еще в два состояния --- \textbf{“остановлен“} и \textbf{“ожидает“}. Если он остановлен, то он остановлен пользователем (например, получен сигнал SIGSTOP). Ожидание наступает тогда, когда процессу нужны какие-то ресурсы. Как только условие, которого ждет процессор, выполняется, он переходит в состояние \textbf{“готов“}, то есть ожидает своей очереди на выполнение. 

Из каждого состояния процесс может перейти в состояние \textbf{“зомби“}. Этот процесс нужен для того, чтобы родительский процесс мог получить код возврата этого процесса. Это промежуточное состояние --- процесса уже нет, но он технически еще есть. Тем не менее, это самое что ни на есть валидное состояние. 

\textbf{В какой момент процесс окончательно исчезает из таблицы процессов?}

После того как его родительский процесс вызовет wait (2), waitpid (2), waitid(2) - позволяет завершить процесс, если он таки стал зомби. 

Традиционная проблема --- порождая процесс, необходимо в какой-то момент сказать ему wait (2), чтобы не плодить зомби. В противном случае у вас может закончится лимит на создание новых процессов. 

\textbf{Что происходит с зомби, если его родитель не вызвал wait (2), а погиб?}

Его родителем становится init. Для всех процессов, которые становятся его потомками он делает wait (2). По факту, если вы написали какой-то код, который выполняет некоторое количество процессов после чего завершился, то на самом деле зомби в системе не останутся, т.к. после того как ваш основной процесс завершился, то потомков этого процесса подхватит init и он сам вызовет для них wait (2).
	

      \section{Разделяемая память}

      \section{Механизм STREAMS}
      Жизненный цикл процесса начинается с того, что его кто-то порождает. Единственный процесс, который никто не порождает, это процесс init. Порождение процесса осуществляется с помощью fork (2) или vfork (2). 

Первая стадия в жизни процесса --- \textbf{“инициализация“}. В это время ядро производит подготовительные работы к дальнейшей работе процесса. На самом деле, это не совсем состояние, но логически для пользователя оно существует.

\begin{figure}[htbp]
  \centering
  \includegraphics[width=0.7\textwidth]{./processes-and-threads/processes/lifecycle/proc-lifecycle.png}
\end{figure}

Когда инициализация процесса завершается --- он оказывается в стадии \textbf{“готов“}. С этого момента процесс находится в ожидании момента, когда его выберет планировщик процессов и даст ему какой-то квант процессорного времени.

При получении процессорного времени процесс переходит в состояние \textbf{“выполняется“}. Выполняться процесс может в режиме ядра --- при осуществлении системных вызовов, прерываний и в режиме задачи --- выполнять инструкции процессора. По окончании квоты времени, процесс может снова вернуться в состояние \textbf{“готов“}.

Также из состояния “выполняется“ процесс может перейти еще в два состояния --- \textbf{“остановлен“} и \textbf{“ожидает“}. Если он остановлен, то он остановлен пользователем (например, получен сигнал SIGSTOP). Ожидание наступает тогда, когда процессу нужны какие-то ресурсы. Как только условие, которого ждет процессор, выполняется, он переходит в состояние \textbf{“готов“}, то есть ожидает своей очереди на выполнение. 

Из каждого состояния процесс может перейти в состояние \textbf{“зомби“}. Этот процесс нужен для того, чтобы родительский процесс мог получить код возврата этого процесса. Это промежуточное состояние --- процесса уже нет, но он технически еще есть. Тем не менее, это самое что ни на есть валидное состояние. 

\textbf{В какой момент процесс окончательно исчезает из таблицы процессов?}

После того как его родительский процесс вызовет wait (2), waitpid (2), waitid(2) - позволяет завершить процесс, если он таки стал зомби. 

Традиционная проблема --- порождая процесс, необходимо в какой-то момент сказать ему wait (2), чтобы не плодить зомби. В противном случае у вас может закончится лимит на создание новых процессов. 

\textbf{Что происходит с зомби, если его родитель не вызвал wait (2), а погиб?}

Его родителем становится init. Для всех процессов, которые становятся его потомками он делает wait (2). По факту, если вы написали какой-то код, который выполняет некоторое количество процессов после чего завершился, то на самом деле зомби в системе не останутся, т.к. после того как ваш основной процесс завершился, то потомков этого процесса подхватит init и он сам вызовет для них wait (2).
	

   \chapter{Потоки}
   Какой есть недостаток у процессов? Для каждого процесса необходимо новое адресное пространство. Порождение процессов --- достаточно длительный и затратный процесс с точки зрения операционной системы. Альтернативой процессам являются потоки. Принципиальная разница в том, что они используют общее адресное пространство для нескольких наборов последовательно выполняемых команд.

\begin{defi}{Поток}
	наименьший набор инструкций, которым может быть выделена квота процессорного времени 
\end{defi}

Потоки одного процесса имеют единое с ним адресное пространство. Наиболее используемый стандарт –-- POSIX. Реализуются семейством функций с префиксом «pthread\_»


      \section{Легковесные процессы}
      Жизненный цикл процесса начинается с того, что его кто-то порождает. Единственный процесс, который никто не порождает, это процесс init. Порождение процесса осуществляется с помощью fork (2) или vfork (2). 

Первая стадия в жизни процесса --- \textbf{“инициализация“}. В это время ядро производит подготовительные работы к дальнейшей работе процесса. На самом деле, это не совсем состояние, но логически для пользователя оно существует.

\begin{figure}[htbp]
  \centering
  \includegraphics[width=0.7\textwidth]{./processes-and-threads/processes/lifecycle/proc-lifecycle.png}
\end{figure}

Когда инициализация процесса завершается --- он оказывается в стадии \textbf{“готов“}. С этого момента процесс находится в ожидании момента, когда его выберет планировщик процессов и даст ему какой-то квант процессорного времени.

При получении процессорного времени процесс переходит в состояние \textbf{“выполняется“}. Выполняться процесс может в режиме ядра --- при осуществлении системных вызовов, прерываний и в режиме задачи --- выполнять инструкции процессора. По окончании квоты времени, процесс может снова вернуться в состояние \textbf{“готов“}.

Также из состояния “выполняется“ процесс может перейти еще в два состояния --- \textbf{“остановлен“} и \textbf{“ожидает“}. Если он остановлен, то он остановлен пользователем (например, получен сигнал SIGSTOP). Ожидание наступает тогда, когда процессу нужны какие-то ресурсы. Как только условие, которого ждет процессор, выполняется, он переходит в состояние \textbf{“готов“}, то есть ожидает своей очереди на выполнение. 

Из каждого состояния процесс может перейти в состояние \textbf{“зомби“}. Этот процесс нужен для того, чтобы родительский процесс мог получить код возврата этого процесса. Это промежуточное состояние --- процесса уже нет, но он технически еще есть. Тем не менее, это самое что ни на есть валидное состояние. 

\textbf{В какой момент процесс окончательно исчезает из таблицы процессов?}

После того как его родительский процесс вызовет wait (2), waitpid (2), waitid(2) - позволяет завершить процесс, если он таки стал зомби. 

Традиционная проблема --- порождая процесс, необходимо в какой-то момент сказать ему wait (2), чтобы не плодить зомби. В противном случае у вас может закончится лимит на создание новых процессов. 

\textbf{Что происходит с зомби, если его родитель не вызвал wait (2), а погиб?}

Его родителем становится init. Для всех процессов, которые становятся его потомками он делает wait (2). По факту, если вы написали какой-то код, который выполняет некоторое количество процессов после чего завершился, то на самом деле зомби в системе не останутся, т.к. после того как ваш основной процесс завершился, то потомков этого процесса подхватит init и он сам вызовет для них wait (2).


      \section{Межпоточное взаимодействие}
      Жизненный цикл процесса начинается с того, что его кто-то порождает. Единственный процесс, который никто не порождает, это процесс init. Порождение процесса осуществляется с помощью fork (2) или vfork (2). 

Первая стадия в жизни процесса --- \textbf{“инициализация“}. В это время ядро производит подготовительные работы к дальнейшей работе процесса. На самом деле, это не совсем состояние, но логически для пользователя оно существует.

\begin{figure}[htbp]
  \centering
  \includegraphics[width=0.7\textwidth]{./processes-and-threads/processes/lifecycle/proc-lifecycle.png}
\end{figure}

Когда инициализация процесса завершается --- он оказывается в стадии \textbf{“готов“}. С этого момента процесс находится в ожидании момента, когда его выберет планировщик процессов и даст ему какой-то квант процессорного времени.

При получении процессорного времени процесс переходит в состояние \textbf{“выполняется“}. Выполняться процесс может в режиме ядра --- при осуществлении системных вызовов, прерываний и в режиме задачи --- выполнять инструкции процессора. По окончании квоты времени, процесс может снова вернуться в состояние \textbf{“готов“}.

Также из состояния “выполняется“ процесс может перейти еще в два состояния --- \textbf{“остановлен“} и \textbf{“ожидает“}. Если он остановлен, то он остановлен пользователем (например, получен сигнал SIGSTOP). Ожидание наступает тогда, когда процессу нужны какие-то ресурсы. Как только условие, которого ждет процессор, выполняется, он переходит в состояние \textbf{“готов“}, то есть ожидает своей очереди на выполнение. 

Из каждого состояния процесс может перейти в состояние \textbf{“зомби“}. Этот процесс нужен для того, чтобы родительский процесс мог получить код возврата этого процесса. Это промежуточное состояние --- процесса уже нет, но он технически еще есть. Тем не менее, это самое что ни на есть валидное состояние. 

\textbf{В какой момент процесс окончательно исчезает из таблицы процессов?}

После того как его родительский процесс вызовет wait (2), waitpid (2), waitid(2) - позволяет завершить процесс, если он таки стал зомби. 

Традиционная проблема --- порождая процесс, необходимо в какой-то момент сказать ему wait (2), чтобы не плодить зомби. В противном случае у вас может закончится лимит на создание новых процессов. 

\textbf{Что происходит с зомби, если его родитель не вызвал wait (2), а погиб?}

Его родителем становится init. Для всех процессов, которые становятся его потомками он делает wait (2). По факту, если вы написали какой-то код, который выполняет некоторое количество процессов после чего завершился, то на самом деле зомби в системе не останутся, т.к. после того как ваш основной процесс завершился, то потомков этого процесса подхватит init и он сам вызовет для них wait (2).
	

         \subsection{Общее адресное пространство}

         \subsection{Переменные volatile}
         Жизненный цикл процесса начинается с того, что его кто-то порождает. Единственный процесс, который никто не порождает, это процесс init. Порождение процесса осуществляется с помощью fork (2) или vfork (2). 

Первая стадия в жизни процесса --- \textbf{“инициализация“}. В это время ядро производит подготовительные работы к дальнейшей работе процесса. На самом деле, это не совсем состояние, но логически для пользователя оно существует.

\begin{figure}[htbp]
  \centering
  \includegraphics[width=0.7\textwidth]{./processes-and-threads/processes/lifecycle/proc-lifecycle.png}
\end{figure}

Когда инициализация процесса завершается --- он оказывается в стадии \textbf{“готов“}. С этого момента процесс находится в ожидании момента, когда его выберет планировщик процессов и даст ему какой-то квант процессорного времени.

При получении процессорного времени процесс переходит в состояние \textbf{“выполняется“}. Выполняться процесс может в режиме ядра --- при осуществлении системных вызовов, прерываний и в режиме задачи --- выполнять инструкции процессора. По окончании квоты времени, процесс может снова вернуться в состояние \textbf{“готов“}.

Также из состояния “выполняется“ процесс может перейти еще в два состояния --- \textbf{“остановлен“} и \textbf{“ожидает“}. Если он остановлен, то он остановлен пользователем (например, получен сигнал SIGSTOP). Ожидание наступает тогда, когда процессу нужны какие-то ресурсы. Как только условие, которого ждет процессор, выполняется, он переходит в состояние \textbf{“готов“}, то есть ожидает своей очереди на выполнение. 

Из каждого состояния процесс может перейти в состояние \textbf{“зомби“}. Этот процесс нужен для того, чтобы родительский процесс мог получить код возврата этого процесса. Это промежуточное состояние --- процесса уже нет, но он технически еще есть. Тем не менее, это самое что ни на есть валидное состояние. 

\textbf{В какой момент процесс окончательно исчезает из таблицы процессов?}

После того как его родительский процесс вызовет wait (2), waitpid (2), waitid(2) - позволяет завершить процесс, если он таки стал зомби. 

Традиционная проблема --- порождая процесс, необходимо в какой-то момент сказать ему wait (2), чтобы не плодить зомби. В противном случае у вас может закончится лимит на создание новых процессов. 

\textbf{Что происходит с зомби, если его родитель не вызвал wait (2), а погиб?}

Его родителем становится init. Для всех процессов, которые становятся его потомками он делает wait (2). По факту, если вы написали какой-то код, который выполняет некоторое количество процессов после чего завершился, то на самом деле зомби в системе не останутся, т.к. после того как ваш основной процесс завершился, то потомков этого процесса подхватит init и он сам вызовет для них wait (2).
		

         \subsection{Мьютексы}
         \begin{defi}{Мьютекс}
	простейший объект синхронизации, имеющий два состояния: «заблокирован» и «свободен».
\end{defi}

При использовании мьютекса только один поток в определенный момент времени может заблокировать мьютекс и получить доступ к разделяемому ресурсу. При завершении работы с ресурсом поток должен разблокировать мьютекс. 

Реализованы pthread\_mutex. 

\begin{CCode}{Приведем некоторые функции для работы с мьютексами:}
	#include <pthread.h>

	int pthread_mutex_init(pthread_mutex_t *mutex, 
            const pthread_mutexattr_t *attr);
	
	int pthread_mutex_destroy(pthread_mutex_t *mutex);

	int pthread_mutex_lock(pthread_mutex_t *mutex);

	int pthread_mutex_trylock(pthread_mutex_t *mutex);

	int pthread_mutex_unlock(pthread_mutex_t *mutex); \end{CCode}

Вам предлагается ознакомиться с ними самостоятельно.

Надо понимать, что используя мьютексы, вы можете эмулировать rwlock . Используя rwlock, вы можете эмулировать семафоры. Используя семафоры --- любое поведение. Самое главное --- вам нужно определиться, что и для какого случая использовать. Они условно взаимозаменяемы, но всё зависит от того, что конкретно вам нужно от кода.
	

         \subsection{rwlock}
         Жизненный цикл процесса начинается с того, что его кто-то порождает. Единственный процесс, который никто не порождает, это процесс init. Порождение процесса осуществляется с помощью fork (2) или vfork (2). 

Первая стадия в жизни процесса --- \textbf{“инициализация“}. В это время ядро производит подготовительные работы к дальнейшей работе процесса. На самом деле, это не совсем состояние, но логически для пользователя оно существует.

\begin{figure}[htbp]
  \centering
  \includegraphics[width=0.7\textwidth]{./processes-and-threads/processes/lifecycle/proc-lifecycle.png}
\end{figure}

Когда инициализация процесса завершается --- он оказывается в стадии \textbf{“готов“}. С этого момента процесс находится в ожидании момента, когда его выберет планировщик процессов и даст ему какой-то квант процессорного времени.

При получении процессорного времени процесс переходит в состояние \textbf{“выполняется“}. Выполняться процесс может в режиме ядра --- при осуществлении системных вызовов, прерываний и в режиме задачи --- выполнять инструкции процессора. По окончании квоты времени, процесс может снова вернуться в состояние \textbf{“готов“}.

Также из состояния “выполняется“ процесс может перейти еще в два состояния --- \textbf{“остановлен“} и \textbf{“ожидает“}. Если он остановлен, то он остановлен пользователем (например, получен сигнал SIGSTOP). Ожидание наступает тогда, когда процессу нужны какие-то ресурсы. Как только условие, которого ждет процессор, выполняется, он переходит в состояние \textbf{“готов“}, то есть ожидает своей очереди на выполнение. 

Из каждого состояния процесс может перейти в состояние \textbf{“зомби“}. Этот процесс нужен для того, чтобы родительский процесс мог получить код возврата этого процесса. Это промежуточное состояние --- процесса уже нет, но он технически еще есть. Тем не менее, это самое что ни на есть валидное состояние. 

\textbf{В какой момент процесс окончательно исчезает из таблицы процессов?}

После того как его родительский процесс вызовет wait (2), waitpid (2), waitid(2) - позволяет завершить процесс, если он таки стал зомби. 

Традиционная проблема --- порождая процесс, необходимо в какой-то момент сказать ему wait (2), чтобы не плодить зомби. В противном случае у вас может закончится лимит на создание новых процессов. 

\textbf{Что происходит с зомби, если его родитель не вызвал wait (2), а погиб?}

Его родителем становится init. Для всех процессов, которые становятся его потомками он делает wait (2). По факту, если вы написали какой-то код, который выполняет некоторое количество процессов после чего завершился, то на самом деле зомби в системе не останутся, т.к. после того как ваш основной процесс завершился, то потомков этого процесса подхватит init и он сам вызовет для них wait (2).
	


\clearpage
\addcontentsline{toc}{chapter}{Литература}
\input{./resources.tex}

\end{document}
