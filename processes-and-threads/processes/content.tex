\begin{defi}{Процесс}
	совокупность программы и метаинформации, описывающей её выполнение
\end{defi}

Когда мы говорим процесс, речь идет именно о выполняющейся программе. Когда программа не выполняется её нельзя назвать процессом.

Каждый процесс представлен в системе двумя основными структурами данных — proc и u\_block, описанными, соответственно, в файлах <sys/proc.h> и <sys/user.h>. Содержимое и формат этих структур различны для разных версий UNIX.
