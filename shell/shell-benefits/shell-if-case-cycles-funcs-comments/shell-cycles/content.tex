Довольно часто условные конструкции можно опускать за счет использования “\&\&” или “||”, а вот такие вещи как циклы имитировать просто так не получится

Всего в shell три вида циклов - while, for и until, рассмотрим их подробнее.

\begin{shCode}{WHILE}
	WHILE ( <condition> )
		DO
	 		<actions>
		DONE
\end{shCode}
Код внутри do - done будет выполняться пока условие истинно

\begin{shCode}{UNTIL}
	UNTIL ( <condition> )
		DO
			<actions>
		DONE
\end{shCode}
Код внутри do - done будет выполняться пока условие ложно

\begin{shCode}{FOR}
	FOR var IN ( <list of values> )
		DO
			<actions>
		DONE
\end{shCode}
Обеспечивает выполнения столько раз, сколько значений в списке значений, при этом переменная var последовательно принимает значения из списка значений при каждой новой итерации

Такая форма for в принципе не вызывает никаких вопросов. Но что будет, если опустить список значений? Дело в том, что это позволит неявно пройти по позиционным параметрам переданным shell, то есть var будет принимать уже их значения

\textbf{BREAK и CONTINUE}

Вспоминая си, у нас есть возможность перейти к следующей итерации цикла с помощью continue и закончить, с помощью break. В shell есть похожие слова. 

Отличие BREAK в shell в том, что он  позволяет выходить не только из последнего цикла, в  котором он написан, но и из любого уровня вложенности циклов. 

BREAK N, где N --- количество уровней циклов, из которых мы выйдем.

Ключевое слово CONTINUE позволяет переходить заново к проверке условия (в начало цикла), то есть работает также как, например, в языке Си. 
